\documentclass[10pt]{article}
 
\usepackage[margin=1in]{geometry} 
\usepackage{amsmath,amsthm,amssymb, graphicx, multicol, array}
\usepackage[]{algorithmicx}
\usepackage{algpseudocode}
 
\newcommand{\N}{\mathbb{N}}
\newcommand{\Z}{\mathbb{Z}}

\setlength{\parindent}{0pt}
 
\newenvironment{Aufgabe}[2][Aufgabe]{\begin{trivlist}
\item[\hskip \labelsep {\bfseries #1}\hskip \labelsep {\bfseries #2.}]}{\end{trivlist}}

\begin{document}
 
\title{ \textbf{Numerische Mathematik - Übungsblatt \#1} }

\author{Amir Miri Lavasani (7310114, Gruppe 2), Bent Müller (7302332, Gruppe 2), \\ 
Johan Kattenhorn (7310602, Gruppe 7)} \maketitle

 
\begin{Aufgabe}{4}
	Implementierung der Vorwärts- und Rückwärtssubstitution
\end{Aufgabe}

\textbf{4.1} Algorithmus der Vorwärtssubstitution: Pseudocode \\

\begin{algorithmic}
	\State Input: 
		\begin{itemize}
			\item Untere Dreiecksmatrix \textbf{A} mit $det(A) \neq 0$,
			\item Vektor \textbf{b}. 
		\end{itemize} \\
	\State $x_1 := b_1/a_{11}$ \\
	\For{i in 2,...,n} 
		\State $x_i = (b_i-\sum_{j=1}^{i-1}a_{ij}*x_j)/a_{ii}$ 
	\EndFor \\

	\State Output: $\textbf{x} = (x_i)_{i=1}^{n}$

\end{algorithmic}

\end{document}
