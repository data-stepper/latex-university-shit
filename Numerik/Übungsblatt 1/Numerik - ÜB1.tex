\documentclass[10pt]{article}
 
\usepackage[margin=1in]{geometry} 
\usepackage{amsmath,amsthm,amssymb, graphicx, multicol, array}
 
\newcommand{\N}{\mathbb{N}}
\newcommand{\Z}{\mathbb{Z}}

\setlength{\parindent}{0pt}
 
\newenvironment{Aufgabe}[2][Aufgabe]{\begin{trivlist}
\item[\hskip \labelsep {\bfseries #1}\hskip \labelsep {\bfseries #2.}]}{\end{trivlist}}

\begin{document}
 
\title{ \textbf{Numerische Mathematik - Übungsblatt \#1} }

\author{Amir Miri Lavasani (7310114, Gruppe 6), Bent Müller (7302332, Gruppe 6), \\ 
Johan Kattenhorn (7310602, Gruppe 7)} \maketitle

 
\begin{Aufgabe}{3}
	Rechenaufwand
\end{Aufgabe}

\textbf{(a)}

\begin{proof}[Beweis]
	Anzugeben war der notwendige Rechenaufwand um die LR-Zerlegung der Matrix A 
	aus Aufgabe 2 durch elementare Zeilenumformungen zu realisieren.

	Hier muss ich nochmal die schritte durchgehen
\end{proof}

\textbf{(b)}

\begin{proof}[Beweis]
	Anzugeben war hier der Rechenaufwand den es braucht um das lineare Gleichungssystem
	$Ly = c$ zu lösen falls $L$ schon eine untere linke normierte Dreiecksmatrix ist.

	Wir kennen nun schon die Matrix $L$ aus der vorherigen Aufgabe:

	\begin{center}
		
	$L$ = 
	\begin{pmatrix}
		1 & 0 & 0 \\
		1 & 1 & 0 \\
		2 & 3 & 1 \\
	\end{pmatrix}

\end{center}
	Nun können wir das Gleichungssystem iterativ lösen:

	Aus der erste Gleichung wissen wir direkt: $c_1 = L_{1,1} \cdot y_1$
	
	Da die Matrix aber normiert ist gilt: $L_{1,1} = 1$ und somit auch $c_1 = y_1$

	Hierfrür braucht es also keine Rechenoperation.

	In der zweiten Gleichung steht dann: 
	$$c_2 = L_{2,1} \cdot y_1 + y_2 = L_{2,1} \cdot c_1 + y_2 \text{ durch umstellen erhalten wir: }
	y_2 = c_2 - L_{2,1} \cdot c_1$$

	Hier war der Rechenaufwand eine Multiplikation und eine Subtraktion.

	Nun können wir die dritte Gleichung lösen wie folgt:
	$$c_3 = L_{3,1} \cdot y_1 + L_{3,2} \cdot y_2 + y_3 \Rightarrow
	y_3 = c_3 - L_{3,1} \cdot y_1 - L_{3,2} \cdot y_2$$

	Hier benötigen wir also $2$ Multiplikationen und zwei Subtraktionen
	(oder auch eine Subtraktion und eine Addition je nach dem welches sich schneller
	implementieren lässt).
	
	Insgesamt brauchen wir hier also:
	$$3 \text{ Multiplikationen und } 3 \text{ Subtraktionen} \Rightarrow 6 \text{ Operationen also insgesamt.}$$

\end{proof} 

\textbf{(c)}

\begin{proof}[Beweis]
	Hier war der Rechenaufwand bei der Lösung des LGS $Rx = d$ wobei $R$ die obere rechte Dreiecksmatrix
	aus der LR-Zerlegung der Aufgabe 2 war. Bemerke dass diesmal die Matrix $R$ nicht normiert ist.

	\begin{center}
		$R = $
		\begin{pmatrix}
			1 & 4 & 5 \\
			0 & 2 & 6 \\
			0 & 0 & 3 \\
		\end{pmatrix}
	\end{center}

	Diesmal lösen wir von der letzten Reihe an auf, wir subsituieren also rückwärts.
	In der letzten Gleichung steht dann also:
	$$d_3 = 3 \cdot x_3 \Rightarrow y_3 = \frac{d_3}{3}$$
	Die Lösung dieser simplen Gleichung erfordert also schon eine Division.

	Dann steht in der 2. Gleichung:
	$$d_2 = 2 \cdot x_2 + 6 \cdot x_3 \Rightarrow x_2 = \frac{d_3 - 6 \cdot x_3}{2}$$
	Hier benötigen wir also eine Multiplikation, eine Subtraktion und eine Division.
	Die erste Gleichung sieht demnacht also wie folgt aus:
	$$d_1 = x_1 + 4 \cdot x_2 + 5 \cdot x_3 \Rightarrow x_1 = d_1 - 4 \cdot x_2 - 5 \cdot x_3$$
	Nun würde man hier eine Division mehr brauchen wenn $R_{1,1} \neq 1$, da wir uns aber unseren Spezialfall
	anschauen brauchen wir diese hier nicht. Also sind es hier 2 Multiplikationen und 2 Subtraktionen.

	Insgesamt kommen wir auf einen Rechenaufwand von:
	
	$$3 \text{ Multiplikationen, } 2 \text{ Divisionen und } 3 \text{ Subtraktionen} \Rightarrow \text{ insgesamt also }
	8 \text{ Operationen}$$
\end{proof}

\end{document} 
