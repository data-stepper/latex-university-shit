\documentclass[10pt]{article}
 
\usepackage[margin=1in]{geometry} 
\usepackage{amsmath,amsthm,amssymb, graphicx, multicol, array}
 
\newcommand{\N}{\mathbb{N}}
\newcommand{\Z}{\mathbb{Z}}

\setlength{\parindent}{0pt}
 
\newenvironment{Aufgabe}[2][Aufgabe]{\begin{trivlist}
\item[\hskip \labelsep {\bfseries #1}\hskip \labelsep {\bfseries #2.}]}{\end{trivlist}}

\begin{document}
 
\title{ \textbf{Mathematische Stochastik - Übungsblatt \#1} }

\author{Amir Miri Lavasani (7310114, Gruppe 6), Bent Müller (7302332, Gruppe 6), \\ 
Johan Kattenhorn (7310602, Gruppe 7)} \maketitle

 
\begin{Aufgabe}{1}
    Satz 1.10 - Folgerungen aus den Axiomen
\end{Aufgabe}

(b) \textit{ Behauptung: $ P\left( \bigcup\limits_{j=1}^{n} A_j \right) = \sum\limits_{j=1}^{n} P(A_j) $ }
\begin{proof}[Beweis]
\end{proof}

(c) \textit{Behauptung: $0 \leq P(A) \leq 1$}
\begin{proof}[Beweis]
\end{proof}

(d) \textit{Behauptung: $P(A^c) = 1 - P(A)$}
\begin{proof}[Beweis]
\end{proof}

(e) \textit{Behauptung: } Aus $A \subseteq B$ folgt $P(A) \leq P(B)$
\begin{proof}[Beweis]
\end{proof}

(g) \textit{Behauptung: $P(A \cup B) = P(B) + P(A) - P(A \cap B)$}
\begin{proof}[Beweis]
\end{proof}

\end{document}
