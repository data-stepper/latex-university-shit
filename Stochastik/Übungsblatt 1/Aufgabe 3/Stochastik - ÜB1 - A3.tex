\documentclass[10pt]{article}
 
\usepackage[margin=1in]{geometry} 
\usepackage{amsmath,amsthm,amssymb, graphicx, multicol, array}
 
\newcommand{\N}{\mathbb{N}}
\newcommand{\Z}{\mathbb{Z}}

\setlength{\parindent}{0pt}
 
\newenvironment{Aufgabe}[2][Aufgabe]{\begin{trivlist}
\item[\hskip \labelsep {\bfseries #1}\hskip \labelsep {\bfseries #2.}]}{\end{trivlist}}

\begin{document}
 
\title{ \textbf{Mathematische Stochastik - Übungsblatt \#1} }

\author{Amir Miri Lavasani (7310114, Gruppe 6), Bent Müller (7302332, Gruppe 6), \\ 
Johan Kattenhorn (7310602, Gruppe 7)} \maketitle

\begin{Aufgabe}{3}

\end{Aufgabe}

(a) \textit{Seien $A, B$ und $C$ drei Ereignisse in einem Warscheinlichkeitsraum, für die gelte: $P(A \cap (B \cup C)) = 0$.}
\textit{So ist zu zeigen, dass dann immer gilt: $P(A \cup B \cup C) = P(A) + P(B) + P(C) - P(B \cap C)$}
\begin{proof}[Beweis]
Wir verwenden ganz einfach die Siebformel für zwei Mengen, indem wir die Menge $(B \cup C)$ als eine Menge betrachten, wie folgt:

\begin{align*}
P(A \cup B \cup C) &= P(A \cup (B \cup C)) = P(A) + P(B \cup C) - \underbrace{ P(A \cap (B \cup C)) }_{\text{$= 0$ nach Vorraussetzung}} \\
&= P(A) + \underbrace{(P(B) + P(C) - (B \cap C)}_{\text{$= P(B \cup C)$ (Siebformel)}}
\end{align*}

\end{proof}

(b) \textit{Es war aus einer Stichprobe von $230$ Personen mit folgenden Angaben auszurechnen wie viele der Befragten sowohl Wein als auch Bier trinken.}

- $108$ trinken Wein,

- $167$ trinken Bier,

- $55$ trinken nichts von beiden
\begin{proof}[Beweis]
Uns fällt direkt auf, dass wir super den Aufgabenteil (a) verwenden können, indem wir einfach die Mengen wie folgt setzen.

\indent \text{ $A$ sind die nicht-trinker }

\indent \text{ $B$ sind die Bier-trinker }

\indent \text{ $C$ sind die Wein-trinker }

Und direkt sehen wir, dass $A \cap (B \cup C) = \emptyset$, da es ja nicht möglich ist Wein/Bier zu trinken und gleichzeitig ein nicht-trinker zu sein.

\begin{align*}
P(A \cup B \cup C) &= P(A) + P(B) + P(C) - P(B \cap C) \text{ $= 1$ (da das alle Befragten sind.)} \\
&\Rightarrow P(B \cap C) = P(A) + P(B) + P(C) - 1 \\
&= \frac{55}{230} + \frac{167}{230} + \frac{108}{230} - 1 \approx 0,43478 \ldots \; \hat{=}\; 43,478 \; \% \\
&\Rightarrow 230 \cdot 0,43478 \ldots \approx 99,99\ldots \approx 100
\end{align*}

Wohlgemerkt ist die Menge $(B \cap C)$ genau die gesuchten Befragten. Und die Brüche (z.B. $\frac{55}{230}$ hier die Warscheinlichkeit ein nicht-trinker zu sein) jeweils die Warscheinlichkeiten ($P(A)$, ...) aus der Gleichung darüber.

Es wird also erwartet, dass $100$ der Befragten sowohl Wein als auch Bier trinken.
\end{proof}

\end{document}
