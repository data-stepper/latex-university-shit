\documentclass[10pt]{article}
 
\usepackage[margin=1in]{geometry} 
\usepackage{amsmath,amsthm,amssymb, graphicx, multicol, array}
 
\newcommand{\N}{\mathbb{N}}
\newcommand{\Z}{\mathbb{Z}}

\setlength{\parindent}{0pt}
 
\newenvironment{Aufgabe}[2][Aufgabe]{\begin{trivlist}
\item[\hskip \labelsep {\bfseries #1}\hskip \labelsep {\bfseries #2.}]}{\end{trivlist}}

\begin{document}
 
\title{ \textbf{Mathematische Stochastik - Übungsblatt \#N} }

\author{Amir Miri Lavasani (7310114, Gruppe 6), Bent Müller (7302332, Gruppe 6), \\ 
Johan Kattenhorn (Mart. Nummer, Gruppe 7)} \maketitle

 
\begin{Aufgabe}{1}
Subject, z.B.: Satz 1.10 - Folgerungen aus den Axiomen
\end{Aufgabe}

(a) \textit{Behauptung: } 
\begin{proof}[Beweis]
Schreibe hier den Beweis

\begin{align*}
x &= y
\end{align*}

\end{proof}

(b) \textit{Behauptung: }
\begin{proof}[Beweis]
Schreibe hier den Beweis. 

\begin{align*}
x &= y
\end{align*}

\end{proof}

(c) \textit{Behauptung: }
\begin{proof}[Beweis]
Schreibe hier den Beweis. 

\begin{align*}
x &= y
\end{align*}

\end{proof}

\end{document}