\documentclass[a4paper]{article}

\usepackage[margin=1in]{geometry} 
\usepackage{amsmath,amsthm,amssymb, graphicx, multicol, array}


\pdfminorversion=7
\pdfsuppresswarningpagegroup=1

\newcommand{\R}{\mathbb{R}}
\newcommand{\N}{\mathbb{N}}
\newcommand{\Z}{\mathbb{Z}}
\newcommand{\beh}{\textit{Behauptung. }}

\setlength{\parindent}{0pt}
\newenvironment{Aufgabe}[2][Aufgabe]{\begin{trivlist}
\item[\hskip \labelsep {\bfseries #1}\hskip \labelsep {\bfseries #2.}]}{\end{trivlist}}

\begin{document}
	\begin{theorem} % Aufgabe #20
	\begin{Aufgabe}{20} % #20
		\textbf{a)} 
		Diskretes logistisches Wachstum
	\end{Aufgabe}

	\begin{proof}[Rechnung]
		Als erstes suchen wir die Gleichgewichtspunkte von unserem dynamischen System.
		Wir suchen also die Punkte die im nächsten Zeitschritt gleich bleiben, also gilt für diese:
		$x_{t+1} = x_{t}$. Diesen Punkten nennen wir im Folgenden einfach nur $x$.
		\begin{align*}
			x &= \alpha x (1 - x)
			= \alpha \left(
				x - x ^2
			\right) \\
			  & \Rightarrow x = 0 \text{ oder } 1 = \alpha ( 1 - x )
			  \Rightarrow x = 1 - \frac{ 1 }{ \alpha }
		\end{align*}
		Offensichtlich ist für uns nur die Lösung $x = 1 - \frac{ 1 }{ \alpha }$ interessant.
		Nun berechnen wir die Jacobian Matrix unserer Differenzengleichung, welche wie folgt aussieht.
		\begin{align*}
			J = \frac{ \partial  }{ \partial x } \left(
				\alpha (x - x ^2)
			\right) = \alpha \left(
				1 - 2 x
			\right) 
		\end{align*}
		Jetzt interessieren wir uns für die Jacobian aber nur genau am Gleichgewichtspunkt $x$, also setzen wir
		die Lösung welche wir für diesen im obigen gefunden haben in die Jacobi Matrix ein und erhalten:
		\begin{align*}
			J_{x ^{\star}} = \alpha \left(
				1 - 2 \left(
					1 - \frac{ 1 }{ \alpha }
				\right) 
			\right) = \alpha - 2 \alpha + 2 = 2 - \alpha
		\end{align*}
		Nun müssen wir die Eigenwerte der Jacobi Matrix bestimmen. Da diese aber in unserem Fall nur eine 1x1 Matrix
		ist, und die Eigenwerte immer genau die Nullstellen des charakteristischen Polynoms sind:
		$P_\lambda (J_{x ^{\star}}) = \det (J_{x ^{\star}} - \lambda \cdot I_1)$
		Nun sehen wir aber, dass die Eigenwerte allerdings schon genau in der obigen Gleichung stehen.
		Jetzt wissen wir, dass unser dynamisches System genau dann stabil ist wenn der Betrag dieser Eigenwerte
		kleiner als $1$ ist.
		\begin{align*}
			\text{System stabil} \Leftrightarrow
			| 2 - \alpha | < 1
			\overset{\alpha > 0} \Longleftrightarrow 1 < \alpha < 3
		\end{align*}
		Wir sehen also für die gegebenen Parameterbereiche:
		\begin{align*}
			\alpha \in (0, 1) & \Rightarrow \text{ System instabil } \\
			\alpha \in (1, 3) & \Rightarrow \text{ System stabil }
		\end{align*}
	\end{proof}
	\end{theorem}

	\begin{theorem} % Aufgabe #21
	\begin{Aufgabe}{21} % #21
		\textbf{b)} Räuber und Beute
	\end{Aufgabe}

	\begin{proof}[Rechnung]
		Wie bei Aufgabe 20 b) bestimmen wir zuerst die Gleichgewichtspunkte des Systems, also die Punkte
		welche sich im nächsten Zeitschritt nicht verändert haben. Hier müssen wir allerdings beachten,
		dass unser Zustandsraum zweidimensional ist. Wieder nennen wir die Lösungen an welche wir interessiert
		sind $p$ und $q$.
		\begin{align*}
			& \text{ (i) } p = p + r p \left(
				1 - \frac{ p }{ K }
			\right)  - s pq \\
			& \text{ (ii) } q = q - u q + v pq \\
			& \overset{(i)} \implies 0 = r p \left(
				1 - \frac{ p }{ K }
			\right) - s pq \overset{p \neq 0} \implies q = \frac{ r }{ s } \left(
				1 - \frac{ p }{ K }
			\right) \\
			& \overset{(ii)} \implies 0 = v pq - u q \overset{q \neq 0} \implies p = \frac{ u }{ v }
		\end{align*}
		Offensichtlich erhalten wir zusätzlich wieder die Lösung $(p, q) = (0,0)$ welche aber uninteressant
		für uns ist. Als nächstes berechnen wir die Jacobian Matrix von unserer Übergangsfunktion:
		$(p_{t+1}, q_{t+1}) = F(p_{t}, q_{t})$. 
		\begin{align*}
			J &= \begin{pmatrix} 
				\frac{ \partial p_{t+1} }{ \partial p_{t} } & \frac{ \partial p_{t+1} }{ \partial q_{t} } \\
				\frac{ \partial q_{t+1} }{ \partial p_{t} } & \frac{ \partial q_{t+1} }{ \partial q_{t} }
			\end{pmatrix} 
				= \begin{pmatrix} 
					1 + r - \frac{ 2 rp }{ K } - sq & - sp \\
					vq & 1 - u + vp
				\end{pmatrix} 
		\end{align*}
		Nun setzen wir die Gleichgewichtspunkte in die Jacobian Matrix ein und erhalten:
		\begin{align*}
			J_{x ^{\star}} &= \begin{pmatrix} 
				1 + r \left(
					1 - \frac{ 2u }{ vK }
				\right) - r \left(
					1 - \frac{ u }{ vK }
				\right) & \frac{ -su }{ v } \\
				\frac{ vr }{ s } (1 - \frac{ u }{ vK })
						& 1 \\
			\end{pmatrix} \\
				   &= \begin{pmatrix} 
					   1 - \frac{ ur }{ vK } & \frac{ -su }{ v } \\
					   \frac{ vr }{ s } ( 1 - \frac{ u }{ vK } ) & 1 \\
				   \end{pmatrix} 
		\end{align*}
		Nun erkennen wir, dass wir eine 2x2 Matrix haben und verwenden das Kriterium welches wir aus 21a wissen.
		Dieses vereinfacht uns die Aussage, dass der Betrag der Eigenwerte kleiner als $1$ ist und welches
		aber genau besagt, ob unser dynamisches System stabil ist oder nicht.
		\begin{align*}
			\text{System stabil} & \Leftrightarrow
			| - \text{Spur} (J_{x ^{\star}}) | < \det (J_{x ^{\star}}) + 1 < 2 \\
				 & \Leftrightarrow \Big | -2 + \frac{ ur }{ vK } \Big |
				 < \left(
				 	1 - \frac{ ur }{ vK }
				\right) \cdot (1) + \frac{ su }{ v } \cdot \frac{ vr }{ s } \left(
					1 - \frac{ u }{ vK }
				\right) + 1 < 2 \\
				 & \Leftrightarrow \Big | 2 - \frac{ ur }{ vK } \Big |
				 < 2 + ur - (1 + u) \frac{ ur }{ vK } < 2 \\
		\end{align*}
		Wir erhalten also ein schönes Kriterium für die Parameter $u, r, v, K$\, , welches uns sagt
		wann unser dynamisches System stabil ist.
	\end{proof}
	\end{theorem}
\end{document}

