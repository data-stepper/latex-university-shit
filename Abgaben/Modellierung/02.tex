\documentclass[a4paper]{article}
 
\usepackage[margin=1in]{geometry} 
\usepackage{amsmath,amsthm,amssymb, graphicx, multicol, array}
 
\newcommand{\R}{\mathbb{R}}
\newcommand{\N}{\mathbb{N}}
\newcommand{\Z}{\mathbb{Z}}
\newcommand{\beh}{\textit{Behauptung. }}

\setlength{\parindent}{0pt}
 
\newenvironment{Aufgabe}[2][Aufgabe]{\begin{trivlist}
\item[\hskip \labelsep {\bfseries #1}\hskip \labelsep {\bfseries #2.}]}{\end{trivlist}}
\newenvironment{amatrix}[1]{%
  \left(\begin{array}{@{}*{#1}{c}|c@{}}
}{%
  \end{array}\right)
}

\begin{document}
 
\title{ \textbf{Mathematische Modellierung - Übungsblatt \#2} }

\author{Amir Miri Lavasani (7310114), Bent Müller (7302332)} \maketitle

 
\begin{Aufgabe}{4}
    (Potenzreihen-Lösung einer Diff.gleichung)
\end{Aufgabe}

\textbf{(a)} Das Anfangswert-Problem ist 
  \begin{align*}
    y^\prime(t) = -\kappa(y(t) - y_a), \quad y(0) = 2.
  \end{align*}
  
  Wir verwenden den Potenzreihen-Ansatz $y(t) = \sum_{k=0}^{\infty} c_k t^k$. Die Koeffizienten $c_k$ sollen 
  für $k=0,\dots,3$ bestimmt werden. Die Potenzreihe von $y^\prime(t)$ ist gegeben durch 
  \begin{align*}
    y^\prime(t) = \sum_{k=0}^{\infty} c_{k+1}(k+1) t^k
  \end{align*}

  Durch Umformung der Differentialgleichung nach $y(t)$ und einsetzen der Potenzreihen erhalten wir:
  \begin{align*}
    \sum_{k=0}^{\infty} c_k t^k = -\frac{1}{\kappa} \sum_{k=0}^{\infty} c_{k+1}(k+1) t^k + y_a
  \end{align*}

  Schreiben wir die Summen bis $k=3$ aus erhalten wir:
  \begin{align*}
    c_0 + c_1t + c_2t^2 + c_3t^3 = -\frac{1}{\kappa}c_1 - \frac{2}{\kappa}c_2t 
                                   - \frac{3}{\kappa}c_3t^2 - \frac{4}{\kappa}c_4t^3 + y_a  \qquad(\star)
  \end{align*}

  Durch Koeffizientenvergleich folgt:
  \begin{itemize}
    \item[-] $c_0 = y_a - \frac{1}{\kappa}c_1 \quad\Rightarrow\quad c_1 = \kappa(y_a - c_0)$
    \item[-] $c_1 = -\frac{2}{\kappa}c_2 \quad\Rightarrow\quad c_2 = -\frac{\kappa}{2} c_1$
    \item[-] $c_2 = -\frac{3}{\kappa} c_3 \quad\Rightarrow\quad c_3 = -\frac{\kappa}{3}c_2$ 
  \end{itemize}

  Einsetzen in $(\star)$ liefert
  \begin{align*}
    y(t) &= c_0 + \kappa(y_a - c_0) t - \frac{\kappa}{2} c_1 t^2 - \frac{\kappa}{3}c_2 t^3. \qquad (\star\star) \\
  \end{align*}

  Aus $y(t) = 2$ folgt $c_0 = 2$. Daraus folgt
  \begin{itemize}
    \item[-] $c_1 = \kappa(y_a - 2)$
    \item[-] $c_2 = -\frac{\kappa}{2} (\kappa (y_a - 2)) = -\frac{k^2}{2}(y_a - 2)$ 
    \item[-] $c_3 = -\frac{\kappa}{3}(-\frac{k^2}{2}(y_a - 2)) = \frac{\kappa^3}{6}(y_a - 2)$
  \end{itemize}

  Einsetzen in $(\star\star)$ liefert schließlich 
  \begin{align*}
    y(t) &= 2 + \kappa(y_a - 2)t - \frac{k^2}{2}(y_a - 2)t^2 + \frac{\kappa^3}{6}(y_a - 2)t^3 \\
         &= 2 + (y_a - 2) \left( \kappa t - \frac{k^2}{2} t^2 + \frac{\kappa^3}{6} t^3 \right). \qquad 
  \end{align*}
  
  \newpage 

  \textbf{(b)} Seien $\kappa = \frac{1}{4}, y_a = 1$. Zunächst bestimmen wir die Pade-Approximation mit $m=n=1$. 
    Die gesuchte rationale Funktion hat die Form 
    \begin{align*}
      R_{1,1}(t) = \frac{a_0 + a_1t}{b_0 + b_1t},
    \end{align*}

    wobei $b_0 = 1$.
    Um die restlichen Koeffizienten nach der Pade-Approximation zu bestimmen, muss folgendes Gleichungssystem gelöst werden
    \begin{enumerate}
      \item[(1)] $\sum_{j=0}^{k} b_j c_{k-j} = a_k$. \quad $(k = 0,1)$
      \item[(2)] $\sum_{j=0}^{k} b_j c_{1-j+k} = 0$. \quad $(k = 1)$ 
    \end{enumerate}

    Aus (1) erhalten wir folgende Gleichungen: 
    \begin{align*}
      b_0\cdot c_0 &= a_0 \iff a_0 = 2    \\
      b_0\cdot c_1 + b_1\cdot c_0 &= a_1 \iff a_1 = -\frac{1}{4} + 2b_1   \\
    \end{align*}

    Aus (2) erhalten wir: 
    \begin{align*}
      b_0\cdot c_2 + b_1\cdot c_1 = 0 \iff \frac{1}{32} - \frac{1}{4}b_1 = 0 \iff b_1 = \frac{1}{8}
    \end{align*}

    Durch einsetzen erhalten wir schließlich:
    \begin{align*}
      a_1 = -\frac{1}{4} + 2\left(\frac{1}{8}\right) = 0.
    \end{align*}

    Somit gilt 
    \begin{align*}
      R_{1,1}(t) = \frac{2}{1+\frac{1}{8}t}\,.
    \end{align*}

    Nun seien $m=2, n=1$. Die zu bestimmende rationale Funktion hat die Gestalt
    \begin{align*}
      R_{2,1}(t) = \frac{a_0 + a_1t + a_2t^2}{b_0 + b_1t},
    \end{align*}

    wobei wir wieder $b_0 = 1$ setzen. Folgendes Gleichungssystem muss gelöst werden:
    \begin{enumerate}
      \item[(1)] $\sum_{j=0}^{k} b_j c_{k-j} = a_k$. \quad $(k = 0,1,2)$
      \item[(2)] $\sum_{j=0}^{k} b_j c_{2-j+k} = 0$. \quad $(k = 1)$
    \end{enumerate}

    Aus (1) erhalten wir folgende Gleichungen:
    \begin{align*}
      b_0c_0 &= a_0 \iff 2b_0 = a_0 \iff a_0 = 2   \\
      b_0c_1 + b_1c_0 &= a_1 \iff a_1 = c_1 + b_1c_0 \iff a_1 = -\frac{1}{4} + 2b_1 \qquad (\star) \\
      b_0c_2 + b_1c_1 + b_2c_0 &= a_2 \iff a_2 = 2\cdot \frac{2}{32} - \frac{1}{4}b_1 + 2\cdot 0
                                      \iff a_2 = \frac{1}{16} - \frac{1}{4}b_1 \qquad (\star\star)
    \end{align*}

    Aus (2) erhalten wir:
    \begin{align*}
      b_0c_3 + b_1c_2 = 0 \iff -\frac{1}{384} + \frac{1}{32}b_1 = 0 \iff b_1 = \frac{1}{12}.
    \end{align*}

    Einsetzen in $(\star)$ liefert 
    \begin{align*}
      a_1 = -\frac{1}{4} + 2\cdot \frac{1}{12} = -\frac{1}{12}
    \end{align*}

    und schließlich erhalten wir durch einsetzen in $(\star\star)$
    \begin{align*}
      a_2 &= \frac{1}{16} - \frac{1}{4}\cdot \frac{1}{12}  \\
          &= \frac{1}{24}.
    \end{align*}

    Insgesamt folgt
    \begin{align*}
      R_{2,1}(t) = \frac{2 - \frac{1}{12}t + \frac{1}{24}t^2}{1 + \frac{1}{12}t}\,.
    \end{align*}
\end{document}