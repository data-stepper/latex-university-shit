\documentclass[a4paper]{article}
 
\usepackage[margin=1in]{geometry} 
\usepackage{amsmath,amsthm,amssymb, graphicx, multicol, array}
 
\newcommand{\R}{\mathbb{R}}
\newcommand{\N}{\mathbb{N}}
\newcommand{\Z}{\mathbb{Z}}
\newcommand{\beh}{\textit{Behauptung. }}

\setlength{\parindent}{0pt}
 
\newenvironment{Aufgabe}[2][Aufgabe]{\begin{trivlist}
\item[\hskip \labelsep {\bfseries #1}\hskip \labelsep {\bfseries #2.}]}{\end{trivlist}}
\newenvironment{amatrix}[1]{%
  \left(\begin{array}{@{}*{#1}{c}|c@{}}
}{%
  \end{array}\right)
}

\begin{document}
 
\title{ \textbf{Mathematische Modellierung - Übungsblatt \#1} }

\author{Amir Miri Lavasani (7310114), Bent Müller (7302332)} \maketitle

\begin{Aufgabe}{1}
Medikamenten-Aufnahme im Körper
\end{Aufgabe}

\textbf{a)} Wir wissen direkt, da 4 Bedingungen gegeben sind, dass die Summe der Grade beider Polynome 3 
sein muss. Also wählen wir für den Zähler ein Polynom 3. Grades und für den
Zähler das konstante 1 Polynom. Wir erhalten folgende Funktionen:
\[
	p(x) = a_3 x^{3} + a_2 x ^{2} + a_1 x + a_0 ,\;\;\; 
	p'(x) = 3 a_3 x ^{2} + 2 a_2 x + a_1 ,\;\;\; 
	p''(x) = 6 a_3 x + 2 a_2
.\] 

Jetzt erhalten wir folgende Gleichungen:

\[
	p(0) = 0,\;\;\; 
	p(18) = 15,\;\;\; 
	p(480) = 2,\;\;\; 
	p'(18) = 0,\;\;\; 
	p''(18) < 0,
\] 

Zuerst bemerken wir, dass $a_0 = 0$ (nach Gleichung 1) und wir
können ebenfalls unsere Bedingung aus der Ableitung benutzen. Also
stellen wir folgendes lineares Gleichungssystem auf in Form einer erweiterten Koeffizientenmatrix
mit Einträgen:

\begin{align*}
	& \left(\begin{array}{ccc|c}  
		18 ^3 & 18 ^2 & 18 & 15\\
		480 ^3 & 480 ^2 & 480 & 2\\
		3 \cdot 18 ^2 & 2 \cdot 18 & 1 & 0\\
	\end{array}\right)
	\rightsquigarrow 
	\left(\begin{array}{ccc|c}  
		5832 & 324 & 18 & 15\\
		110592000 & 230400 & 480 & 2\\
		972 & 36 & 1 & 0\\
	\end{array}\right) \\
	&\rightsquigarrow 
	\left(\begin{array}{ccc|c}  
		1 & \frac{ 1 }{ 480 } & \frac{ 1 }{ 480 ^2 } & \frac{ 2 }{ 480 ^3 }\\
		5832 & 324 & 18 & 15\\
		972 & 36 & 1 & 0\\
	\end{array}\right)
	\rightsquigarrow
	\left(\begin{array}{ccc|c}  
		1 & \frac{ 1 }{ 480 } & \frac{ 1 }{ 480 ^2 } & \frac{ 2 }{ 480 ^3 }\\
		0 & \frac{ 6237 }{ 20 } & \frac{ 57519 }{ 3200 } & \frac{ 3839973 }{ 256000 } \\
		972 & 36 & 1 & 0\\
	\end{array}\right)
\end{align*}

Jetzt können wir schon erkennen, dass alle Zeilen unserer quadratischen Koeffizientenmatrix linear unabhängig
sind und, somit eine eindeutige Lösung für unsere Koeffizienten $a_3, a_2, a_1, \text{ und } a_0$ existiert. \\
\phantom{a} \\
\textbf{b)} Hier ging es um die selben Datenpunkte welche aber mit Hilfe einer exponentiellen Funktion
approximiert werden sollen.

\[
	g(x) = \theta_1 x e ^{-( \theta_2 x + \theta_3 x ^{ 2 } )},\;\;\; 
	g'(x) = \theta_1 \left( 
		1 - \theta_2 x - 2 \theta_3 x ^2
	\right) e ^{-( \theta_2x + \theta_3 x ^2 )}
\] 

Nach Aufgabenstellung haben wir wieder folgende Bedingungen gegeben:

\[
	g(0) = 0, \;\;\;
	g(18) = 15,\;\;\;
	g(480) = 2,\;\;\;
	g'(18) = 0,\;\;\;
	g''(18) < 0,
\] 

Direkt sehen wir, dass $\theta_1 \neq 0$, da wir sonst die konstante 0-Funktion erhalten
welche keine der anderen Bedingungen erfüllen kann.
Zuerst bemerken wir, dass Bedingung \#1 keine richtige Bedingung ist, da diese für alle
Parameter $\theta_1, \theta_2, \theta_3$ gilt. Wir fangen an mit Bedingung 4, welche
wir aus der Ableitung wissen:

\begin{align*}
	g'(x) = 0 &\Leftrightarrow \theta_1 \left( 
		1 - \theta_2 x - 2 \theta_3 x ^2
	\right) = 0 \text{ ( da $e ^{a} \neq 0 $ für alle $a$ )} \\
			   &\Leftrightarrow
		1 - 2 \theta_3 x ^2
	 = \theta_2 x
	 \overset{ \text{setze } x=18 } \Longleftrightarrow
	 \theta_2 = \frac{ 1 }{ 18 } - 36 \; \theta_3 \;\;\;\;\; \qquad \hspace{1cm} \text{ ( I ) }
	 \\ \\
		g(18) = 15 &= 18 \theta_1 e ^{ - (18 \theta_2 + 324 \theta_3) }
		\Leftrightarrow \ln \left( 
			\frac{ 15 }{ 18 \theta_1 }
		\right) = - \left( 
		18 \theta_2 + 324 \theta_3
	\right) \text{ benutze ( I ) } \\
				   &= - \left( 
					   18 \left( 
						   \frac{ 1 }{ 18 } - 36 \; \theta_3
					   \right) 
					   + 324 \theta_3
				   \right) 
				   = \ln \left( \frac{ 15 }{ 18 } \right)  - \ln (\theta_1)
\end{align*}

Diese Gleichung stellen wir jetzt nach $\theta_1$ um und erhalten:

\begin{align*}
	& \theta_1 = \frac{ 15 }{ 18 } \; e ^{ 1 - 324\; \theta_3 }
	\qquad \qquad \text{ (II) }
	\\ \\
	g(480) = 2 &= 480 \; \theta_1 e ^{ - \left( 
			\frac{ 80 }{ 3 } + 213120 \theta_3
	\right)  }
	\text{ direkt nach ( I ) } \\
			   &= 400 \; e ^{ 1 - 324 \theta_3 }
			   e ^{ - (\frac{ 80 }{ 3 } + 213120 \theta_3)}
			   \text{ nach ( II ) } \\
	&= 400 e ^{ - ( \frac{ 77 }{ 3 } + 213444 \theta_3 ) } = 2 \\
	&\Leftrightarrow \frac{ 77 }{ 3 } + 213444 \theta_3 = \ln \left( 
		200
	\right)  \\
	&\Leftrightarrow \theta_3 = \frac{ \ln (200) - \frac{ 77 }{ 3 } }{ 
		213444
	}
	\approx -0,00009542713 \\
	&\Rightarrow \theta_2 \approx 0,0589909324\\
	&\Rightarrow \theta_1 \approx 2,336366243 \text{ (beide durch einsetzen) }
\end{align*}
 
\begin{Aufgabe}{2}
    Wunschgeschwindigkeit, Populationsverlauf
\end{Aufgabe}

\textbf{a)} \beh Die rationale Funktion $V_{2,2}(x) = \frac{x^2}{x^2 + 1}$ erfüllt die Bedingungen (i)-(vi).
\begin{proof}[Beweis]
    Die Bedingungen (i), (iv) und (v) sind klar. Für (ii): 
    \begin{align*}
        \lim_{x\to\infty} V_{2,2}(x) = 1. 
    \end{align*}

    Der Grenzwert existiert also und ist invertierbar. 
    Bedingung (iii) besagt: Für $x_0\in\R$ gilt
    \begin{align*}
        V_{2,2}^{\prime\prime}(x_0) = 0 \quad\Longrightarrow\quad V_{2,2}^{\prime\prime\prime}(x_0) \neq 0.
    \end{align*}

    Wir berechnen zunächst $V_{2,2}^{\prime}$: 
    \begin{align*}
        V_{2,2}^{\prime}(x) = \frac{2x(x^2+1)-2x^3}{(x^2+1)^2} = \frac{2x}{(x^2+1)} \,. \\
    \end{align*}

    Nun $V_{2,2}^{\prime\prime}$:
    \begin{align*}
        V_{2,2}^{\prime\prime}(x) = \frac{2(x^2+1)^2 - 2x\cdot 2 (x^2+1)\cdot 2x}{(x^2+1)^4} 
                                  = \frac{2-6x^2}{(x^2+1)^3} \,. 
    \end{align*}
    
    Und schließlich $V_{2,2}^{\prime\prime\prime}$:
    \begin{align*}
        V_{2,2}^{\prime\prime\prime}(x) = \frac{-6(x^2+1)^3 - (2-6x^2)\cdot 3(x^2 + 1)\cdot 2x}{(x^2+1)^6} 
                                        = \frac{24x(x^2-1)}{(x^2+1)^4} \,.
    \end{align*}

    Die Nullstellen von $V_{2,2}^{\prime\prime}$ sind gegeben durch:
    \begin{align*}
        2 - x_0^2 = 0 \Longleftrightarrow x_0 = \pm\frac{1}{\sqrt{3}} \,.
    \end{align*}

    Setzen wir $x_0$ in $V_{2,2}^{\prime\prime\prime}$ ein erhalten wir:
    \begin{align*}
        V_{2,2}^{\prime\prime\prime}(x_0) = \pm \frac{24}{\sqrt{3}}\left(\frac{1}{3} - 1\right) \neq 0.
    \end{align*}

    Also erfüllt $V_{2,2}$ auch die Bedingung (iii). \\
    Bedingung (vi) ist auch erfüllt, denn setzen wir einen der drei Koeffizienten in $V_{2,2}$ auf Null, 
    so würde damit einer oder mehrere der anderen Bedingungen verletzt werden: 
    \begin{enumerate}
        \item[(1)] Setze den Koeffizienten von $x^2$ im Zähler auf Null $\Longrightarrow$ Bedingung (ii) wird verletzt (u.a.),
        \item[(2)] Setze den Koeffizienten von $x^2$ im Nenner auf Null $\Longrightarrow$ Bedingung (v) wird verletzt (u.a.),
        \item[(3)] Setze die $1$ im Nenner auf Null $\Longrightarrow$ Bedingung (iv) wird verletzt (u.a.).
    \end{enumerate}

    Somit hat $V_{2,2}$ die 
    geringste Anzahl an Koeffizienten die eine rationale Funktion haben kann, um diese 6 Bedingungen zu erfüllen.
\end{proof}

\textbf{b)} Im Folgenden konstruieren die Menge von Funktionen die Bedingungen (i)-(v) erfüllen:
\begin{proof}[Beweis]
	Wir können aus den Bedingungen (ii) und (v) direkt die Form unserer Funktionen ablesen. Denn aus (ii)
	folgt direkt, dass die Grade der jeweiligen Polynome gleich sein muss, da die Funktion ansonsten
	divergieren bzw. gegen $0$  konvergieren würde. Jetzt wissen wir also mit (v), dass auch $n = 4$ sein
	muss. Also sieht unsere Funktion wie folgt aus:
	\[
		R_{m, n} (t) = \frac{ 
			350 \cdot b_4 x ^{4} + a_3 x ^{3} + a_2 x ^2 + a_1 x + a_0
			}{ 
			b_4 x ^{4} + b_3 x ^3 + b_2 x ^2 + b_1 x + b_0
		}
		= \frac{ p(x) }{ q(x) }
	\] 

	Außerdem wissen wir auch aus (ii) dass $a_4 = 350 \cdot b_4$ welche wir hier schon direkt eingesetzt haben.
	Des weiteren sehen wir auch sofort, dass aus Bedingung (i) direkt $a_0 = 2$ folgt.
	Nun erkennen wir für die weiteren Bedingungen, dass diese alle genau dann erfüllt sind, wenn jeweils 
	der Zähler unserer rationalen Funktion jeweils gleich 0 ist.
	\begin{align*}
		R_{m, n}'(0) &= 0 \Leftrightarrow p'(0)q(0) = p(0)q'(0) \text{ (nach Quotientenregel)} \\
				  &\Leftrightarrow (a_1)(b_0) = (a_0)(b_1)
				  \Leftrightarrow a_1 = 2 b_1
	\end{align*}
	Nun wenden wir die selbe Strategie auf die zweite Ableitung an:
	\begin{align*}
		R_{m, n} '' (0) &= 0 \Leftrightarrow \left(
				p'(0)q(0) - p(0)q'(0)
			\right) ' q ^2 (0) = \left(
				p'(0)q(0) - p(0)q'(0)
			\right) (2 q' (0)) \\
			&\Leftrightarrow \left(
				p'(0)q(0) - p(0)q'(0)
			\right) ' = 0
			\Leftrightarrow \left(
				p'(0)q(0)
			\right) ' = \left(
				p(0)q'(0)
			\right) ' \\
			&\Leftrightarrow {
				p''(0)q(0) + p'(0)q'(0)
			} = {
				p'(0)q'(0) + p(0)q''(0)
			} \\
			& \Leftrightarrow 2 a_2 + a_1 b_1 = a_1 b_1 + a_0 b_2
			\Rightarrow a_2 = b_2
	\end{align*}
	Im oberen haben wir benutzt, dass in der ersten Gleichung der rechte Faktor $p'(0)q(0) - p(0)q'(0)$
	nach der vorherigen Aufgabe 0 war und somit das gesamte Produkt auf der rechten Seite 0 wird.
	Bei der dritten Ableitung gehen wir analog vor:
	\begin{align*}
		R_{m, n} '''(0) &= 0 \Leftrightarrow \left(
			\left(
				p'(0)q(0) - p(0)q'(0)
			\right) ' q ^2 (0) - \left(
				p'(0)q(0) - p(0)q'(0)
			\right) (2 q' (0))
		\right) ' q ^{4} (0) \\
		&= \left(
			\left(
				p'(0)q(0) - p(0)q'(0)
			\right) ' q ^2 (0) - \left(
				p'(0)q(0) - p(0)q'(0)
			\right) (2 q' (0))
		\right) \cdot 4 q ^3 (0) \\
		& \Leftrightarrow \left(
			\left(
				p'(0)q(0) - p(0)q'(0)
			\right) ' q ^2 (0) - \left(
				p'(0)q(0) - p(0)q'(0)
			\right) (2 q' (0))
		\right) ' = 0 \\
		&\Leftrightarrow
		\left(
			\left(
				p'(0)q(0) - p(0)q'(0)
			\right) ' q ^2 (0)
		\right) '
		=  \left(
			p'(0)q(0) - p(0)q'(0)
			(2 q' (0)) \;
		\right) ' \\
		&\Leftrightarrow 
			\left(
				p'(0)q(0) - p(0)q'(0)
			\right) '' q ^2 (0) + 
			\left(
				p'(0)q(0) - p(0)q'(0)
			\right) ' 2 q' (0) \\
		&= 
			\left(
				p'(0)q(0) - p(0)q'(0)
			\right) ' 2 q' (0) + 
			\left(
				p'(0)q(0) - p(0)q'(0)
			\right) 2 q'' (0) \\
		&\Leftrightarrow 
			\left(
				p'(0)q(0) - p(0)q'(0)
			\right) '' 1 ^2 = 0
		\Leftrightarrow 
		\left(
			p'(0)q(0) 
		\right) ''
			=
		\left(
			p(0)q'(0)
		\right) '' \\
		&\Leftrightarrow 
			\left(
			p''(0)q(0) + p'(0)q'(0)
			\right) '
		=
			\left(
			p'(0)q'(0) + p(0)q''(0)
			\right) ' \\
		&\Leftrightarrow 
		{
			p'''(0)q(0) + p''(0)q'(0)
		}
		=
		{
			p'(0)q''(0) + p(0)q'''(0)
		} \\
		&\Leftrightarrow
			6 a_3 + 2 a_2 b_1 = a_1 2 b_2 + 2 \cdot 6 b_3
		\Leftrightarrow
			6 a_3 + 2 a_2 b_1 = 4 b_1 a_2 + 2 \cdot 6 b_3
	\end{align*}
\end{proof}


\begin{Aufgabe}{3}
    Rationale Interpolation
\end{Aufgabe}

\textbf{a)} \beh Die Aufgabe die Stützstellen $(-1; 2), (1; 3)$ und $(2; 3)$ durch eine rationale Funktion
mit $m = 1 = n$ zu interpolieren, besitzt keine Lösung.

\begin{proof}[Beweis]

    Wir betrachten die rationale Funktion $V_{1,1}(x) = \frac{a_0 + a_1\,x}{b_0 + b_1\,x}$\,. Es soll gezeigt werden,
    dass keine solche Funktion existiert, welche die Eigenschaften
    \begin{itemize}
        \item[(a)] $V_{1,1}(-1) =  2$,
        \item[(b)] $V_{1,1}(1) =  3$,
        \item[(c)] $V_{1,1}(2) =  3$.
    \end{itemize}

    erfüllt. Um dies zu tun, nehme an, dass $V_{1,1}$ die Eigenschaften (a) und (b) erfüllt. Nun wird gezeigt, dass diese Funktion dann
    nicht Eigenschaft (c) erfüllen kann. Es gibt drei Möglichkeiten für die Steigung von $V_{1,1}$ im Punkt $(1;3)$:
    \begin{itemize}
        \item[(1)] $V_{1,1}^\prime(1) = 0$,
        \item[(2)] $V_{1,1}^\prime(1) > 0$,
        \item[(3)] $V_{1,1}^\prime(1) < 0$.
    \end{itemize}

    Betrachte Möglichkeit (1). Dafür berechnen wir zunächst die Ableitung von $V_{1,1}$ mit der Quotientenregel:
    \begin{align*}
        V_{1,1}^\prime(x) \quad&=\quad \frac{a_1x\cdot (b_0 + b_1x) - (a_0 + a_1x)b_1}{(b_0 + b_1x)^2} \\\\
                          \quad&=\quad \frac{a_1b_1x + a_1b_1x^2 - a_0b_1 - a_1b_1x^2}{(b_0 + b_1x)^2} 
                          \quad=\quad \frac{a_1b_0 - a_0b_1}{(b_0 + b_1x)^2}.
    \end{align*}

    Nun bestimmen wir die Nullstellen von $V_{1,1}^\prime$:
    \begin{align*}
        0 &= V_{1,1}^\prime(x)                      \\
          &= \frac{a_1b_0 - a_0b_1}{(b_0 + b_1x)^2} \\
          &\Leftrightarrow a_1b_0 - a_0b_1 = 0.
    \end{align*}

    Daraus folgt, dass die Nullstellen der Ableitung nicht von $x$ abhängen. Hat $V_{1,1}^\prime$ also \textit{eine} 
    Nullstelle, so ist sie bereits \textit{überall} konstant $0$. Das würde aber bedeuten, dass $V_{1,1}$ konstant wäre, was aber
    nicht sein kann, denn $V_{1,1}$ geht n.V. durch die Punkte $(-1; 2)$ und $(1; 3)$. Möglichkeit (1) trifft also nicht zu.\par
    Betrachten wir nun Möglichkeit (2). Ist $V_{1,1}^\prime(1) > 0$, so ist die Steigung von $V_{1,1}$ an der Stelle $1$ positiv.
    Das heißt also, es gibt $\epsilon>0$ und $\delta>0$, sodass $V_{1,1}(1+\epsilon) = V_{1,1}(1) + \delta = 3 + \delta$.
    Um Eigenschaft (c) zu erfüllen, muss $V_{1,1}$ also wieder fallen. Das bedeutet $V_{1,1}^\prime$ muss auf dem Intervall 
    $(1,2)$ negativ werden. Da $V_{1,1}^\prime$ aber stetig ist, folgt aber aus dem Zwischenwertsatz, dass es ein $x_0\in (1,2)$ geben muss, sodass 
    $V_{1,1}^\prime(x_0) = 0$. Wir haben aber eben bereits gezeigt, dass dies nicht möglich ist. Also trifft auch Eigenschaft (b) nicht zu.
    Analog folgt, dass auch Eigenschaft (c) nicht zutrifft. Somit folgt die Behauptung.
\end{proof}

\textbf{b)} Wir berechnen die Koeffizienten von $V_{0,2}(x) = \frac{a_0}{b_0+b_1x+b_2x^2}$\,, sodass 
    \begin{itemize}
        \item[(a)] $V_{0,2}(-1) = 2$, 
        \item[(b)] $V_{0,2}(1) = 3$,
        \item[(c)] $V_{0,2}(2) = 3$. 
    \end{itemize}

    Aus den Voraussetzungen (a)-(c) erhalten wir folgende Gleichungen:
    \begin{itemize}
        \item[(1)] $\frac{a_0}{b_0-b_1+b_2} = 2  \Longleftrightarrow  b_0 - b_1 + b_2 = \frac{1}{2}a_0$,
        \item[(2)] $\frac{a_0}{b_0+b_1+b_2} = 3  \Longleftrightarrow  b_0 + b_1 + b_2 = \frac{1}{3}a_0$,
        \item[(3)] $\frac{a_0}{b_0+2b_1+4b_2} = 3 \Longleftrightarrow b_0 + 2b_1 + 4b_2 = \frac{1}{3}a_0$.  
    \end{itemize}

    Wir haben also drei linear unabhängige Gleichungen und $4$ Variablen. Also besitzt das Gleichungssystem einen 
    Freiheitsgrad. Diesen setzen wir als $a_0$ fest. Nun überführen wir das LGS in eine erweiterte Koeffizientenmatrix und
    lösen diese:
    \begin{align*}
        \begin{amatrix}{3}
            1 & -1 & 1 & \frac{a_0}{2} \\
            1 & 1  & 1 & \frac{a_0}{3} \\
            1 & 2  & 4 & \frac{a_0}{3}
        \end{amatrix}
        &\stackrel{Z2-Z1, Z3-Z1}{\rightsquigarrow}  
        \begin{amatrix}{3}
            1 & -1 &  1 & \frac{a_0}{2}  \\
            0 &  2 &  0 & -\frac{a_0}{6} \\
            0 &  3 &  3 & -\frac{a_0}{6}
        \end{amatrix} \\
        &\stackrel{Z2/2, Z3/3}{\rightsquigarrow} \hspace*{0.6cm}
        \begin{amatrix}{3}
            1 & -1 &  1 & \frac{a_0}{2}   \\
            0 &  1 &  0 & -\frac{a_0}{12} \\
            0 &  1 &  1 & -\frac{a_0}{18} 
        \end{amatrix} 
        \stackrel{Z3-Z2, Z1+Z2, Z1-Z3}{\rightsquigarrow}
        \begin{amatrix}{3}
            1 &  0 &  0 & \frac{7a_0}{18} \\
            0 &  1 &  0 & -\frac{a_0}{12} \\
            0 &  0 &  1 & \frac{a_0}{36} 
        \end{amatrix}.
    \end{align*}     

    Die gesuchte rationale Interpolante ist somit gegeben durch
    \begin{align*}
        V_{0,2}(x) &= \frac{a_0}{\frac{7a_0}{18} -\frac{a_0}{12}x + \frac{a_0}{36}x^2} \\
                   &= \frac{1}{\frac{7}{18} -\frac{1}{12}x + \frac{1}{36}x^2}\,.
    \end{align*}
    
    für $0 \neq a_0\in\R$.

\end{document}
