\documentclass[a4paper]{article}

\usepackage[margin=1in]{geometry} 
\usepackage{amsmath,amsthm,amssymb, graphicx, multicol, array}


\pdfminorversion=7
\pdfsuppresswarningpagegroup=1

\newcommand{\R}{\mathbb{R}}
\newcommand{\N}{\mathbb{N}}
\newcommand{\Z}{\mathbb{Z}}
\newcommand{\beh}{\textit{Behauptung. }}

\setlength{\parindent}{0pt}
\newenvironment{Aufgabe}[2][Aufgabe]{\begin{trivlist}
\item[\hskip \labelsep {\bfseries #1}\hskip \labelsep {\bfseries #2.}]}{\end{trivlist}}

\begin{document}

\begin{theorem} % Aufgabe #22
\begin{Aufgabe}{22} % #22
	Ein zeitdiskretes deterministisches Epidemie-Modell
\end{Aufgabe}

\textbf{a)} Gleichgewichtspunkte und Stabilität

\begin{proof}[Rechnung]
	Zuerst beobachten wir, dass die Anzahl der 'Recovered' also der Genesenen garnicht Teil des
	dynamischen Systems ist, sondern nur eine Kennzahl zu jedem Zeitpunkt. Also handelt es sich
	um ein 2-dimensionales dynamisches System. Nun wollen wir die Gleichgewichtspunkte dieses finden:
	\begin{align*}
		(i) \qquad s &= s - \alpha si \\
		(ii) \qquad i &= i - \beta i + \alpha si \\
		(i) \implies - \alpha si = 0 & \implies \left(
			s = 0 \text{ oder } i = 0
		\right) \\
			(ii) \implies \alpha si - \beta i = 0 &
			\overset{(i)} \implies \beta i = 0
			\overset{\beta > 0} \implies i = 0
	\end{align*}
	Wir sehen jetzt also, dass unser System immer genau dann im Gleichgewicht ist wenn wir keine
	Infizierten in der Gesellschaft haben.
	Also ist jeder Zustand der Form $(s, 0)$, für eine beliebige Anzahl an Gesunden Bürgern $s$
	ein Gleichgewichtspunkt des Systems.
	Nun interessiert uns die Stabilität des Systems.
	Also berechnen wir zuerst die Jacobi-Matrix, welche wie folgt aussieht:
	\begin{align*}
		J &= \begin{pmatrix} 
			\frac{ \partial S(t+1) }{ \partial S(t) } &
			\frac{ \partial S(t+1) }{ \partial I(t) } \\
			\frac{ \partial I(t+1) }{ \partial S(t) } &
			\frac{ \partial I(t+1) }{ \partial I(t) }
		\end{pmatrix} 
		= \begin{pmatrix} 
			1 - \alpha i & - \alpha s \\
			\alpha i & 1 - \beta + \alpha s
		\end{pmatrix} 
	\end{align*}
	Nun wollen wir uns die Jacobi-Matrix genauer an den Gleichgewichtspunkten ansehen.
	Wir setzen also die Lösung für die Gleichgewichtspunkte in die Jacobi-Matrix ein und erhalten:
	\begin{align*}
		J_{x ^{\star}} & \overset{i = 0} = \begin{pmatrix} 
			1 & - \alpha s \\
			0 & 1 - \beta + \alpha s
		\end{pmatrix} 
	\end{align*}
	Leider dürfen wir das Kriterium aus 21a) hier nicht verwenden, da die Jacobi-Matrix nicht symmetrisch ist.
	Also berechnen wir die Eigenwerte per Hand, indem wir die Nullstellen des charakteristischen Polynoms
	berechnen:
	\begin{align*}
		P_\lambda (J_{x ^{\star}}) &=
		\det (J_{x ^{\star}} - \lambda I_2) \\
			   &= (1 - \lambda) (1 - \beta + \alpha s - \lambda ) \\
			   &= \lambda ^2 + \lambda \left(
				\beta - \alpha s + 2
			   \right) + 1 - \beta + \alpha s \\
		P_\lambda (J_{x ^{\star}}) = 0 & \Leftrightarrow
		\lambda_{1, 2} = \frac{ 1 }{ 2 } \left(
			2 + \alpha s - \beta \pm \sqrt{
				\beta ^2 - 2 \beta \alpha s + 8 \beta + \alpha ^2 s ^2 - 8 \alpha s
			} 
		\right) 
	\end{align*}
	Wir überlegen uns also zusätzlich auch noch, dass das System offensichtlich auch stabil bleibt,
	wenn wir nur an der Anzahl der Gesunden etwas ändern. Im Hinblick auf die Stabilität sind
	also nur Änderungen an der Anzahl der Infizierten für uns interessant.
	Wohlgemerkt ist die Stabilität des Systems bei Änderungen an einem Gleichgewichtspunkt
	auch abhängig von der Anzahl der Gesunden. Hier überlegt man sich einmal, dass auch dies sehr
	viel Sinn macht, da jetzt ein Infizierter eine höhere beziehungsweise niedrigere Chance
	hat andere Bürger zu infizieren.
	Folglich müssen wir uns für die Eigenwerte der Jacobi-Matrix nun anschauen wann dessen Beträge
	kleiner beziehungsweise größer Null sind.
	\begin{align*}
		| \lambda_{1, 2} | < 1 & \Leftrightarrow
		\Big | 1 + \frac{ 1 }{ 2 } \left(
			\alpha s - \beta \pm \sqrt{
				\beta ^2 - 2 \beta \alpha s + 8 \beta + \alpha ^2 s ^2 - 8 \alpha s
			} 
		\right) \Big | < 1
	\end{align*}
	Wir überlegen uns jetzt, dass $\alpha \cdot s = \frac{ \alpha_0 }{ N } \cdot s = \frac{ s }{ N } \alpha_0$
	ist, da wir
	diesen Parameter ja so wählen. Diese Zahl sagt also wie groß der Anteil an Gesunden in der Gesellschaft
	ist, multipliziert mit dem Kontaktfaktor $\alpha_0$. Wenn wir jetzt unser dynamisches System aus dem
	Gleichgewicht bringen, indem wir einen Infizierten hinzufügen, dann sagt dieser Wert uns wie viele
	Gesunde Menschen dieser Infizierte anstecken kann. 
	Vereinfachen wir die Gleichung ein wenig erhalten wir das Folgende Stabilitätskriterium:
	\[
	\text{System bleibt stabil } \Leftrightarrow
	-4 < \alpha s - \beta \pm \sqrt{
		\beta ^2 - 2 \beta \alpha s + 8 \beta + \alpha ^2 s ^2 - 8 \alpha s
	} < 0
	\] 
	Die erste Differenz sagt also aus, ob die Störung des Systems (unser neuer Infizierter) mehr Menschen
	anstecken wird als er wieder 'gesundet'. 
\end{proof}
\end{theorem}

\textbf{b)} \beh $R(t+1) = R(t) + \beta I(t)$

\begin{proof}[Beweis]
	\begin{align*}
		R(t+1) &= N - S(t+1) - I(t+1) \\
			   &= N - \left(
				   S(t) - \alpha S(t) I(t)
			   \right) - \left(
			   I(t) - \beta I(t) + \alpha S(t) I(t)
			   \right) \\
			   &= N - S(t) - I(t) + \beta I(t) \\
			   &= R(t) + \beta I(t)
	\end{align*}
\end{proof}

\textbf{c)} Hierfür haben wir ein Python Script geschrieben, welches Sie unter dem Namen '08.py' finden.
Unter anderem haben wir auch 10 Bilder der Simulation für verschiedene Werte von $\alpha$
hinzugefügt. Man sieht, dass sobald $\alpha$ den Wert von ca. 1,7 überschreitet, das Wachstum
der Infizierten rasant ansteigt. Das Maximum für den Wert der total Infizierten finden
wir natürlich bei $\alpha = \frac{ 3 }{ N }$. Allerdings ist diese Simulation fehlerhaft,
da der Wert von $\alpha S(t) I(t)$ in diesem Fall so schnell ansteigt, dass
die Anzahl an Gesunden Bürgern kurzzeitig ins negative fällt.
Schauen wir uns jedoch die Simulation für $\alpha < \frac{ 1 }{ N }$ an, so erkennen wir
schöne logistische Wachstumskurven, und auch schnell eine Sättigung der Bevölerung
an Infizierten.
\\

Wir schlussfolgern also, dass dieses Modell nur brauchbar ist für kleine Werte von $\alpha$.

\end{document}

