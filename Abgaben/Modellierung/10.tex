\documentclass[a4paper]{article}

\usepackage[margin=1in]{geometry} 
\usepackage{amsmath,amsthm,amssymb, graphicx, multicol, array}

\usepackage{tikz}
\usetikzlibrary{automata, positioning}

\pdfminorversion=7
\pdfsuppresswarningpagegroup=1

\newcommand{\R}{\mathbb{R}}
\newcommand{\N}{\mathbb{N}}
\newcommand{\Z}{\mathbb{Z}}
\newcommand{\beh}{\textit{Behauptung. }}

\setlength{\parindent}{0pt}
\newenvironment{Aufgabe}[2][Aufgabe]{\begin{trivlist}
\item[\hskip \labelsep {\bfseries #1}\hskip \labelsep {\bfseries #2.}]}{\end{trivlist}}

\begin{document}
\title{ \textbf{Modellierung Blatt 10} }
\author{Amir Miri Lavasani, Bent Mueller}
\date{18. Juni 2021}
\maketitle

\begin{theorem} % Aufgabe #29
\begin{Aufgabe}{29} % #29
	Deterministische Prozesse mit stochastischen Matrizen
\end{Aufgabe}

\textbf{a)} Behälter Aufgabe

\begin{proof}[Beweis]
	Wir können das Problem als Markowkette modellieren,
	indem wir den Zustandsraum wie folgt wählen.
	\\

	$R_A$ sei Rum Anteil in Behälter A und $W_A$ sei Wasser
	Anteil in Behälter A. Analog $R_B$ und $W_B$ für Behälter
	B.
	Betrachten wir den Zustand unseres Experimentes wie folgt
	\[
	\left(
		R_A, W_A, R_B, W_B
	\right),
	\] 
	so können wir die Übergangsmatrix wie folgt hinschreiben:
	\[
	\mathbb{P} = \begin{pmatrix} 
		0.8 & 0 & 0.15 & 0 \\
		0 & 0.8 & 0 & 0.15 \\
		0.2 & 0 & 0.85 & 0 \\
		0 & 0.2 & 0 & 0.85 \\
	\end{pmatrix}
	\] 
	Nun suchen wir für diese eine invariante Verteilung.
	Wir überlegen uns also wann die Variablen jeweils im
	nächsten Zeitschritt auf sich selber geschickt werden.
	Hierfür lösen wir folgendes Gleichungssystem.
	\begin{align*}
		(i) \quad R_A &= 0.8 R_A + 0.15 R_B \\
		(ii) \quad R_B &= 0.2 R_A + 0.85 R_B \\
		   & \overset{(i)} \implies R_A = 0.75 R_B \\
		   & \overset{(ii)} \implies R_B 
		   = 0.2 R_A + 0.85 R_B \\
		   & \implies R_B = \left(
		   	0.2 \cdot 0.75 + 0.85
		   \right) R_B
	\end{align*}
	Wir sehen also, solange $R_A = 0.75 R_B$ gilt, so ist dies
	eine invariante Verteilung der Markowkette. Dies heisst für 
	unser dynamisches System, dass dieses ein stationärer Punkt
	ist. Durch Anwendung des Experimentes ändert sich nicht
	der Zustand des Systems.
	\\

	Da wir für den Wasseranteil jeweils die exakt selben
	Gleichungen erhalten, und Rum nicht zu Wasser werden kann
	und auch nicht anders herum, gilt das selbe auch für 
	die beiden Variablen $W_A$ und $W_B$.
	\\

	Jetzt müssen wir nur noch passende Variablen für 
	unser tatsächliches Experiment finden.
	\[
	R_A + R_B = 0.7 L
	\overset{(i)} \implies R_B \left(
		1 + 0.75
	\right) = 0.7 L
	\implies 
	R_B = 0.4L, \,
	R_A = 0.3L
	\] 
	Also finden wir als stationären Zustand $x$ folgende Anteile
	an Wasser bzw Rum in unseren Behältern:
	\[
	x = \left(
		0.3 L, 0.3 L, 0.4 L, 0.4 L
	\right) 
	\] 
	Somit wissen wir, dass dieser Zustand auch auf lange Zeit
	erreicht wird beim wiederholten ausführen des Experimentes.
\end{proof}
\end{theorem}

\textbf{b)} Fussball Weltmeisterschaft

\begin{proof}[Rechnung]
	Wir geben pro Gewinn eines Spiels einer Mannschaft jeweils
	3 Punkte wie es beim Fussball üblich ist.
	Die Punkte notieren wir jeweils mit $P_A$ als Punkte für 
	Team A, und auch analog für die anderen Teams.
	Somit erhalten wir Folgenden Rechenweg:
	\begin{align*}
		\text{A besiegt B} & \Rightarrow P_A = 3 \\
		\text{B besiegt C} & \Rightarrow P_B = 3 \\
		\text{B besiegt D} & \Rightarrow P_B = 6 \\
		\text{A besiegt C} & \Rightarrow P_A = 6 \\
		\text{D besiegt A} & \Rightarrow P_D = 3 \\
		\text{C besiegt D} & \Rightarrow P_C = 3 \\
	\end{align*}
	Nun können wir die Teams nach der Anzahl deren Punkte ordnen
	und erhalten folgende Reihenfolge:
	\begin{center}
	\begin{tabular}{ |c|c|c| } 
	 \hline
	 Name des Teams & Anzahl der Punkte \\ 
	 \hline
	 Team A & 6 \\ 
	 Team B & 6 \\ 
	 Team D & 3 \\ 
	 Team C & 3 \\ 
	 \hline
	\end{tabular}
\end{center}
\end{proof}

\begin{theorem} % Aufgabe #30
\begin{Aufgabe}{30} % #30
	Fahrschulformel
\end{Aufgabe}

\textbf{b)} Schulmathematik

\begin{proof}[Rechnung]
	Wir betrachten die gleichförmig gebremste Bewegung bei
	Fahrt mit $v_0$ Anfangsgeschwindigkeit (in km/h ).
	\[
	
	\] 
\end{proof}
\end{theorem}

\end{document}

