\documentclass[a4paper]{article}

\usepackage[margin=1in]{geometry} 
\usepackage{amsmath,amsthm,amssymb, graphicx, multicol, array}

\usepackage{tikz}
\usetikzlibrary{automata, positioning}

\pdfminorversion=7
\pdfsuppresswarningpagegroup=1

\newcommand{\R}{\mathbb{R}}
\newcommand{\N}{\mathbb{N}}
\newcommand{\Z}{\mathbb{Z}}
\newcommand{\beh}{\textit{Behauptung. }}

\setlength{\parindent}{0pt}
\newenvironment{Aufgabe}[2][Aufgabe]{\begin{trivlist}
\item[\hskip \labelsep {\bfseries #1}\hskip \labelsep {\bfseries #2.}]}{\end{trivlist}}

\begin{document}
\title{ \textbf{Modellierung Blatt #11} }
\author{Amir Miri Lavasani (7310114), Bent Müller (7302332)}
\date{28.06.2021}
\maketitle
	\begin{Aufgabe}{30}
		Analytisch lösbare Differentialgleichungen
	\end{Aufgabe}

	\textbf{c)} 
	\begin{proof}[Rechnung]
		Wir wählen nach dem Hinweis den linearen Ansatz der Lösung.
		\[
			y_p (x) = ax + b, \;\;
			y_p ' (x) = a
		\]
		Jetzt setzen wir dies in die Differentialgleichung aus der
		Aufgabenstellung ein und erhalten:
		\begin{align*}
			1 - x ^3 &= y' (x) - 2 x ^2 y(x) + xy(y) ^2 \\
			 &= a - 2x ^2 \left(
			 	ax + b
			 \right) + x \left(
			 	ax + b
			 \right) ^2 \\
			 &= a - 2a x ^3 - 2 ab x ^2
			 + x \left(
			 	a ^2 x ^2 + 2 ab x + b ^2
			 \right) \\
			 &= a - 2a x ^3 - 2 ab x ^2
			 + a ^2 x ^3 + 2 ab x ^2 + b ^2 x \\
			 &= a + b ^2 x - x ^2 \left(
			 	-2b + 2 ab
			 \right) + x ^3 \left(
			 	-2a + a ^2
			 \right) = 1 - x ^3 \\
			 & \overset{\text{Koeffizientenvergleich}} \implies 
			 \qquad a = 1, \quad b = 0
		\end{align*}
		Wir erkennen also, dass die einzige lineare Funktion die unsere
		Differentialgleichung löst $y_p (x) = x$ ist.
	\end{proof}
\end{document}
