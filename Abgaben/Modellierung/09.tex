\documentclass[a4paper]{article}

\usepackage[margin=1in]{geometry} 
\usepackage{amsmath,amsthm,amssymb, graphicx, multicol, array, tikz}
\usetikzlibrary{automata, positioning}

\pdfminorversion=7
\pdfsuppresswarningpagegroup=1

\newcommand{\R}{\mathbb{R}}
\newcommand{\N}{\mathbb{N}}
\newcommand{\Z}{\mathbb{Z}}
\newcommand{\beh}{\textit{Behauptung. }}

\setlength{\parindent}{0pt}
\newenvironment{Aufgabe}[2][Aufgabe]{\begin{trivlist}
\item[\hskip \labelsep {\bfseries #1}\hskip \labelsep {\bfseries #2.}]}{\end{trivlist}}

\begin{document}
\textbf{Aufgabe 25:} Stochastisches Tennis und das schwarze Schaf
\\

\textbf{b)} 
\begin{center}
	\begin{tikzpicture}
		% Add the states
		\node[state]			 (a) {1};
		\node[state, below=of a] (b) {2};
		\node[state, right=of b] (c) {3};
		\node[state, right=of c] (d) {4};
		\node[state, right=of d] (e) {5};

		% Connect the states with arrows
		\draw[every loop]
			% a -> b -> c -> d
			(a) edge[bend left, auto=right] node {0.5} (b)
			(a) edge[bend right, auto=left] node {0.5} (c)
			(c) edge[bend right, auto=right] node {0.5} (d)
			(d) edge[bend right, auto=right] node {$\frac{ 1 }{ 3 }$} (b)
			(d) edge[bend right, auto=right] node {$\frac{ 1 }{ 3 }$} (e)

			% d -> c -> b -> a
			(d) edge[bend right, auto=right] node {$\frac{ 1 }{ 3 }$} (c)
			(c) edge[bend right, auto=right] node {0.5} (a)

			% a -> a, b -> b, ...
			(b) edge[loop left]			 node {1.0} (b)
			(e) edge[loop right]			 node {1.0} (e)
	\end{tikzpicture}
\end{center}
\\

Die entsprechende Übergangsmatrix sieht wie folgt aus:
\[
\mathbb{P} = \begin{pmatrix} 
	0 & \frac{ 1 }{ 2 } & \frac{ 1 }{ 2 } & 0 & 0 \\
	0 & 1 & 0 & 0 & 0 \\
	\frac{ 1 }{ 2 } & 0 & 0 & \frac{ 1 }{ 2 } & 0 \\
	0 & \frac{ 1 }{ 3 } & \frac{ 1 }{ 3 } & 0 & \frac{ 1 }{ 3 } \\
	0 & 0 & 0 & 0 & 1 \\
\end{pmatrix}
\] 
\end{document}
