\documentclass[a4paper]{article}

\usepackage[margin=1in]{geometry} 
\usepackage{amsmath,amsthm,amssymb, graphicx, multicol, array, tikz}
\usetikzlibrary{automata, positioning}

\pdfminorversion=7
\pdfsuppresswarningpagegroup=1

\newcommand{\R}{\mathbb{R}}
\newcommand{\N}{\mathbb{N}}
\newcommand{\Z}{\mathbb{Z}}
\newcommand{\beh}{\textit{Behauptung. }}

\setlength{\parindent}{0pt}
\newenvironment{Aufgabe}[2][Aufgabe]{\begin{trivlist}
\item[\hskip \labelsep {\bfseries #1}\hskip \labelsep {\bfseries #2.}]}{\end{trivlist}}

\begin{document}
\textbf{Aufgabe 25:} Stochastisches Tennis und das schwarze Schaf
\\

\textbf{a)} 
Wir modellieren den Stand beim Tennismatch mit Hilfe einer Markowkette.
Schauen wir uns die Reihenfolge der Aufschläge genau an, dann können wir diese
Aufteilen in Folgende Teile.

\[
AB\;\; BA\;\; AB\;\; BA\;\; AB\;\; \cdots
\] 
Bei jedem Aufschlag hat jeder der Spieler eine Chance $p$ einen Punkt zu machen.
Die Markowkette ist nun eine Irrfahrt mit absorbierendem Rand.
Der Zustand ist dann die jeweilige Punktedifferenz nachdem $A$ und $B$ beide ihre Aufschläge
gemacht haben.
Ein Zeitschritt heisst also zwei Aufschläge.
Somit ergibt sich folgender Übergangsgraph:

\begin{center}
	\begin{tikzpicture}
		% Add the states
		\node[state]			 (a) {-2};
		\node[state, right=of a] (b) {-1};
		\node[state, right=of b] (c) {0};
		\node[state, right=of c] (d) {+1};
		\node[state, right=of d] (e) {+2};

		% Connect the states with arrows
		\draw[every loop]
			% a -> b -> c -> d
			(a) edge[bend right, auto=right] node {} (b)
			(b) edge[bend right, auto=right] node {} (c)
			(c) edge[bend right, auto=right] node {} (d)
			(d) edge[bend right, auto=right] node {} (e)

			% d -> c -> b -> a
			(e) edge[bend right, auto=right] node {} (d)
			(d) edge[bend right, auto=right] node {} (c)
			(c) edge[bend right, auto=right] node {} (b)
			(b) edge[bend right, auto=right] node {} (a)

			% a -> a, b -> b, ...
			(a) edge[loop left]			     node {1} (a)
			(e) edge[loop right]             node {1} (e)
			(b) edge[loop above]			 node {} (b)
			(c) edge[loop above]			 node {} (c)
			(d) edge[loop above]			 node {} (d)
	\end{tikzpicture}
\end{center}
\\

Die Punktedifferenz messen wir immer von Spieler $A$ zu Spieler $B$.
Wir überlegen uns, dass das Spiel bei Zustand $\pm2$ beendet ist, da nun ein Spieler mit
zwei Punkten vorne liegt.

\textbf{b)} 
\begin{center}
	\begin{tikzpicture}
		% Add the states
		\node[state]			 (a) {1};
		\node[state, below=of a] (b) {2};
		\node[state, right=of b] (c) {3};
		\node[state, right=of c] (d) {4};
		\node[state, right=of d] (e) {5};

		% Connect the states with arrows
		\draw[every loop]
			% a -> b -> c -> d
			(a) edge[bend left, auto=right] node {0.5} (b)
			(a) edge[bend right, auto=left] node {0.5} (c)
			(c) edge[bend right, auto=right] node {0.5} (d)
			(d) edge[bend right, auto=right] node {$\frac{ 1 }{ 3 }$} (b)
			(d) edge[bend right, auto=right] node {$\frac{ 1 }{ 3 }$} (e)

			% d -> c -> b -> a
			(d) edge[bend right, auto=right] node {$\frac{ 1 }{ 3 }$} (c)
			(c) edge[bend right, auto=right] node {0.5} (a)

			% a -> a, b -> b, ...
			(b) edge[loop left]			 node {1.0} (b)
			(e) edge[loop right]			 node {1.0} (e)
	\end{tikzpicture}
\end{center}
\\

Die entsprechende Übergangsmatrix sieht wie folgt aus:
\[
\mathbb{P} = \begin{pmatrix} 
	0 & \frac{ 1 }{ 2 } & \frac{ 1 }{ 2 } & 0 & 0 \\
	0 & 1 & 0 & 0 & 0 \\
	\frac{ 1 }{ 2 } & 0 & 0 & \frac{ 1 }{ 2 } & 0 \\
	0 & \frac{ 1 }{ 3 } & \frac{ 1 }{ 3 } & 0 & \frac{ 1 }{ 3 } \\
	0 & 0 & 0 & 0 & 1 \\
\end{pmatrix} ^{T}
\] 

Also berechnen wir $\mathbb{P} ^{3}$, welche wie folgt aussieht:
\[
\mathbb{P} ^{3} = \begin{pmatrix} 
	0 & 0.708 \overline{3} & 0.208 \overline{3} & 0 & 0.08 \overline{3} \\
	0 & 1 & 0 & 0 & 0 \\
	0.208 \overline{3} & 0.41 \overline{6} & 0 & 0.208 \overline{3} & 0.1 \overline{6} \\
	0 & 0.47 \overline{2} & 0.13 \overline{8} & 0 & 0.3 \overline{8} \\
	0 & 0 & 0 & 0 & 1 \\
\end{pmatrix} ^{T}
\] 
Wir erkennen also, dass die Wahrscheinlichkeit in Zustand 2 zu landen genau in Spalte 2
(beziehungsweise Zeile 2 nach dem transponieren) steht. 
Wir ignorieren den Fall in welchem wir in Zustand 5 starten, also den letzten Eintrag.
Nun erkennen wir, dass das minimum in dieser Zeile $0.41 \overline{6} = \frac{ 5 }{ 12 }$ ist. Dies entspricht aber genau
der Aufgabenstellung, nämlich die kleinste Wahrscheinlichkeit, dass das Schaf stirbt wobei es überall
aber nicht in 5 startet.
\end{document}
