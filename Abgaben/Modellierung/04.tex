\documentclass[a4paper]{article}

\usepackage[margin=1in]{geometry} 
\usepackage{amsmath,amsthm,amssymb, graphicx, multicol, array}


\pdfminorversion=7
\pdfsuppresswarningpagegroup=1

\newcommand{\R}{\mathbb{R}}
\newcommand{\N}{\mathbb{N}}
\newcommand{\Z}{\mathbb{Z}}
\newcommand{\beh}{\textit{Behauptung. }}

\setlength{\parindent}{0pt}
\newenvironment{Aufgabe}[2][Aufgabe]{\begin{trivlist}
\item[\hskip \labelsep {\bfseries #1}\hskip \labelsep {\bfseries #2.}]}{\end{trivlist}}

\begin{document}

\begin{document}
\title{ \textbf{Modellierung - Blatt \#4} }
\author{Amir Miri Lavasani (7310114), Bent Müller (7302332)}
\date{14. April 2021}
\maketitle
\begin{Aufgabe}{12}
	Optimierung auf einer Kreisscheibe.
\end{Aufgabe}

\textbf{(a)} Bestimmen sie alle stationären Punkte von $f(x_1, x_2)$ 
auf dem Rand der Einheitskreisscheibe.
\begin{proof}[]
	Im Folgenden setzen wir der Einfachheit halber $x_1 = x, x_2 = y$ und bestimmen
	nun die Lagrange Funktion welche wie folgt aussieht:
	\[
		L(x,y) = 4x ^2 - 2 x y + \lambda (x ^2 + y ^2 - 1) 
	\]
	Folglich berechnen wir die Einträge der Jakobi-Matrix, beziehungsweise die
	partiellen Ableitungen unserer Lagrange Funktion bezüglich dessen
	Parametern. Zusammen mit der Nebenbedingung, welche wir auch als die partielle
	Ableitung nach $\lambda$ auffassen können erhalten wir nun ein nichtlineares
	Gleichungssystem mit 3 Gleichungen, welches wie folgt aussehen:
	\begin{align}
		0 &= \frac{ \partial L(x,y) }{ \partial x } =
		8x - 2y + 2 \lambda x \\
		0 &= \frac{ \partial L(x,y) }{ \partial y } =
		-2x + 2 \lambda y \\
		0 &= \frac{ \partial L(x,y) }{ \partial \lambda  } =
		x ^2 + y ^2 - 1
	\end{align}

	Dieses lösen wir nun wie folgt nach allen gesuchten Parametern auf:
	\begin{align*}
		(2) &\implies x = \lambda y
		\overset{(3)} \implies \lambda ^2 y ^2 + y ^2 - 1 = 0
			\implies (\lambda ^2 + 1) y ^2 - 1 = 0
			\implies y ^2 = \frac{ 1 }{ 1 + \lambda ^2 } \\
			& \overset{(1)} \implies 8 \lambda y - 2 y + 2 \lambda ^2 y = 0
			\implies 8 \lambda y - 2 y + 2 \left(
				\frac{ 1 }{ y ^2 } - 1
			\right) y = 0 \\
			& \overset{\cdot y} \implies
			(8 \lambda  - 2) y ^2 + 2 - 2 y ^2 = 0
			\implies y ^2 = \frac{ -1 }{ 4 \lambda - 2 }
			\overset{(2)} = \frac{ 1 }{ 1 + \lambda ^2 } \\
			& \implies 1 + \lambda ^2 = - (4 \lambda - 2)
			\implies 0 = \lambda ^2 + 4 \lambda - 1
			\implies \lambda = \pm \sqrt{5} - 2 \\
			& \implies y = \pm \sqrt{-\frac{ 1 }{ 4 \lambda - 2 }}
			\text{ und } x = \lambda \cdot \pm \sqrt{-\frac{ 1 }{ 4 \lambda - 2 }} 
	\end{align*}
	Wir erhalten also zwei Lösungen für $\lambda$ und für jede dieser
	Lösungen jeweils zwei verschiedene $y$ und $x$. Somit folgt,
	dass wir zunächst 4 stationäre Punkte auf unserem Einheitskreis
	gefunden haben welche alle zulässig sind.
	Ohne großen Aufwand können wir diese Punkte auch noch einmal
	explizit aufschreiben:
	\begin{align*}
		y_1 &= \sqrt{-\frac{ 1 }{ 4 \sqrt{5} -10 }},
		x_1 = (\sqrt{5} - 2) \sqrt{-\frac{ 1 }{ 4 \lambda - 2 }} \\
		y_2 &= - \sqrt{-\frac{ 1 }{ 4 \sqrt{5} -10 }},
		x_2 = - (\sqrt{5} - 2) \sqrt{-\frac{ 1 }{ 4 \lambda - 2 }} \\
		y_3 &= \sqrt{-\frac{ 1 }{ - 4 \sqrt{5} -10 }},
		x_3 = - (\sqrt{5} - 2) \sqrt{-\frac{ 1 }{ - 4 \lambda - 2 }} \\
		y_4 &= - \sqrt{-\frac{ 1 }{ - 4 \sqrt{5} -10 }},
		x_4 = (\sqrt{5} - 2) \sqrt{-\frac{ 1 }{ - 4 \lambda - 2 }} \\
	\end{align*}
\end{proof}

\textbf{(b)} Zeigen sie bei allen stationären Punkten aus (a), dass
es sich bei diesen um lokale Extrema handelt.

\begin{proof}[Beweis]
	Um dies zu zeigen berechnen wir die Einträge der Hesse-Matrix, indem
	wir einfach die partiellen Ableitungen aus der vorherigen Aufgabe
	noch einmal jeweils nach $x$ bzw. $y$ ableiten.
	\begin{align*}
		\frac{ \partial L(x,y) }{ \partial x } &= 8 x - 2y + 2 \lambda x \\
		\frac{ \partial L(x,y) }{ \partial y } &= -2x + 2 \lambda y
	\end{align*}
	Und diese beiden Ableitungen können wir jetzt noch einmal wie folgt jeweils nach
	$x$ bzw $y$ ableiten und erhalten:
	\begin{align*}
		\frac{ \partial L(x,y) }{ \partial x ^2 } &= 8 + 2 \lambda \\
		\frac{ \partial L(x,y) }{ \partial x \partial y } &= -2 \\
		\frac{ \partial L(x,y) }{ \partial y \partial x } &= -2 \\
		\frac{ \partial L(x,y) }{ \partial y ^2 } &= 2 \lambda \\
	\end{align*}
	Somit erhalten wir die Hesse-Matrix welche wie folgt aussieht:
	\[
		HL(x,y) = \begin{pmatrix} 
			8 + 2 \lambda & -2 \\
			-2 & 2 \lambda \\
		\end{pmatrix} 
	\]
	Also berechnen wir im Folgenden das charakteristische Polynom der Hesse-Matrix.
	Die gesuchten Eigenwerte sind nämlich die Nullstellen dieses Polynoms:
	\begin{align*}
		0 = \det (HL(x,y) - X \cdot I_2) &=
		\det \begin{pmatrix} 
			8+ 2 \lambda - X & -2 \\
			-2 & 2 \lambda - X
		\end{pmatrix} \\
		 &= (8 + 2 \lambda - X) (2 \lambda - X) - 4 \\
		 &= 16 \lambda  - 8 X + 4 \lambda ^2 - 2 \lambda X - 2 \lambda X + X ^2 - 4 \\
		 &= X ^2 + (-8 - 4 \lambda) X - 4 + 4 \lambda ^2 + 16 \lambda \\
		 & \implies X = \pm \sqrt{32 \lambda - 24} + 2 \lambda + 4
	\end{align*}
\end{proof}
\end{theorem}

\end{document}
