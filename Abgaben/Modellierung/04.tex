\documentclass[a4paper]{article}

\usepackage[margin=1in]{geometry} 
\usepackage{amsmath,amsthm,amssymb, graphicx, multicol, array}


\pdfminorversion=7
\pdfsuppresswarningpagegroup=1

\newcommand{\R}{\mathbb{R}}
\newcommand{\N}{\mathbb{N}}
\newcommand{\Z}{\mathbb{Z}}
\newcommand{\beh}{\textit{Behauptung. }}

\setlength{\parindent}{0pt}
\newenvironment{Aufgabe}[2][Aufgabe]{\begin{trivlist}
\item[\hskip \labelsep {\bfseries #1}\hskip \labelsep {\bfseries #2.}]}{\end{trivlist}}

\begin{document}

\begin{document}
\title{ \textbf{Modellierung - Blatt \#4} }
\author{Amir Miri Lavasani (7310114), Bent Müller (7302332)}
\date{14. April 2021}
\maketitle
\begin{Aufgabe}{12}
	Optimierung auf einer Kreisscheibe.
\end{Aufgabe}

\textbf{(a)} Bestimmen sie alle stationären Punkte von $f(x_1, x_2)$ 
auf dem Rand der Einheitskreisscheibe.
\begin{proof}[]
	Im Folgenden setzen wir der Einfachheit halber $x_1 = x, x_2 = y$ und bestimmen
	nun die Lagrange Funktion welche wie folgt aussieht:
	\[
		L(x,y) = 4x ^2 - 2 x y + \lambda (x ^2 + y ^2 - 1) 
	\]
	Folglich berechnen wir die Einträge der Jakobi-Matrix, beziehungsweise die
	partiellen Ableitungen unserer Lagrange Funktion bezüglich dessen
	Parametern. Zusammen mit der Nebenbedingung, welche wir auch als die partielle
	Ableitung nach $\lambda$ auffassen können erhalten wir nun ein nichtlineares
	Gleichungssystem mit 3 Gleichungen, welches wie folgt aussehen:
	\begin{align}
		0 &= \frac{ \partial L(x,y) }{ \partial x } =
		8x - 2y + 2 \lambda x \\
		0 &= \frac{ \partial L(x,y) }{ \partial y } =
		-2x + 2 \lambda y \\
		0 &= \frac{ \partial L(x,y) }{ \partial \lambda  } =
		x ^2 + y ^2 - 1
	\end{align}

	Dieses lösen wir nun wie folgt nach allen gesuchten Parametern auf:
	\begin{align*}
		(2) &\implies x = \lambda y
		\overset{(3)} \implies \lambda ^2 y ^2 + y ^2 - 1 = 0
			\implies (\lambda ^2 + 1) y ^2 - 1 = 0
			\implies y ^2 = \frac{ 1 }{ 1 + \lambda ^2 } \\
			& \overset{(1)} \implies 8 \lambda y - 2 y + 2 \lambda ^2 y = 0
			\implies 8 \lambda y - 2 y + 2 \left(
				\frac{ 1 }{ y ^2 } - 1
			\right) y = 0 \\
			& \overset{\cdot y} \implies
			(8 \lambda  - 2) y ^2 + 2 - 2 y ^2 = 0
			\implies y ^2 = \frac{ -1 }{ 4 \lambda - 2 }
			\overset{(2)} = \frac{ 1 }{ 1 + \lambda ^2 } \\
			& \implies 1 + \lambda ^2 = - (4 \lambda - 2)
			\implies 0 = \lambda ^2 + 4 \lambda - 1
			\implies \lambda = \pm \sqrt{5} - 2
	\end{align*}
\end{proof}
\end{document}
