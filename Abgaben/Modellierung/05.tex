\documentclass[a4paper]{article}

\usepackage[margin=1in]{geometry} 
\usepackage{amsmath,amsthm,amssymb, graphicx, multicol, array}


\pdfminorversion=7
\pdfsuppresswarningpagegroup=1

\newcommand{\R}{\mathbb{R}}
\newcommand{\N}{\mathbb{N}}
\newcommand{\Z}{\mathbb{Z}}
\newcommand{\beh}{\textit{Behauptung. }}

\setlength{\parindent}{0pt}
\newenvironment{Aufgabe}[2][Aufgabe]{\begin{trivlist}
\item[\hskip \labelsep {\bfseries #1}\hskip \labelsep {\bfseries #2.}]}{\end{trivlist}}

\begin{document}

\begin{theorem} % Aufgabe #13
\begin{Aufgabe}{13} % #13
	Optimierung unter Gleichungsnebenbedingungen
\end{Aufgabe}

\textbf{(c)}  Bestimmen sie die Punkte auf der Parabel $x_2 = x_1 ^2 + 2$, die vom Ursprung
den kürzesten Abstand haben

\begin{proof}[Rechnung]
	Wir benutzen wieder $x_1 = x$ und $x_2 = y$ und formulieren nun die Lagrange Funktion wie folgt:
	\[
		L(x,y) = \sqrt{x ^2 + y ^2} + \lambda ( x ^2 + 2 - y )
	\] 
	Wir sehen die zu minimierende Funktion ist der euklidische Abstand zum Ursprung. Nun bestimmen
	wir die folgenden partiellen Ableitungen welche uns dann unser Gleichungssystem geben:
	\begin{align}
		\frac{ \partial L(x,y) }{ \partial x } &= - \frac{ 2 x }{ 2 \sqrt{x ^2 + y ^2}  } + 2 \lambda x
		= - \frac{ x }{ \sqrt{x ^2 + y ^2}  } + 2 \lambda x = 0 \\
		\frac{ \partial L(x,y) }{ \partial y } &= -\frac{ y }{ \sqrt{x ^2 + y ^2}  } - \lambda = 0 \\
		\frac{ \partial L(x,y) }{ \partial \lambda } &= x ^2 + 2 - y = 0
	\end{align}
	Mit diesen drei Gleichungen lösen wir nun nach den Variablen $x, y, \lambda$ auf wie folgt:
	\begin{align*}
		(3) & \implies y = x ^2 + 2 \overset{(2)} \implies \frac{-( x ^2 + 2)}{
			\sqrt{x ^2 + x ^4 + 4x ^2 + 4} 
		} = \lambda \\
		& \overset{(1)} \implies -\frac{ x }{ \sqrt{x ^2 + x ^{4} + 4 x ^2 + 4}  }
		+ 2 x \left(
			\frac{ - ( x ^2 + 2 ) }{ \sqrt{x ^{4} + 5 x ^2 + 4}  }
		\right) = 0 \\
		& \implies \frac{ - 2x (x ^2 + 2) }{ \sqrt{x ^{4} + 5 x ^2 + 4}  }
		= \frac{ x }{ \sqrt{x ^{4} + 5 x ^2 + 4}  } \\
		& \implies -2x ( x ^2 + 2 ) = x
		\implies 2 ( x ^2 + 2 ) + 1 = 0
		\text{ oder } x = 0
	\end{align*}
	Da $x$ aber eine reelle Zahl ist wissen wir dass von diesen beiden Gleichungen nur $x = 0$ funktionieren
	kann. Somit können wir auch die anderen Parameter wie folgt berechnen:
	\begin{align*}
		y &= x ^2 + 2 = 0 ^2 + 2 = 2 \\
		\lambda &= \frac{ -y }{ \sqrt{x ^2 + y ^2}  }
		= \frac{ -2 }{ \sqrt{0 ^2 + 2 ^2}  } = -1
	\end{align*}
	Nun kennen wir auch den einzigen Punkt der unsere Gleichung erfüllt mit:
	\[
		(x,y) = (0, 2)
	\] 
\end{proof}
\end{theorem}

\end{document}
