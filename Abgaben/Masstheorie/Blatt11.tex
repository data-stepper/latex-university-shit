\documentclass[10pt]{article}
 
\usepackage[margin=1in]{geometry} 
\usepackage{amsmath,amsthm,amssymb, graphicx, multicol, array}
 \usepackage{bbm, mathtools}
 
\newcommand{\R}{\mathbb{R}}
\newcommand{\N}{\mathbb{N}}
\newcommand{\Z}{\mathbb{Z}}
\newcommand{\A}{\mathcal{A}}
\newcommand{\C}{\mathcal{C}}
\newcommand{\D}{\mathcal{D}}
\newcommand{\B}{\mathbb{B}}
\newcommand{\normal}{\mathcal{N}}
\newcommand{\gap}{\,\vert\,}
\newcommand{\beh}{\textit{Behauptung. }}

\setlength{\parindent}{0pt}
 
\newenvironment{Aufgabe}[2][Aufgabe]{\begin{trivlist}
\item[\hskip \labelsep {\bfseries #1}\hskip \labelsep {\bfseries #2.}]}{\end{trivlist}}
\newenvironment{amatrix}[1]{%
  \left(\begin{array}{@{}*{#1}{c}|c@{}}
}{%
  \end{array}\right)
}

\begin{document}
 
\title{ \textbf{Maßtheoretische Konzepte der Stochastik \\ -- Übungsblatt \#11 --} }

\author{Amir Miri Lavasani (7310114), Bent Müller (7302332),
        Michael Hermann (6981007)}
\maketitle

\begin{Aufgabe}{1} % #1
	Sei $X:(\Omega, \mathcal{A}, P) \rightarrow (\mathbb{R}, \mathbb{B})$ eine endlich integrierbare Zufallsvariable, 
	$\mathcal{C} \subseteq \mathcal{A}$ eine Unter-$\sigma$-Algebra und $Z$ eine $\mathcal{C},\mathbb{B}$-messbare Zufallsvariable.
\end{Aufgabe}

\beh Die folgenden beiden Aussagen sind äquivalent:
\begin{itemize}
	\item[(i)] $Z = E(X|\mathcal{C}) \;\;$ P-f.s.
	\item[(ii)] $E(XY) = E(ZY)$ für alle  $\mathcal{C},\mathbb{B}$-messbaren beschränkten Zufallsvariablen $Y$
\end{itemize}

\begin{proof}[Beweis]  
Es gelte (ii). Für eine beliebige Menge $D \in \mathcal{C}$ ist $\mathbbm{1}_D$ eine beschränkte $\mathcal{C},\mathbb{B}$-messbare Zufallsvariable und es gilt
$$
\int_{D} X \,dP = \int X \mathbbm{1}_D \,dP = E(X \mathbbm{1}_D) = E(Z \mathbbm{1}_D) =  \int Z \mathbbm{1}_D \,dP = \int_{D} Z \,dP
$$
Da $Z$ nach Voraussetzung $\mathcal{C},\mathbb{B}$-messbar ist, ist $Z$ ein bedingter Erwartungswert von $X$ gegeben $\mathcal{C}$. \\

Es gelte (i). Wir zeigen die Gültigkeit von (ii) durch algebraische Induktion. \\
Sei zuerst $Y \coloneqq \sum_{i=1}^{n} a_i \mathbbm{1}_{A_i}$ eine Elementarfunktion mit $a_i > 0$ und paarweise disjunkten Mengen $A_i \in \mathcal{C}$ für $i=1, \dots n$. Dann gilt
\begin{align*}
E(XY) &= E(X  \sum_{i=1}^{n} a_i \mathbbm{1}_{A_i}) =  \sum_{i=1}^{n} a_i E(X \mathbbm{1}_{A_i}) =  \sum_{i=1}^{n} a_i \int_{A_i} X dP \\
	  &=  \sum_{i=1}^{n} a_i \int_{A_i} Z dP =  \sum_{i=1}^{n} a_i E(Z \mathbbm{1}_{A_i}) =  E(Z \sum_{i=1}^{n} a_i \mathbbm{1}_{A_i}) = E(ZY) 
\end{align*}
wobei die Linearität des Erwartungswertes und die Radon-Nikodym-Gleichungen verwendet wurden. \\
Ist $Y:(\Omega, \mathcal{A}) \rightarrow (\mathbb{R}, \mathbb{B})$ eine beliebige nichtnegative beschränkte $\mathcal{C},\mathbb{B}$-messbare Zufallsvariable, dann gibt es eine Folge von Elementarfunktionen $(Y_n)_{n \in \mathbb{N}}$, die monoton von unten gegen $Y$ konvergiert. Es gilt dann $XY_n \rightarrow XY$ und $ZY_n \rightarrow ZY$ für $n \to \infty$ und, wie soeben gesehen, $E(XY_n) = E(ZY_n)$ für alle $n \in \mathbb{N}$. Mit $Y$ sind auch alle $Y_n$ beschränkt und, da $X$ und $Z$ endlich integrierbar sind, sind die Folgen $XY_n$ und $ZY_n$ gleichgradig $P$-integrierbar, denn $|XY_n| \leq |XY|$ und $XY \in \mathcal{L}_1$ bzw. $|ZY_n| \leq |ZY|$ und $ZY \in \mathcal{L}_1$. Es folgt
$$
E(XY) = E( \lim_{n \to \infty} XY_n) =  \lim_{n \to \infty} E(X Y_n) =  \lim_{n \to \infty} E(Z Y_n) = E(\lim_{n \to \infty} ZY_n) = E(ZY)
$$
Ist schliesslich $Y$ eine beliebige beschränkte $\mathcal{C},\mathbb{B}$-messbare Zufallsvariable, dann folgt die Aussage nach Zerlegung in Positiv- und Negativteil und der Linearität des Erwartungswertes:
$$
E(XY) = E(XY^+)-E(XY^-) = E(ZY^+)-E(ZY^-) = E(ZY) 
$$
 
\end{proof}
 
 \newpage
 
 \begin{Aufgabe}{2} % #2
 \begin{itemize}
 \item[a)]
	Seien $Y_n, n \in \mathbb{N}$ unabhängige, identisch verteilte Zufallsvariablen mit $E(Y_1)=1$. Dann ist 
	$$
	X_n \coloneqq \prod_{i=1}^n Y_i
	$$ 
	für $n \in \mathbb{N}_0$ (mit der Konvention $X_0 = 1$) ein Martingal.
 \item[b)]
	Für unabhängige, identisch verteilte Zufallsvariablen $Z_n, n \in \mathbb{N}$, $M(t) \coloneqq E(e^{tZ_1}) < \infty$ für $t \in \mathbb{R} \setminus \{0\}$ und $S_n \coloneqq \sum{i=1}^n Z_i$ für $n \in \mathbb{N}$ ist
	$$
	X_n \coloneqq \frac{\exp(tS_n)}{(M(t))^n}, \; n \in \mathbb{N}_0
	$$
	ein Martingal.
\end{itemize}
\end{Aufgabe}

\begin{proof}[Beweis]  
 \begin{itemize}
 \item[a)]
	Es gilt 
	$$
	E(|X_n|) = E(\prod_{i=1}^n |Y_i|) = \prod_{i=1}^n E(|Y_i|) < \infty 
	$$
	da die $Y_i$ insbesondere endlich integrierbar sind. Die Adaption ist klar, da wir die natürliche Filtration verwenden. Ausserdem gilt
	$$
	E(X_{n+1}|(X_i)_{1 \leq i \leq n}) = E(Y_{n+1}X_{n}|(X_i)_{1 \leq i \leq n}) \overset{(*)}{=} X_n E(Y_{n+1} | (X_i)_{1 \leq i \leq n}) \overset{(**)}{=}  X_n E(Y_{n+1}) = X_n 
	$$
	wobei wir für $(*)$ Teil (ix) von Satz 9.5 mit $g((X_i)_{1 \leq i \leq n})) \coloneqq X_n $ benutzt haben. Die Gleichheit $(**)$ folgt aus der stoch. Unabhängigkeit der Zufallsvariablen $Y_i$. 
	Damit ist $(X_n)_{n \in \mathbb{N}_0}$ ein Martingal.
 \item[b)] Mit $Y_i \coloneqq \frac{\exp(tZ_i)}{M(t)}$ ist
 $$
 X_n = \frac{\exp(tS_n)}{(M(t))^n} = \frac{\exp(\sum_{i=1}^n tZ_i)}{(M(t))^n} = \prod_{i=1}^n \frac{\exp(tZ_i)}{M(t)} = \prod_{i=1}^n Y_i
 $$
 und es gilt
 $$
 E(Y_i) = E\left(\frac{\exp(tZ_i)}{M(t)}\right) = \frac{E(\exp(tZ_i))}{M(t)} = \frac{E(\exp(tZ_i))}{E(\exp(tZ_1)} = 1
 $$
 Die Aussage folgt nun direkt aus a), da mit den $Z_i$ natürlich auch die $Y_i$ stochastisch unabhängig sind.
 
\end{itemize}
\end{proof}


\begin{Aufgabe}{3} % #3
	Sei $(S_n)_{n \in \mathbb{N}}$ eine Folge von Stoppzeiten bzgl. einer Filtration $(\mathcal{A}_t)_{t \in \mathbb{N}_0}$. \\
	Dann gilt:
	\begin{itemize}
		\item[(i)] $\min_{n \in \mathbb{N}_0} S_n$ und $\sup_{n \in \mathbb{N}}$ sind Stoppzeiten bzgl. $(\mathcal{A}_t)_{t \in \mathbb{N}_0}$
		\item[(ii)] Die Ereignisse $\{S_1 \leq S_2\}$ und $\{S_1= S_2\}$ liegen in $\mathcal{A}_{S_1} \cap \mathcal{A}_{S_2}$
	\end{itemize}
\end{Aufgabe}

\begin{proof}[Beweis]  
\begin{itemize}
	\item[(i)] Es gilt $\{S_n \leq t\} \in \mathcal{A}_t$ für jedes $n \in \mathbb{N}_0$, da jedes $S_n$ eine Stoppzeit ist. \\
	Da $\mathcal{A}_t$ eine $\sigma$-Algebra ist folgt damit sofort
	$$
	\{\min_{n \in \mathbb{N}_0} S_n \leq t\} = \{S_n \leq t \;\; \text{für ein } n \in \mathbb{N}_0 \} = \bigcup_{n \in \mathbb{N}_0} \{S_n \leq t\} \in \mathcal{A}_t
	$$
	und
	$$
	\{\sup_{n \in \mathbb{N}_0} S_n \leq t\} = \{S_n \leq t \;\; \text{für alle }  n \in \mathbb{N}_0 \} = \bigcap_{n \in \mathbb{N}_0} \{S_n \leq t\} \in \mathcal{A}_t,
	$$
	dh. die beiden Funktionen sind tatsächlich Stoppzeiten.
	\item[(ii)] Es gilt
	$$
	\{S_1 \leq S_2\} = \bigcup_{t \in \mathbb{N}_0} \{S_1 \leq t\} \cap  \{S_2 \geq t\} 
	$$
	und 
	$$
	\{S_1 = S_2\} = \bigcup_{t \in \mathbb{N}_0} \{S_1 = t\} \cap  \{S_2 = t\} 
	$$
	Die Mengen $\{S_i \leq t\}$, $\{S_i \geq t\} = \{S_i \leq t-1\}^c$ und $\{S_i = t\} = \{S_i \leq t\} \cap \{S_i \geq t\}$ liegen alle in $\mathcal{A}_t \subseteq \mathcal{A}$, also auch in $\mathcal{A}_{S_i}$, jeweils für $i = 1,2$.
	Damit ist $\{S_1 \leq t\} \cap  \{S_2 \geq t\} \in \mathcal{A}_{S_1} \cap \mathcal{A}_{S_2}$ und $\{S_1 = t\} \cap  \{S_2 = t\} \in \mathcal{A}_{S_1} \cap \mathcal{A}_{S_2}$ für jedes $t \in \mathbb{N}_0$. 
	Die Aussage folgt nun, da jede $\sigma$-Algebra alle abzählbaren Vereinigungen ihrer Mengen enthält und der Schnitt zweier $\sigma$-Algebren wiederum eine $\sigma$-Algebra ist.
\end{itemize}
\end{proof}


\end{document}












