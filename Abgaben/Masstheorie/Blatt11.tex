\documentclass[10pt]{article}
 
\usepackage[margin=1in]{geometry} 
\usepackage{amsmath,amsthm,amssymb, graphicx, multicol, array}
 \usepackage{bbm, mathtools}
 
\newcommand{\R}{\mathbb{R}}
\newcommand{\N}{\mathbb{N}}
\newcommand{\Z}{\mathbb{Z}}
\newcommand{\A}{\mathcal{A}}
\newcommand{\C}{\mathcal{C}}
\newcommand{\D}{\mathcal{D}}
\newcommand{\B}{\mathbb{B}}
\newcommand{\normal}{\mathcal{N}}
\newcommand{\gap}{\,\vert\,}
\newcommand{\beh}{\textit{Behauptung. }}

\setlength{\parindent}{0pt}
 
\newenvironment{Aufgabe}[2][Aufgabe]{\begin{trivlist}
\item[\hskip \labelsep {\bfseries #1}\hskip \labelsep {\bfseries #2.}]}{\end{trivlist}}
\newenvironment{amatrix}[1]{%
  \left(\begin{array}{@{}*{#1}{c}|c@{}}
}{%
  \end{array}\right)
}

\begin{document}
 
\title{ \textbf{Maßtheoretische Konzepte der Stochastik \\ -- Übungsblatt \#11 --} }

\author{Amir Miri Lavasani (7310114), Bent Müller (7302332),
        Michael Hermann (6981007)}
\maketitle

\begin{Aufgabe}{1} % #1
	Sei $X:(\Omega, \mathcal{A}, P) \rightarrow (\mathbb{R}, \mathbb{B})$ eine endlich integrierbare Zufallsvariable, 
	$\mathcal{C} \subseteq \mathcal{A}$ eine Unter-$\sigma$-Algebra und $Z$ eine $\mathcal{C},\mathbb{B}$-messbare Zufallsvariable.
\end{Aufgabe}

\beh Die folgenden beiden Aussagen sind äquivalent:
\begin{itemize}
	\item[(i)] $Z = E(X|\mathcal{C}) \;\;$ P-f.s.
	\item[(ii)] $E(XY) = E(ZY)$ für alle  $\mathcal{C},\mathbb{B}$-messbaren beschränkten Zufallsvariablen $Y$
\end{itemize}

\begin{proof}[Beweis]  
Es gelte (ii). Für eine beliebige Menge $D \in \mathcal{C}$ ist $\mathbbm{1}_D$ eine beschränkte $\mathcal{C},\mathbb{B}$-messbare Zufallsvariable und es gilt
$$
\int_{D} X \,dP = \int X \mathbbm{1}_D \,dP = E(X \mathbbm{1}_D) = E(Z \mathbbm{1}_D) =  \int Z \mathbbm{1}_D \,dP = \int_{D} Z \,dP
$$
Da $Z$ nach Voraussetzung $\mathcal{C},\mathbb{B}$-messbar ist, ist $Z$ ein bedingter Erwartungswert von $X$ gegeben $\mathcal{C}$. \\

Es gelte (i). Wir zeigen die Gültigkeit von (ii) durch algebraische Induktion. \\
Sei zuerst $Y \coloneqq \sum_{i=1}^{n} a_i \mathbbm{1}_{A_i}$ eine Elementarfunktion mit $a_i > 0$ und paarweise disjunkten Mengen $A_i \in \mathcal{C}$ für $i=1, \dots n$. Dann gilt
\begin{align*}
E(XY) &= E(X  \sum_{i=1}^{n} a_i \mathbbm{1}_{A_i}) =  \sum_{i=1}^{n} a_i E(X \mathbbm{1}_{A_i}) =  \sum_{i=1}^{n} a_i \int_{A_i} X dP \\
	  &=  \sum_{i=1}^{n} a_i \int_{A_i} Z dP =  \sum_{i=1}^{n} a_i E(Z \mathbbm{1}_{A_i}) =  E(Z \sum_{i=1}^{n} a_i \mathbbm{1}_{A_i}) = E(ZY) 
\end{align*}
wobei die Linearität des Erwartungswertes und die Radon-Nikodym-Gleichungen verwendet wurden. \\
Ist $Y:(\Omega, \mathcal{A}) \rightarrow (\mathbb{R}, \mathbb{B})$ eine beliebige nichtnegative beschränkte $\mathcal{C},\mathbb{B}$-messbare Zufallsvariable, dann gibt es eine Folge von Elementarfunktionen $(Y_n)_{n \in \mathbb{N}}$, die monoton von unten gegen $Y$ konvergiert. Es gilt dann $XY_n \rightarrow XY$ und $ZY_n \rightarrow ZY$ für $n \to \infty$ und, wie soeben gesehen, $E(XY_n) = E(ZY_n)$ für alle $n \in \mathbb{N}$. Mit $Y$ sind auch alle $Y_n$ beschränkt und, da $X$ und $Z$ endlich integrierbar sind, sind die Folgen $XY_n$ und $ZY_n$ gleichgradig $P$-integrierbar. Es folgt
$$
E(XY) = E( \lim_{n \to \infty} XY_n) =  \lim_{n \to \infty} E(X Y_n) =  \lim_{n \to \infty} E(Z Y_n) = E(\lim_{n \to \infty} ZY_n) = E(ZY)
$$
Ist schliesslich $Y$ eine beliebige beschränkte $\mathcal{C},\mathbb{B}$-messbare Zufallsvariable, dann folgt die Aussage nach Zerlegung in Positiv- und Negativteil und der Linearität des Erwartungswertes:
$$
E(XY) = E(XY^+)-E(XY^-) = E(ZY^+)-E(ZY^-) = E(ZY) 
$$
 
\end{proof}
 


\end{document}












