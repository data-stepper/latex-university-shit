\documentclass[10pt]{article}
 
\usepackage[margin=1in]{geometry} 
\usepackage{amsmath,amsthm,amssymb, graphicx, multicol, array}
 \usepackage{bbm, mathtools}
 
 
\newcommand{\R}{\mathbb{R}}
\newcommand{\N}{\mathbb{N}}
\newcommand{\Z}{\mathbb{Z}}
\newcommand{\beh}{\textit{Behauptung. }}

\setlength{\parindent}{0pt}
 
\newenvironment{Aufgabe}[2][Aufgabe]{\begin{trivlist}
\item[\hskip \labelsep {\bfseries #1}\hskip \labelsep {\bfseries #2.}]}{\end{trivlist}}
\newenvironment{amatrix}[1]{%
  \left(\begin{array}{@{}*{#1}{c}|c@{}}
}{%
  \end{array}\right)
}

\begin{document}
 
\title{ \textbf{Maßtheoretische Konzepte der Stochastik \\ -- Übungsblatt \#02 --} }

\author{Amir Miri Lavasani (7310114), Bent Müller (7302332),
        Michael Hermann (6981007)}
\maketitle

%\begin{theorem} % Aufgabe #1
\begin{Aufgabe}{1} % #1
	\[
	f: \left(
		\Omega, \mathcal{A}
	\right) \longrightarrow \left(
		\overline{\mathbb{R}}, \overline{\mathbb{B}}
	\right) , \;
	f(x) \geq 0 \; \forall x \in \mathbb{R} 
	\]
\end{Aufgabe}

\beh $\mu \{f= \infty \} > 0 \implies \int_{}^{} f \; d\mu = \infty $

\begin{proof}[Beweis]
	Wir schreiben unsere Definition vom Integral von $f$ zuerst aus.
	\begin{align*}
		\int_{}^{} f \; d\mu &= \sup_{n \in \mathbb{N} } \sum_{i=0}^{n 2 ^{n}}
		i 2 ^{-n} \mu \{
			i 2 ^{-n} \leq f < (i+1) 2 ^{-n}
		\} \; + \; n \mu \{
			f \geq n
		\} 
	\end{align*}

	Nun teilen wir unseren Grundraum auf wie folgt:
	\[
	\Omega_\infty := \{
		f = \infty
	\}  \; , \;
	\Omega_E := \Omega_\infty ^{c}
	= \Omega \setminus \Omega_\infty
	= \{
		f < \infty
	\} 
	\] 
	Nun bemerken wir, dass wir so auch die Summe in der Definition des Integrals
	aufteilen können wir folgt:
	\begin{align*}
		\int_{}^{} f \; d\mu &= 
		\sup_{n \in \mathbb{N} } \sum_{i=0}^{n 2 ^{-n}} i 2 ^{-n}
		\mu \{
			i 2 ^{-n} \leq f < (i+1) 2 ^{-n}
		\} \; + \; n \mu \{
			f \geq n
		\} \\
		 &= \lim_{n \to \infty} \sum_{i=0}^{n 2 ^{ -n }} ... +
		 n \mu \{
		 	f \geq n
		 \} = \infty
	\end{align*}

	Im oberen haben wir zusätzlich benutzt, dass wir hier das Supremum mit dem
	Limes austauschen dürfen (nach Skript). Letztendlich sieht man, dass
	wir eine unendliche Summe haben und nach Voraussetzung dieser Summand
	nicht 0 ist, also wird die Summe unendlich sein. Insbesondere gilt, dass
	für alle $n \in \mathbb{N} $ diese eine Menge welche von $f$ auf unendlich
	abgebildet wird immer größer als null gemessen wird.


	In der Umkehrung sehen wir, dass wenn die obere Summe unendlich war es zwei
	Möglichkeiten gab. Diese sind einmal, dass $\mu \{
		f = \infty
	\} > 0$ gemessen wird. Jedoch kann die Summe ebenfalls unendlich sein wenn
	wir ein Element haben welches von null verschieden ist, jedoch dessen Urbild
	von unserem Maß als unendlich gemessen wird $\mu \{
		f = a
	\} = \infty$ für ein passendes Maß und ein passendes Element $a > 0$.


	Die letzte Frage können wir im Allgemeinen nicht als wahr beantworten, da
	wir genau zwei Situationen haben welche beide eintreten können:
	\begin{eqnarray}
		&\mu \{
			f = \infty
		\} > 0 \; \text{und} \;
		\mu \{
			f = - \infty
		\} = 0 \implies \int_{}^{} f \; d\mu = \infty
	\end{eqnarray}
	\begin{eqnarray}
		&\mu \{
			f = \infty
		\} > 0 \; \text{und} \;
		\mu \{
			f = - \infty
		\} > 0 \implies \int_{}^{} f \; d\mu = \infty - \infty
		= \text{ nicht definiert }
	\end{eqnarray}
	Bei Gleichung (2) teilen wir nach Skript unsere Funktion auf jeweils in
	$f ^{+}$  und $f ^{-}$ und erhalten jeweils in beiden einzelnen Integralen
	$\infty$. Deshalb können wir nichts über den Wert des gesamten Integrals
	aussagen.
\end{proof}
%\end{theorem}


%\begin{theorem} % Aufgabe #2

	\begin{Aufgabe}{2}
		Sei $(\Omega,\mathcal{A},\mu)$ ein Maßraum und sei $f: (\Omega,\mathcal{A})\to (\overline{\R} , \overline{\mathbb{B}})$
		diskret. 
	\end{Aufgabe}

	\beh $f$ ist genau dann $\mu$-integrierbar, wenn $\sum_{a\in f(\Omega)} a\mu\{f=a\}$ 
	wohldefiniert ist, und in dem Fall gilt 
	\begin{align*}
		\int f \,d\mu = \sum_{a\in f(\Omega)} a\mu\{f=a\}.
	\end{align*}
	\begin{proof}[Beweis]
		Sei zunächst $f$ $\mu$-integrierbar. Das heißt mindestens eines der Integrale 
		$\int f^{+} \,d\mu$ oder $\int f^{-} \,d\mu$ ist endlich.
		Wir zeigen zunächst, dass in diesem Fall
		\begin{align*}
			\int f \,d\mu = \sum_{a\in f(\Omega)} a\mu\{f=a\}
		\end{align*}

		gilt. Da $f(\Omega)$ diskret ist, können wir diese Menge schreiben als 
		$f(\Omega) = \{ a_1,a_2,\dots \}$, d.\,h. wir können die Elemente von $f(\Omega)$  
		ordnen. Wir teilen die Menge auf in negative und positive Werte:
		\begin{align*}
			f(\Omega)^{+} &:= f(\Omega)\cap(0,\infty], \\
			f(\Omega)^{-} &:= f(\Omega)\cap[-\infty,0).
		\end{align*}

		Definiere:
		\begin{align*}
			A^{+} &:= f(\Omega)^{+} \,,  \\
			A^{-} &:= -f(\Omega)^{-}.
		\end{align*}

		Außerdem setzen wir:
		\begin{align*}
			A_i^{+} &:= \{ f = a_i^+ \} \quad\text{\hspace{0.15cm} für $a_i^+\in A^+, i\in\N$}, \\
			A_i^{-} &:= \{ f = -a_i^- \} \quad\text{für $a_i^-\in A^-, i\in\N$}.
		\end{align*}

		Hierbei benutzen wir, dass $A_i^{+}$ und $A_i^{-}$ als Teilmengen einer diskreten Menge 
		auch diskret sind und wir daher die Elemente dieser Mengen ordnen können.  
		Als nächstes definieren wir Elementarfunktionen $f_n^\pm : (\Omega,\mathcal{A})\to (\overline{\R} , \overline{\mathbb{B}})$ durch:
		\begin{align*}
			f^{+}_n &:= \sum_{i=1}^{n} a_i^+ \text{\textbf{1}}_{A^{+}_i}, \\															   
			f^{-}_n &:= \sum_{i=1}^{n} a_i^- \text{\textbf{1}}_{A^{-}_i}.								   
		\end{align*}

		Dann gelten per Konstruktion:
		\begin{itemize}
			\item[(1)] $0\leq f_n^+ \leq f^{+}$, $0\leq f_n^- \leq f^{-}$\,,
			\item[(2)] $f_n^+ \leq f_{n+1}^+$, $f_n^- \leq f_{n+1}^-$\, und
			\item[(3)] $\lim_{n\to\infty} f_n^{+} = f^{+}$, $\lim_{n\to\infty} f_n^{-} = f^{-}$.
		\end{itemize}

		Daraus folgen mit Lemma $2.6$ 
		\begin{align}
			\lim_{n\to\infty} \int f_n^{+} \,d\mu &= \int f^{+} \,d\mu \quad \text{ und} \\
			\lim_{n\to\infty} \int f_n^{-} \,d\mu &= \int f^{-} \,d\mu. 
		\end{align}

		Insgesamt erhalten wir:
		\begin{align*}
			\int f\,d\mu &\overset{\text{Def.}}{=} \int f^+\,d\mu - \int f^-\,d\mu \\
						 &\overset{(1),(2)}{=} \lim_{n\to\infty} \int f_n^{+} \,d\mu - \lim_{n\to\infty} \int f_n^{-} \,d\mu \\
						 &\overset{\text{Def.}}{=}  \lim_{n\to\infty}\sum_{i=1}^{n} a_i^{+}\mu(A_i^+) - \lim_{n\to\infty}\sum_{i=1}^{n} a_i^{-}\mu(A_i^-) \\
						 &= \sum_{i=1}^{\infty} a_i^{+}\mu(A_i^+) - \sum_{i=1}^{\infty} a_i^{-}\mu(A_i^-) \\
						 &= \sum_{a\in A^+} a\mu\{f = a\} + \sum_{a\in A^-} (-a)\mu\{f = a\} \\
						 &= \sum_{a\in f(\Omega)^+} a\mu\{f = a\} + \sum_{a\in f(\Omega)^-} a\mu\{f = a\} + 0\cdot \mu\{f = 0\}\\
						 &\overset{(3)}{=} \sum_{a\in f(\Omega)} a \mu\{f = a\},
		\end{align*}

		wobei bei (3) benutzt wird, dass $f(\Omega)^+$ und $f(\Omega)^-$
		(und evtl. auch $\{0\}$, falls $0\in f(\Omega)$) die Menge $f(\Omega)$ disjunkt zerlegen.  
		Aus dieser Darstellung folgt sofort, dass wenn $\int f\,d\mu$ $\mu$-integrierbar ist, dann auch
		$\sum_{a\in f(\Omega)} a \mu\{f = a\}$ es ist. Für die andere Implikationsrichtung können wir 
		nicht direkt davon ausgehen, dass die obige Gleichung gilt, da wir dies nur für den Fall gezeigt haben, 
		wenn $\int f\,d\mu$ $\mu$-integrierbar ist. Was wir nun aber tun können, ist die Gleichungskette rückwerts 
		hochzugehen. Da wir nun davon ausgehen, dass $\sum_{a\in f(\Omega)} a \mu\{f = a\}$ wohldefiniert ist, 
		können wir die Reihe ohne Probleme auseinanderziehen. Insbesondere ist der Schritt in der vierten Gleichheit 
		(von oben) zulässig.
		Nun folgt wieder direkt, dass auch $\int f\,d\mu$ $\mu$-integrierbar ist. 
		Damit ist die Behauptung bewiesen.
	\end{proof}
%\end{theorem}

\begin{Aufgabe}{3}
Sei $f:(\Omega, \mathcal{A}) \rightarrow (\bar{\mathbb{R}}, \bar{\mathbb{B}})$ eine nicht-negative numerische Funktion auf dem Maßraum $(\Omega, \mathcal{A}, \mu)$. \\

\beh 
Die folgenden Aussagen sind äquivalent:

\begin{itemize}
	\item[(i)]  $\int f \; d\mu = 0$ 
	\item[(ii)]  $f = 0$ $\mu$-fast-überall 
\end{itemize}
\end{Aufgabe}

\begin{proof}[Beweis]
% Ich denke, diesen Fall kann man weglassen:
%Ist $f$ eine Elementarfunktion, dh. von der Form $f = \sum_{i \in \mathbb{N}} a_i \mathbbm{1}_{A_i}$ für disjunkte Mengen $A_i \subset \mathbb{R}$ mit $\bigcup_{i \in \mathbb{N}} A_i = \Omega$ und $a_i \geq 0$ für alle $i \in \mathbb{N}$,  
%so gilt $\int f \; d\mu = \sum_{i \in \mathbb{N}} a_i \mu\{A_i\}$. Das heißt, es gilt $\int f \; d\mu = 0$ genau dann, wenn alle Terme in der Summe gleich Null sind. Dies ist genau dann der Fall, falls $\mu\{A_i\} = 0$ für alle $i$ mit $a_i > 0$, also wenn $f = 0$ $\mu$-fast-überall. \\

Ist $f$ eine beliebige nicht-negative numerische Funktion, dann kann man das Integral durch
$$
\int f \; d\mu \coloneqq \sup_{n \in \mathbb{N}} \int f^{(n)} \; d\mu
$$
mit
$$
f^{(n)} \coloneqq  \sum_{i=1}^{n2^n-1} i2^{-n} \mathbbm{1}_{\{i2^{-n} \leq f < (i+1)2^{-n}\}} + n\mathbbm{1}_{\{f \geq n\}}
$$
definieren.
Gilt (i), also $\int f \; d\mu = 0$, dann folgt mit Lemma 2.5 für alle $n \in \mathbb{N}$ auch $\int f^{(n)} \; d\mu = 0$, da $f^{(n)} \leq f$, also 
$$
\sum_{i=1}^{n2^n-1} i2^{-n} \mu\{i2^{-n} \leq f < (i+1)2^{-n}\} + n\mu\{f \geq n\} =  0
$$ 
für alle $n \in \mathbb{N}$. Diese Summe kann nur gleich Null sein, wenn alle beteiligten Maße gleich Null sind, da die anderen Faktoren in allen Termen positiv sind. Mit
$$
\{f \neq 0\} = \bigcup_{n \in \mathbb{N}} \bigcup_{1 \leq i \leq n2^n-1} \{i2^{-n} \leq f < (i+1)2^{-n}\} \cup \{f \geq n\}
$$ 
und der $\sigma$-Subadditivität von $\mu$ folgt $\mu\{f \neq 0\} = 0$, also (ii). \\
Es gelte jetzt (ii). Wir können die Funktion $f$ auch als Summe $f = f\mathbbm{1}_{\{f = 0\}} + f\mathbbm{1}_{\{f \neq 0\}}$ schreiben. Aus der Linearität des Integrals folgt
$$
\int f \; d\mu = \int f\mathbbm{1}_{\{f = 0\}} \; d\mu + \int f\mathbbm{1}_{\{f \neq 0\}} \; d\mu = 0
$$
da beide Summanden gleich Null sind. Für den Ersten ist es offensichtlich, beim Zweiten gilt es, da die elementare numerische Funktion $\omega \mapsto \mathbbm{1}{\{f \neq 0\}}(\omega) \cdot \infty$ eine Majorante mit Integral gleich Null ist. \\
\end{proof} 


\end{document}
