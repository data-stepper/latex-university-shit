\documentclass[10pt]{article}
 
\usepackage[margin=1in]{geometry} 
\usepackage{amsmath,amsthm,amssymb, graphicx, multicol, array}
 
\newcommand{\R}{\mathbb{R}}
\newcommand{\N}{\mathbb{N}}
\newcommand{\Z}{\mathbb{Z}}
\newcommand{\beh}{\textit{Behauptung. }}

\setlength{\parindent}{0pt}
 
\newenvironment{Aufgabe}[2][Aufgabe]{\begin{trivlist}
\item[\hskip \labelsep {\bfseries #1}\hskip \labelsep {\bfseries #2.}]}{\end{trivlist}}
\newenvironment{amatrix}[1]{%
  \left(\begin{array}{@{}*{#1}{c}|c@{}}
}{%
  \end{array}\right)
}

\begin{document}
 
\title{ \textbf{Maßtheoretische Konzepte der Stochastik \\ -- Übungsblatt \#02 --} }

\author{Amir Miri Lavasani (7310114), Bent Müller (7302332),
        Michael Hermann (MATRIKELNUMMER)} \maketitle

 
\begin{Aufgabe}{2}
    Sei $(\Omega,\mathcal{A},\mu)$ ein Maßraum und sei $f: (\Omega,\mathcal{A})\to (\overline{\R} , \overline{\mathbb{B}})$
    diskret. 
\end{Aufgabe}

\beh $f$ ist genau dann $\mu$-integrierbar, wenn $\sum_{a\in f(\Omega)} a\mu\{f=a\}$ 
wohldefiniert ist, und in dem Fall gilt 
  \begin{align*}
    \int f \,d\mu = \sum_{a\in f(\Omega)} a\mu\{f=a\}.
  \end{align*}
\begin{proof}[Beweis]
  Sei zunächst $f$ $\mu$-integrierbar. Das heißt mindestens eines der Integrale 
  $\int f^{+} \,d\mu$ oder $\int f^{-} \,d\mu$ ist endlich.
  Wir zeigen zunächst, dass in diesem Fall
  \begin{align*}
    \int f \,d\mu = \sum_{a\in f(\Omega)} a\mu\{f=a\}
  \end{align*}

  gilt. Da $f(\Omega)$ diskret ist, können wir diese Menge schreiben als 
  $f(\Omega) = \{ a_1,a_2,\dots \}$, d.\,h. wir können die Elemente von $f(\Omega)$  
  ordnen. Wir teilen die Menge auf in negative und positive Werte:
  \begin{align*}
    f(\Omega)^{+} &:= f(\Omega)\cap(0,\infty], \\
    f(\Omega)^{-} &:= f(\Omega)\cap[-\infty,0).
  \end{align*}
  
  Definiere:
  \begin{align*}
    A^{+} &:= f(\Omega)^{+} \,,  \\
    A^{-} &:= -f(\Omega)^{-}.
  \end{align*}

  Außerdem setzen wir:
  \begin{align*}
    A_i^{+} &:= \{ f = a_i^+ \} \quad\text{für $a_i^+\in A^+, i\in\N$}, \\
    A_i^{-} &:= \{ f = a_i^- \} \quad\text{für $a_i^-\in A^-, i\in\N$}.
  \end{align*}

  Hierbei benutzen wir, dass $A_i^{+}$ und $A_i^{-}$ als Teilmengen einer diskreten Menge 
  auch diskret sind und wir daher die Elemente dieser Mengen ordnen können.  
  Als nächstes definieren wir folgende Elementarfunktionen:
  \begin{align*}
      f^{+}_n(\omega) &:= \sum_{i=1}^{n} a_i^+ \text{\textbf{1}}_{A^{+}_i}(\omega), \\                                                               
      f^{-}_n(\omega) &:= \sum_{i=1}^{n} a_i^- \text{\textbf{1}}_{A^{-}_i}(\omega).                                   
  \end{align*}

  Dann gelten per Konstruktion:
  \begin{itemize}
    \item[(1)] $f_n^+ \leq f^{+}$, $f_n^- \leq f^{-}$\,,
    \item[(2)] $f_n^+ \leq f_{n+1}^+$, $f_n^- \leq f_{n+1}^-$\, und
    \item[(3)] $\lim_{n\to\infty} f_n^{+} = f^{+}$, $\lim_{n\to\infty} f_n^{-} = f^{-}$.
  \end{itemize}

  Daraus folgen mit Lemma $2.6$ 
  \begin{align}
    \lim_{n\to\infty} \int f_n^{+} \,d\mu &= \int f^{+} \,d\mu \quad \text{ und} \\
    \lim_{n\to\infty} \int f_n^{-} \,d\mu &= \int f^{-} \,d\mu. 
  \end{align}

  Insgesamt erhalten wir:
  \begin{align*}
    \int f\,d\mu &\overset{\text{Def.}}{=} \int f^+\,d\mu - \int f^-\,d\mu \\
                 &\overset{(1),(2)}{=} \lim_{n\to\infty} \int f_n^{+} \,d\mu - \lim_{n\to\infty} \int f_n^{-} \,d\mu \\
                 &\overset{\text{Def.}}{=}  \lim_{n\to\infty}\sum_{i=1}^{n} a_i^{+}\mu(A_i^+) - \lim_{n\to\infty}\sum_{i=1}^{n} a_i^{-}\mu(A_i^-) \\
                 &= \sum_{i=1}^{\infty} a_i^{+}\mu(A_i^+) - \sum_{i=1}^{\infty} a_i^{-}\mu(A_i^-) \\
                 &= \sum_{a\in A^+} a\mu\{f = a\} + \sum_{a\in A^-} (-a)\mu\{f = a\} \\
                 &= \sum_{a\in f(\Omega)^+} a\mu\{f = a\} + \sum_{a\in f(\Omega)^-} a\mu\{f = a\} + 0\cdot \mu\{f = 0\}\\
                 &\overset{(3)}{=} \sum_{a\in f(\Omega)} a \mu\{f = a\},
  \end{align*}

  wobei bei (3) benutzt wird, dass $f(\Omega)^+$ und $f(\Omega)^-$
  (und evtl. auch $\{0\}$, falls $0\in f(\Omega)$) die Menge $f(\Omega)$ disjunkt zerlegen.  
  Aus dieser Darstellung folgt sofort, dass wenn $\int f\,d\mu$ $\mu$-integrierbar ist, dann auch
  $\sum_{a\in f(\Omega)} a \mu\{f = a\}$ es ist. Für die andere Implikationsrichtung können wir 
  nicht direkt davon ausgehen, dass die obige Gleichung gilt, da wir dies nur für den Fall gezeigt haben, 
  wenn $\int f\,d\mu$ $\mu$-integrierbar ist. Was wir nun aber tun können, ist die Gleichungskette rückwerts 
  hochzugehen. Da wir nun davon ausgehen, dass $\sum_{a\in f(\Omega)} a \mu\{f = a\}$ wohldefiniert ist, 
  können wir die Reihe ohne Probleme auseinanderziehen. Insbesondere ist der Schritt in der vierten Gleichheit 
  (von oben) zulässig.
  Nun folgt wieder direkt, dass auch $\int f\,d\mu$ $\mu$-integrierbar ist. 
  Damit ist die Behauptung bewiesen.
\end{proof}


\end{document}