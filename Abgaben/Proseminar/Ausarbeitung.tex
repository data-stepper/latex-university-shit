\documentclass[a4paper]{article}

\usepackage[margin=1in]{geometry} 
\usepackage{amsmath,amsthm,amssymb, graphicx, multicol, array}
\usepackage[german]{babel}

\usepackage{tikz}
\usetikzlibrary{automata, positioning}

\pdfminorversion=7
\pdfsuppresswarningpagegroup=1

\newcommand{\R}{\mathbb{R}}
\newcommand{\N}{\mathbb{N}}
\newcommand{\Z}{\mathbb{Z}}
\newcommand{\beh}{\textit{Behauptung. }}

\setlength{\parindent}{0pt}

\begin{document}

\begin{titlepage}
   \begin{center}
       \vspace*{1cm}

       \textbf{Ausarbeitung Proseminar}

       \vspace{0.5cm}
	   \section*{
		   Langzeitverhalten von Markowketten
	   }
            
       \vspace{1.5cm}

	   \textbf{Nogarbek Sharykpayev (123),\\ Bent Müller (7302332)}

       \vfill
            
       \vspace{0.8cm}
     
            
       Fachbereich Mathematik\\
       Universität Hamburg\\
       Deutschland\\
       17.08.2021
            
   \end{center}
\end{titlepage}

\pagebreak
\tableofcontents
\pagebreak

\section{Kleine Wiederholung - Markowketten}

Bevor wir uns mit dem Langzeitverhalten der Markow-Ketten beschäftigen, folgt hier eine kleine Wiederholung:
\\

Der aktuelle Zustand einer Markow-Kette besitzt die Eigenschaft von Zuständen aus mindestens 2 Schritten aus der Vergangenheit unabhängig zu sein, also wenn gilt:
\\

1. Es sei $\{X_n : n \in \mathbb{Z} \}$ ein stochastischer Prozess mit dem abzählbaren Zustandsraum $I$ und es gilt für alle $i_0, …, i_{n+1}$ Element in $I$ : 

\end{document}
