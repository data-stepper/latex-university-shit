\documentclass{beamer}

%Information to be included in the title page:
\title{Langzeitverhalten von Markowketten}
\author{Nogarbek Sharykpayev, Bent Müller}
\institute{Universität Hamburg}
\date{15.06.2021}

\usepackage{tikz}
\usetikzlibrary{automata, positioning}

\usepackage{amsmath, amsthm}
\setbeamertemplate{headline}[default]
\setbeamertemplate{navigation symbols}{}
\mode<beamer>{\setbeamertemplate{blocks}[rounded][shadow=true]}
\setbeamercovered{transparent}
\setbeamercolor{block body example}{fg=blue, bg=black!20}

\useoutertheme[subsection=false]{miniframes}
\usetheme{default}

\begin{document}

\frame{\titlepage}
\begin{frame}
	\frametitle{Struktur der Präsentation}
	\tableofcontents
\end{frame}

\section{Kurze Wiederholung}
\subsection{Homogene Markowketten}
\begin{frame}
	\frametitle{Wiederholung}
	\begin{center}
		Wir nennen eine Markowkette \textit{homolog} , wenn die Übergangswahrscheinlichkeiten
		\textit{unabhängig} vom Zeitpunkt sind.
	\end{center}
\end{frame}
\subsection{Übergangsmatrix}
\begin{frame}
	\frametitle{Wiederholung - Übergangsmatrix}
	In diesem Fall können wir eine Übergangsmatrix wie folgt bilden:
	\[
		\mathbb{P} = \left(
		p_{ij}
		\right) = \begin{pmatrix}
			p_{11} & \cdots & p_{1n} \\
			\vdots & \ddots & \vdots \\
			p_{n1} & \cdots & p_{nn} \\
		\end{pmatrix}
	\]
	\only<2>{
		Wobei hier der Zustandsraum $n$ Elemente enthält, und jeweils $p_{ij}$ die Wahrscheinlichkeit
		ist in einem \textit{beliebigen} Zeitpunkt von Zustand $i$ in Zustand $j$ zu gelangen.
	}
\end{frame}

\section{Konvergenz der Übergangsmatrix}
\subsection{Invariante Wahrscheinlichkeitsverteilung}
\begin{frame}[t]
	\frametitle{Invariante Verteilung}
	Wir betrachten eine Markowkette mit Übergangsgraphen:
	\begin{center}
		\begin{tikzpicture}
			% Add the states
			\node[state]			 (a) {1};
			\node[state, above right=of a] (b) {2};
			\node[state, right=of b] (c) {3};
			\node[state, below right=of c] (d) {4};
			\node[state, below left=of d] (e) {5};
			\node[state, left=of e] (f) {6};

			% Connect the states with arrows
			\draw[every loop]
			% a -> b -> c -> d
			(a) edge[bend right, auto=right] node {} (b)
			% (b) edge[bend right, auto=right] node {} (c)
			(c) edge[bend right, auto=right] node {} (d)
			(d) edge[bend right, auto=right] node {} (e)
			(e) edge[bend right, auto=right] node {} (f)
			(f) edge[bend right, auto=right] node {} (a)

			% d -> c -> b -> a
			(d) edge[bend right, auto=right] node {} (c)
			(c) edge[bend right, auto=right] node {} (b)
			(b) edge[bend right, auto=right] node {} (a)
			(a) edge[bend right, auto=right] node {} (f)
			(f) edge[bend right, auto=right] node {} (e)
			(e) edge[bend right, auto=right] node {} (d)

			% a -> a, b -> b, ...
			(a) edge[loop left]			     node {} (a)
			(b) edge[loop above]			 node {} (b)
			(c) edge[loop above]			 node {} (c)
			(d) edge[loop right]			 node {} (d)
			(e) edge[loop below]			 node {} (e)
			(f) edge[loop below]			 node {} (f)
		\end{tikzpicture}
	\end{center}
	\\
\end{frame}

\begin{frame}
	\frametitle{Invariante Verteilung}
	Die Übergangsmatrix sei Folgende:
	\begin{align*}
		\mathbb{P} =
		\begin{pmatrix}
			0.60 & 0.15 & 0.00 & 0.00 & 0.00 & 0.25 \\
			0.25 & 0.75 & 0.00 & 0.00 & 0.00 & 0.00 \\
			0.00 & 0.40 & 0.50 & 0.10 & 0.00 & 0.00 \\
			0.00 & 0.00 & 0.35 & 0.60 & 0.05 & 0.00 \\
			0.00 & 0.00 & 0.00 & 0.25 & 0.50 & 0.25 \\
			0.25 & 0.00 & 0.00 & 0.00 & 0.25 & 0.50
		\end{pmatrix}
	\end{align*}
	\only<2-3>{
		Wir behaupten, es gibt ein $L \in \mathbb{N}$, sodass alle Einträge
		von $\mathbb{P} ^{(L)}$ strikt positiv sind. \\
	}
	\only<3>{
		\vspace{5mm}
		Wir schauen uns jetzt ein paar Potenzen dieser Matrix an.
	}
\end{frame}

\begin{frame}
	\frametitle{Potenzen der Übergangsmatrix}
	\begin{center}
		\begin{align*}
			\mathbb{P} ^{(2)} =
			\begin{pmatrix}
				0.4600 & 0.2025 & 0.0000 & 0.0000 & 0.0625 & 0.2750 \\
				0.3375 & 0.6000 & 0.0000 & 0.0000 & 0.0000 & 0.0625 \\
				0.1000 & 0.5000 & 0.2850 & 0.1100 & 0.0050 & 0.0000 \\
				0.0000 & 0.1400 & 0.3850 & 0.4075 & 0.0550 & 0.0125 \\
				0.0625 & 0.0000 & 0.0875 & 0.2750 & 0.3250 & 0.2500 \\
				0.2750 & 0.0375 & 0.0000 & 0.0625 & 0.2500 & 0.3750
			\end{pmatrix}
		\end{align*}
	\end{center}
\end{frame}

\begin{frame}
	\frametitle{Potenzen der Übergangsmatrix}
	\begin{center}
		\begin{align*}
			\mathbb{P} ^{(3)} =
			\begin{pmatrix}
				0.3954 & 0.2209 & 0.0000 & 0.0156 & 0.1000 & 0.2681 \\
				0.3681 & 0.5006 & 0.0000 & 0.0000 & 0.0156 & 0.1156 \\
				0.1850 & 0.5040 & 0.1810 & 0.0958 & 0.0080 & 0.0263 \\
				0.0381 & 0.2590 & 0.3351 & 0.2967 & 0.0510 & 0.0200 \\
				0.1000 & 0.0444 & 0.1400 & 0.2550 & 0.2388 & 0.2219 \\
				0.2681 & 0.0694 & 0.0219 & 0.1000 & 0.2219 & 0.3187
			\end{pmatrix}
		\end{align*}
	\end{center}
\end{frame}

\begin{frame}
	\frametitle{Potenzen der Übergangsmatrix}
	\begin{center}
		\begin{align*}
			\mathbb{P} ^{(4)} =
			\begin{pmatrix}
				0.3595 & 0.2250 & 0.0055 & 0.0344 & 0.1178 & 0.2579 \\
				0.3749 & 0.4307 & 0.0000 & 0.0039 & 0.0367 & 0.1538 \\
				0.2436 & 0.4781 & 0.1240 & 0.0775 & 0.0153 & 0.0614 \\
				0.0926 & 0.3340 & 0.2714 & 0.2243 & 0.0453 & 0.0323 \\
				0.1266 & 0.1043 & 0.1593 & 0.2267 & 0.1876 & 0.1956 \\
				0.2579 & 0.1010 & 0.0459 & 0.1177 & 0.1956 & 0.2819
			\end{pmatrix}
		\end{align*}
	\end{center}
\end{frame}

\begin{frame}
	\frametitle{Potenzen der Übergangsmatrix}
	\begin{center}
		\begin{align*}
			\mathbb{P} ^{(5)} =
			\begin{pmatrix}
				0.3364 & 0.2248 & 0.0148 & 0.0506 & 0.1251 & 0.2483 \\
				0.3711 & 0.3793 & 0.0014 & 0.0115 & 0.0570 & 0.1798 \\
				0.2810 & 0.4448 & 0.0891 & 0.0628 & 0.0269 & 0.0954 \\
				0.1472 & 0.3730 & 0.2142 & 0.1731 & 0.0420 & 0.0506 \\
				0.1509 & 0.1609 & 0.1590 & 0.1988 & 0.1540 & 0.1764 \\
				0.2505 & 0.1328 & 0.0641 & 0.1241 & 0.1742 & 0.2543
			\end{pmatrix}
		\end{align*}
	\end{center}
	\only<2-3>{
		Also sehen wir, dass unsere Behauptung für $L = 5$ stimmt. \\
	}
	\only<3>{
		\vspace{5mm}
		Nach $5$ Schritten ist es also möglich von jedem Zustand in jeden anderen Zustand
		zu gelangen.
	}
	\only<4>{
		Weiter behaupten wir, dass die Übergangsmatrix jetzt konvergiert
		wenn wir die Potenz gegen $\infty$ laufen lassen.
	}
\end{frame}

\begin{frame}
	\frametitle{Potenzen der Übergangsmatrix}
	\begin{center}
		\begin{align*}
			\mathbb{P} ^{(10)} =
			\begin{pmatrix}
				0.2893 & 0.2395 & 0.0532 & 0.0850 & 0.1199 & 0.2131 \\
				0.3213 & 0.2652 & 0.0292 & 0.0589 & 0.1086 & 0.2168 \\
				0.3218 & 0.3119 & 0.0365 & 0.0530 & 0.0863 & 0.1904 \\
				0.2926 & 0.3478 & 0.0688 & 0.0698 & 0.0680 & 0.1531 \\
				0.2518 & 0.2880 & 0.0950 & 0.1064 & 0.0951 & 0.1636 \\
				0.2598 & 0.2433 & 0.0802 & 0.1059 & 0.1170 & 0.1939
			\end{pmatrix}
		\end{align*}
	\end{center}
\end{frame}

\begin{frame}
	\frametitle{Potenzen der Übergangsmatrix}
	\begin{center}
		\begin{align*}
			\mathbb{P} ^{(20)} =
			\begin{pmatrix}
				0.2882 & 0.2653 & 0.0587 & 0.0828 & 0.1074 & 0.1977 \\
				0.2884 & 0.2609 & 0.0577 & 0.0831 & 0.1096 & 0.2003 \\
				0.2918 & 0.2608 & 0.0547 & 0.0807 & 0.1097 & 0.2023 \\
				0.2958 & 0.2648 & 0.0518 & 0.0773 & 0.1079 & 0.2023 \\
				0.2935 & 0.2705 & 0.0548 & 0.0783 & 0.1051 & 0.1978 \\
				0.2899 & 0.2694 & 0.0578 & 0.0810 & 0.1055 & 0.1963
			\end{pmatrix}
		\end{align*}
	\end{center}
\end{frame}

\begin{frame}
	\frametitle{Potenzen der Übergangsmatrix}
	\begin{center}
		\begin{align*}
			\mathbb{P} ^{(200)} =
			\begin{pmatrix}
				0.2900 & 0.2652 & 0.0570 & 0.0815 & 0.1075 & 0.1988 \\
				0.2900 & 0.2652 & 0.0570 & 0.0815 & 0.1075 & 0.1988 \\
				0.2900 & 0.2652 & 0.0570 & 0.0815 & 0.1075 & 0.1988 \\
				0.2900 & 0.2652 & 0.0570 & 0.0815 & 0.1075 & 0.1988 \\
				0.2900 & 0.2652 & 0.0570 & 0.0815 & 0.1075 & 0.1988 \\
				0.2900 & 0.2652 & 0.0570 & 0.0815 & 0.1075 & 0.1988
			\end{pmatrix}
		\end{align*}
	\end{center}
\end{frame}

\begin{frame}[t]
	\frametitle{Potenzen der Übergangsmatrix - Invariante Verteilung}
	Wir nennen den Zeilenvektor gegen welche die Übergangsmatrix konvergiert
	\textbf{invariante Verteilung}, beziehungsweise $\rho$.
	\only<2-3>{
		\begin{align*}
			\mathbb{P} ^{(200)} =
			\begin{pmatrix}
				0.2900 & 0.2652 & 0.0570 & 0.0815 & 0.1075 & 0.1988 \\
				0.2900 & 0.2652 & 0.0570 & 0.0815 & 0.1075 & 0.1988 \\
				0.2900 & 0.2652 & 0.0570 & 0.0815 & 0.1075 & 0.1988 \\
				0.2900 & 0.2652 & 0.0570 & 0.0815 & 0.1075 & 0.1988 \\
				0.2900 & 0.2652 & 0.0570 & 0.0815 & 0.1075 & 0.1988 \\
				0.2900 & 0.2652 & 0.0570 & 0.0815 & 0.1075 & 0.1988
			\end{pmatrix}
		\end{align*}
	}
	\only<3->{
		\[
			\Rightarrow \rho = \left(
			0.29,  0.2652 ,  0.057 ,  0.0815 ,  0.1075 ,  0.1988
			\right)
		\]
	}
	\only<4->{
		Dieser sagt uns, unabhängig vom Startpunkt, wie wahrscheinlich es ist
		zu einem beliebigen Zeitpunkt in einem bestimmten Zustand zu sein.
	}
\end{frame}

\begin{frame}[t]
	\frametitle{Invariante Verteilung der Markowkette}
	\begin{center}
		\begin{tikzpicture}
			% Add the states
			\node[state]			 (a) {1};
			\node[state, above right=of a] (b) {2};
			\node[state, right=of b] (c) {3};
			\node[state, below right=of c] (d) {4};
			\node[state, below left=of d] (e) {5};
			\node[state, left=of e] (f) {6};

			% Connect the states with arrows
			\draw[every loop]
			% a -> b -> c -> d
			(a) edge[bend right, auto=right] node {} (b)
			% (b) edge[bend right, auto=right] node {} (c)
			(c) edge[bend right, auto=right] node {} (d)
			(d) edge[bend right, auto=right] node {} (e)
			(e) edge[bend right, auto=right] node {} (f)
			(f) edge[bend right, auto=right] node {} (a)

			% d -> c -> b -> a
			(d) edge[bend right, auto=right] node {} (c)
			(c) edge[bend right, auto=right] node {} (b)
			(b) edge[bend right, auto=right] node {} (a)
			(a) edge[bend right, auto=right] node {} (f)
			(f) edge[bend right, auto=right] node {} (e)
			(e) edge[bend right, auto=right] node {} (d)

			% a -> a, b -> b, ...
			(a) edge[loop left]			     node {} (a)
			(b) edge[loop above]			 node {} (b)
			(c) edge[loop above]			 node {} (c)
			(d) edge[loop right]			 node {} (d)
			(e) edge[loop below]			 node {} (e)
			(f) edge[loop below]			 node {} (f)
		\end{tikzpicture}
	\end{center}
	\[
		\rho = \left(
		0.29,  0.2652 ,  0.057 ,  0.0815 ,  0.1075 ,  0.1988
		\right)
	\]
\end{frame}

\begin{frame}
	\frametitle{Invariante Verteilung - Intuition}
	\[
		\rho = \left(
		0.29,  0.2652 ,  0.057 ,  0.0815 ,  0.1075 ,  0.1988
		\right)
	\]
	\vspace{3mm}
	\begin{center}
		Lassen wir die Markowkette unendlich lange 'laufen', so
		halten wir uns $29\%$ der Zeit in Zustand $1$ auf.
	\end{center}
\end{frame}

\begin{frame}[t]
	\frametitle{Invariante Verteilung - Eigenschaft}
	$\rho$ als Zeilenvektor ist \textbf{linker}
	Eigenvektor der Übergangsmatrix $\mathbb{P}$:
	\[
		\rho = \rho \mathbb{P}
	\]
	\only<2>{
		Das heißt aber einfach:
		\[
			\rho = \mathbb{P} ^{T} \rho
		\]
	}
\end{frame}

\subsection{Beweis der Konvergenz}
\begin{frame}[t]
	\frametitle{Beweis der Konvergenz}
	\only<1>{
		\textit{Aussage:} Gibt es für eine Markowkette mit endlich vielen Zuständen ein $L \in \mathbb{N}$,
		sodass die L-Schritt-Übergangsmatrix $\mathbb{P} ^{(L)}$ nur strikt positive Elemente
		enthält, dann konvergieren
		die $p_{ij}$ gegen die von $i$ unabhängige Verteilung $\rho$. \\
	}
	\only<2->{
		\textit{Aussage:} $\exists L \in \mathbb{N} : \mathbb{P} ^{(L)} > 0$
		$\Rightarrow \lim_{L \to \infty} \mathbb{P} ^{(L)}$ konvergiert. \\
		\textit{Beweis:} \\
		\vspace{5mm}
	}
	\only<2-3>{
	Wir setzen
	\[
		m_j ^{(n)} = \min_{i} p_{ij} ^{(n)}
		\;\;\;
		\text{ und }
		\;\;\;
		M_j ^{(n)} = \max_i p_{ij} ^{(n)}.
	\]
	}
	\only<3>{
	Dann gilt
	\[
		m_j ^{(n+1)} = \min_i \sum_{h \in I} p_{ih} p_{hj} ^{(n)}
		\geq \min_i \sum_{h \in I} p_{ih} m_j ^{(n)} = m_j ^{(n)},
	\]
	und analog $M_j ^{(n+1)} \leq M_j ^{(n)}$.
	}
	\only<4-5>{
		Wir setzen außerdem
		\[
			\delta = \min_{(i,j) \in I ^2} p_{ij} ^{(L)} \ge 0.
		\]
	}
	\only<5>{
		Weiter setzen wir für feste $h, i \in I$, die Summe über alle Indize $k$ für welche
		$p_{hk} ^{(L)} \geq p_{ik} ^{(L)}$ gilt als $\sum_{k+}$ und $\sum_{k-}$
		für die übrigen Indize $k$.
	}
	\only<6-7>{
		Wir veranschaulichen dies an der Übergangsmatrix $\mathbb{P}$ von vorhin:
		\begin{align*}
			\begin{pmatrix}
				0.60   & 0.15   & 0.00   & 0.00   & 0.00   & 0.25   \\
				\cdots & \cdots & \cdots & \cdots & \cdots & \cdots \\
				0.00   & 0.40   & 0.50   & 0.10   & 0.00   & 0.00   \\
				\cdots & \cdots & \cdots & \cdots & \cdots & \cdots \\
				\cdots & \cdots & \cdots & \cdots & \cdots & \cdots \\
				\cdots & \cdots & \cdots & \cdots & \cdots & \cdots
			\end{pmatrix}
			\begin{matrix}
				\leftarrow \text{ h } \\
				\\
				\leftarrow i          \\
				\\
				\\
				\\
			\end{matrix}
		\end{align*}
	}
	\only<7>{
		Somit wären hier die beiden Indexmengen $k+$ und $k-$
		\[
			k+ = \{
			1, 5, 6
			\}, \quad
			k- = \{
			2, 3, 4
			\}.
		\]
	}
	\only<8-9>{
		Wir beobachten:
		\begin{align*}
			\sum_{k+} \left(
			p_{hk} ^{(L)} - p_{ik} ^{(L)}
			\right) + \sum_{k-} \left(
			p_{hk} ^{(L)} - p_{ik} ^{(L)}
			\right) = 1 - 1 = 0
		\end{align*}
	}
	\only<9-11>{
		Sei nun für festes $n$, h ein Zustand, sodass $p_{hj} ^{(n+L)}$ maximal ist und $i$
		ein Zustand, sodass $p_{ij} ^{(n+L)}$ minimal ist. \\
	}
	\only<10-11>{
		\vspace{5mm}
		\only<10-11>{
			Dann gilt:
		}
		\begin{align*}
			\only<10->{
			M_j ^{(n+L)} - m_j ^{(n+L)} & = p_{hj} ^{(n+L)}
			- p_{ij} ^{(n+L)}                               \\
			}
			\only<11->{
			                            & = \sum_{k} \left(
				p_{hk} ^{(L)} - p_{ik} ^{(L)}
			\right) p_{kj} ^{(n)}                           \\
			}
		\end{align*}
	}
	\only<12-13>{
		\begin{align*}
			 & = \sum_{k} \left(
			p_{hk} ^{(L)} - p_{ik} ^{(L)}
			\right) p_{kj} ^{(n)}    \\
			\only<12->{
			 & \leq \sum_{k+} \left(
			p_{hk} ^{(L)} - p_{ik} ^{(L)}
			\right) \left(
			M_j ^{(n)} - m_j ^{(n)}
			\right)                  \\
			}
			\only<13->{
			 & \leq \left(
			1- \delta
			\right) \left(
			M_j ^{(n)} - m_j ^{(n)}
			\right)                  \\
			}
		\end{align*}
	}
	\only<13->{
	Nun folgt induktiv für ein $a \geq 0$:
	\[
		M_j ^{(aL)} - m_j ^{(aL)} \leq \left(
		1- \delta
		\right) ^{a}
	\]
	}
	\only<14>{
	Da aber die Folge (über $n$) $M_j ^{(n)}$ fallend und $m_j ^{(n)}$ wachsend ist,
	konvergieren die Einträge der Matrix. $\qquad \qquad \qquad \qed$
	}
\end{frame}

\section{Kommunizieren und Periodizität}
\subsection{Kommunizierende Zustände}
\begin{frame}
	\frametitle{Begriffe}
	$i, j \in I$ seien Zustände
	\begin{align*}
		\only<1->{
		i \rightsquigarrow j [n]                               & \Leftrightarrow
		p_{ij} ^{(n)} > 0                                      & \Leftrightarrow
		\text{ i führt in n Schritten zu j }                                                                       \\
		}
		\only<2->{
		i \rightsquigarrow j                                   & \Leftrightarrow
		\exists n \in \mathbb{N} : i \rightsquigarrow j [n]    &                                                   \\
		}
		\only<3->{
		i \leftrightsquigarrow j                               & \Leftrightarrow
		i \rightsquigarrow j \text{ und } j \rightsquigarrow i & \Leftrightarrow
		\text{ i \textit{kommuniziert}  mit j }                                                                    \\
		}
		\only<4->{
		                                                       & \forall h \in I \text{ mit } i \rightsquigarrow h
		\Rightarrow h \rightsquigarrow i                       & \Leftrightarrow
			\text{ i ist \textit{wesentlich} }
		}
	\end{align*}
\end{frame}

\begin{frame}
	\frametitle{Wichtige Folgerungen}
	Kommunizieren ist eine \textbf{Äquivalenzrelation} auf der
	Teilmenge der wesentlichen Zustände.
	\\
	\only<2->{
		Es gilt für $i, j, k$ wesentliche Zustände in $I$:
		\begin{itemize}
			\item $i \leftrightsquigarrow i$ (reflexiv)
			\item $i \leftrightsquigarrow j \Rightarrow$
			      $j \leftrightsquigarrow i$ (symmetrisch)
			\item $i \leftrightsquigarrow j$ und
			      $j \leftrightsquigarrow k \Rightarrow
				      i \leftrightsquigarrow k$ (transitiv)
		\end{itemize}
	}
\end{frame}

\begin{frame}
	\frametitle{Wieso interessiert uns das?}
	\begin{center}
		\begin{tikzpicture}
			% Add the states
			\node[state]			 (a) {1};
			\node[state, right=of a] (b) {2};
			\node[state, right=of b] (c) {3};
			\node[state, right=of c] (d) {4};

			% Connect the states with arrows
			\draw[every loop]
			% a -> b -> c -> d
			(a) edge[bend right, auto=right] node {} (b)
			(c) edge[bend right, auto=right] node {} (d)

			% d -> c -> b -> a
			(d) edge[bend right, auto=right] node {} (c)
			(c) edge[bend right, auto=right] node {} (b)
			(b) edge[bend right, auto=right] node {} (a)

			% a -> a, b -> b, ...
			(a) edge[loop left]			     node {} (a)
			(d) edge[loop right]			 node {} (d)
		\end{tikzpicture}
	\end{center}

	Zustände 1 und 2 sind wesentlich, 3 und 4 aber nicht.
	\\
	\only<2->{
		\vspace{5mm}
		Was heißt das für das Langzeitverhalten der Kette?
	}
	\\
	\vspace{5mm}
	\only<3->{
		Eine invariante Verteilung würde den Zuständen 3 und 4 also
		Wahrscheinlichkeit 0 zuweisen.
	}
\end{frame}
\subsection{Periodische Markowketten}

\begin{frame}
	\frametitle{Periodische Markowketten}
	Wir betrachten folgende Markowkette:
	\begin{center}
		\begin{tikzpicture}
			% Add the states
			\node[state]			 (a) {1};
			\node[state, right=of a] (b) {2};
			\node[state, right=of b] (c) {3};
			\node[state, right=of c] (d) {4};
			\node[state, right=of d] (e) {5};

			% Connect the states with arrows
			\draw[every loop]
			% a -> b -> c -> d
			(a) edge[bend right, auto=right] node {} (b)
			(b) edge[bend right, auto=right] node {} (c)
			(c) edge[bend right, auto=right] node {} (d)
			(d) edge[bend right, auto=right] node {} (e)

			% d -> c -> b -> a
			(c) edge[bend right, auto=right] node {} (a)
			(e) edge[bend right, auto=right] node {} (c)
		\end{tikzpicture}
	\end{center}
	\\
	\only<2-3>{
		Mögliche Rückkehrzeiten für Zustand 1 sind
		\[
			\{
			3, 6, 9, 12, ...
			\}
			\only<3>{
				=
				\{
				n \in \mathbb{N} \; | \; i \rightsquigarrow i[n]
				\}
			}
		\]
		$\Rightarrow$ Zustand 1 ist periodisch mit Periode 3.
	}
	\only<4>{
		Dies gilt sogar für alle Zustände.
		\vspace{5mm}

		$\Rightarrow$ Die Markowkette ist periodisch mit Periode 3.
	}
\end{frame}

\begin{frame}
	\frametitle{Definition Periode}
	Wir definieren den Begriff der Periode:
	\[
		d_i = \text{ggT} \{
		n \in \mathbb{N} \; | \; i \rightsquigarrow i[n]
		\}
	\]
	\only<1>{
		$d_i \Rightarrow$ Periode für Zustand $i$ \\
	}
	\only<2-3>{
		$d \Rightarrow$ Periode für gesamte Markowkette \\
		\vspace{5mm}
		Wenn alle Zustände die selbe Periode haben. \\
	}
	\only<3>{
		Also wenn
		\[
			\forall i \in I : d = d_i \geq 2.
		\]
	}
	\only<4>{
		Falls $d=1$, so nennen wir die Markowkette \textbf{aperiodisch} .
	}
\end{frame}

\begin{frame}[t]
	\frametitle{Kommunizierende Zustände haben die selbe Periode}
	\textit{Aussage}: $i \leftrightsquigarrow j \Rightarrow d_i = d_j$ \\
	\only<2-5>{
		\textit{Beweis:} \\
		\vspace{5mm}
		Es gelte $j \rightsquigarrow j [n]$, \\
		und seien $k, m$ Zeitpunkte, sodass
		$i \rightsquigarrow j [k]$ und $j \rightsquigarrow i [m]$. \\
		\vspace{5mm}
	}
	\only<3-5>{
		Dann gilt:
		\[
			i \rightsquigarrow i [k+m]
			\text{ und }
			i \rightsquigarrow i [k+m+n]
		\] \\
		\vspace{5mm}
	}
	\only<4-5>{
		Somit teilt $d_i$ dann $k+m$ und $k+m+n$,
		dann teilt $d_i$ auch $n$. \\
		\vspace{5mm}
	}
	\only<5>{
		Damit ist $d_i$ gemeinsamer Teiler aller $n$ mit
		$j \rightsquigarrow j [n]$. \\
		\[
			\Rightarrow d_i \leq d_j
		\]
	}
	\only<6>{
		\\
		\vspace{5mm}
		Nun folgt aus Symmetriegründen auch $d_j \leq d_i$. \\
		\vspace{5mm}
		Und damit insbesondere
		\[
			d_i = d_j.
		\]
	}
\end{frame}

\subsection{Zerlegung in Teilgruppen}
\begin{frame}
	\frametitle{Zerlegung in Teilgruppen}
	Wir betrachten wieder unsere periodische Markowkette von vorhin:
	\begin{center}
		\begin{tikzpicture}
			% Add the states
			\node[state]			 (a) {1};
			\node[state, right=of a] (b) {2};
			\node[state, right=of b] (c) {3};
			\node[state, right=of c] (d) {4};
			\node[state, right=of d] (e) {5};

			% Connect the states with arrows
			\draw[every loop]
			% a -> b -> c -> d
			(a) edge[bend right, auto=right] node {} (b)
			(b) edge[bend right, auto=right] node {} (c)
			(c) edge[bend right, auto=right] node {} (d)
			(d) edge[bend right, auto=right] node {} (e)

			% d -> c -> b -> a
			(c) edge[bend right, auto=right] node {} (a)
			(e) edge[bend right, auto=right] node {} (c)
		\end{tikzpicture}
	\end{center}
	\\
	\vspace{5mm}
	\only<2>{
		Wir definieren
		\[
			C(i) = \{
			\;
			j \in I \; | \; j \leftrightsquigarrow i
			\;
			\}
		\]
	}
	\only<3-4>{
		In unserem Fall sind das aber alle Zustände:
		\[
			\forall i \in I : C(i) = I
		\]
	}
	\only<4>{
		Jetzt sehen wir: Wenn einer dieser Zustände periodisch ist,
		so sind es die anderen direkt auch alle.
	}
	\only<5-6>{
		Wir können die Menge $C(i)$ in Teilgruppen wie folgt zerlegen:
		\[
			C_r (i) = \{
			\;
			j \in C(i) \; | \;
			r \equiv n \; (\text{mod } d_i)
			\text{ mit } i \rightsquigarrow j [n]
			\;
			\}
		\]
	}
	\only<6>{
		Also die Menge an Zuständen, die mit
		$i$ kommunizieren, welche wir von $i$ aus in
		$n = r + k \cdot d_i$ Schritten erreichen können (für
		alle $k \in \mathbb{N}$).
	}
	\only<7>{
		Für unsere Markowkette heißt das:
		\[
			C_0 (1) = \{
			1, 4
			\},
			C_1 (1) = \{
			2, 5
			\},
			C_2 (1) = \{
			3
			\}
		\]
	}
\end{frame}

\section{Rekurrenz und Transienz}
\subsection{Definition Rekurrenz}
\begin{frame}
	\frametitle{Rekurrenz}
	\only<1-2>{
		\begin{center}
			Rekurrenz beschreibt das Rückkehrverhalten einer Markowkette.
		\end{center}
	}
	\only<2>{
		\vspace{5mm}
		Wie oft 'besucht' eine Markowkette einen bestimmten Zustand $i$?
	}
	\only<3-5>{
		Definiere folgende zwei Begriffe für Zustände:
		\begin{itemize}
			\only<4-5>{
			\item 'rekurrent' heißt, dass eine Markowkette einen Zustand unendlich oft besucht.
			      }
			      \only<5>{
			\item 'transient' meint genau das Gegenteil.
			      }
		\end{itemize}
	}
\end{frame}

\begin{frame}[t]
	\frametitle{Kriterium für Rekurrenz}
	\only<1-6>{
		Wir setzen für $n \in \mathbb{N}$:
		\[
			f_{ij} ^{(n)} = P_i \left(
			X_n = j, X_{n-1} \neq j, ..., X_1 \neq j
			\right)
		\]
	}
	\only<2>{
		Die Wahrscheinlichkeit bei Start in $i$ zum ersten mal nach $n$ Schritten den Zustand $j$
		zu besuchen.
	}
	\only<3-6>{
		Wir erkennen $f_{ij} ^{(0)} = 0$ uns setzen weiter:
		\[
			f_{ij} ^{*} = \sum_{n=1}^{\infty} f_{ij} ^{(n)}
			\;\;\;
			\text{ und }
			\;\;\;
			p_{ij} ^{*} = \sum_{n=1}^{\infty} p_{ij} ^{(n)}
		\]
	}
	\only<4-5>{
		\begin{itemize}
			\only<4-5>{
			\item $f_{ij} ^{*}$ ist die Wahrscheinlichkeit je von $i$ nach $j$ zu gelangen.
			      }
			      \only<5>{
			\item $p_{ij} ^{*}$ ist die erwartete Anzahl an Besuchen in $j$ bei Start in $i$.
			      }
		\end{itemize}
	}
	\only<6>{
		Es gilt nämlich:
		\begin{align*}
			p_{ij} ^{*} & = \sum_{n=1}^{\infty} p_{ij} ^{(n)}
			= \sum_{n=1}^{\infty} E_i \left(
			1_{\{
				X_n = j
				\} }
			\right)
			= E_i \left(
			\sum_{n=1}^{\infty} 1_{\{
				X_n = j
				\} }
			\right)                                           \\
			            & = E_i \left(
			\text{Anzahl der Besuche in $j$}
			\right)
		\end{align*}
	}
\end{frame}

\begin{frame}[t]
	\frametitle{Kriterium für Rekurrenz}
	Wir nennen eine ZV
	\[
		\tau : \Omega \to \mathbb{N}_0
	\]
	\only<2->{
		\textbf{Stoppzeit}
		, wenn für alle $n \geq 0$
		das Ereignis
		\[
			\{
			\omega : \tau (\omega) = n
			\}
		\]
		nur von $X_0, ..., X_n$ abhängt.
	}
	\only<3->{
		\vspace{5mm}

		Dies bedeutet für ein geeignetes $A \subset I^{n+1}$:
		\[
			\{
			\tau = n
			\} = \{
			\left(
			X_0, ..., X_n
			\right) \in A
			\}
		\]
	}
	\only<4->{
		Im Folgenden schreiben wir $B_i$ für die Anzahl der Besuche in $i$.
	}
\end{frame}

\begin{frame}[t]
	\frametitle{Kriterium für Rekurrenz}
	\textit{Aussage:} $P_i (B_i \geq m) = \left(
		f_{ii} ^{*}
		\right) ^{m}, m \in \mathbb{N}$

	\textit{Beweis:}
	\vspace{5mm}

	\only<2-3>{
		Seien:
		\begin{align*}
			\tau_1 (\omega)     & = \inf \{
			n \in \mathbb{N} : X_n (\omega) = i
			\}                              \\
			\tau_{m+1} (\omega) & = \inf \{
			n > \tau_m (\omega) : X_n (\omega) = i
			\}
		\end{align*}
	}
	\only<3>{
		Es ist $\tau_m (\omega)$ der Zeitpunkt des m-ten Besuches in $i$.
		\\
		\vspace{5mm}
		Und wenn dieser nicht existiert, dann ist $\tau_m (\omega) = \infty$.
	}
	\only<4>{
		\textit{Bemerkung:} Die $\tau_m$ sind Stoppzeiten.
		\\
		\vspace{5mm}
		Setzen wir nämlich $A_{mn}$ als die Menge der Folgen von Realisationen
		$\left(
			j_0, ..., j_{n-1}
			\right) \in I ^{n} $ mit $j_0 = i$, \\
		welche $i$ noch $m-1$ weitere Male
		besucht haben. \\
		\vspace{5mm}
		Dann ist
		\[
			\{
			X_0 = i, \tau_m = n
			\} = \{
			\left(
			X_0, ..., X_{n-1}
			\right) \in A_{mn}, X_n = i
			\}.
		\]
	}
	\only<5-6>{
		Jetzt sehen wir
		\[
			\{
			\tau_m < \infty
			\} = \{
			B_i \geq m
			\}.
		\]
		Die Behauptung beweisen wir nun nach Induktion: \\
		\vspace{5mm}
		Für $m = 1$ gilt
		\[
			P_i (\tau_m < \infty) = \left(
			f_{ii} ^{*}
			\right) ^{m}.
		\]
		\\
		Die Wahrscheinlichkeit \textbf{einmal} zu $i$ zurückzukehren.
	}
	\only<6>{
	\\
	\vspace{5mm}
	Wir definieren
	\[
		D_n ^{n+k} = \{
		X_{n+1} \neq i, ..., X_{n+k-1} \neq i, X_{n+k} = i
		\}
	\]
	}
	\only<7>{
		\vspace{5mm}
		Nun machen wir den Induktionsschritt:
		\[
			P_i(\tau_m < \infty) = \left(
			f_{ii} ^{*}
			\right) ^{m} \Rightarrow
			P_i \left(
			\tau_{m+1} < \infty
			\right) = \left(
			f_{ii} ^{*}
			\right) ^{m+1}
		\]
	}
	\only<8->{
		\baselineskip
		\begin{align*}
			\only<8-9>{
				P_i \left(
				\tau_{m+1} < \infty
			\right) & = \sum_{k=1}^{\infty} \sum_{n=1}^{\infty} P_i \left(
				\tau_{m+1} - \tau_m = k, \tau_m = n
			\right)                                                        \\
			}
			\only<9-10>{
			        & = \sum_{k=1}^{\infty} \sum_{n=1}^{\infty} P_i \left(
				\tau_{m+1} - \tau_m | \tau_m = n
				\right)
				P_i \left(
				\tau_m = n
				\right)
			\\
			}
			\only<10-11>{
			        & = \sum_{k=1}^{\infty} \sum_{n=1}^{\infty} P_i \left(
				D_n ^{n+k} | X_n = i, \left(
				X_0, ..., X_{n-1}
				\right) \in A_{mn}
				\right)
				P_i (\tau_m = n)
			\\
			}
			\only<11-12>{
			        & = \sum_{k=1}^{\infty} \sum_{n=1}^{\infty} P_i \left(
				D_n ^{n+k} | X_n = i
				\right) P_i (\tau_m = n)
			\\
			}
			\only<12->{
			        & = \sum_{k=1}^{\infty} \sum_{n=1}^{\infty} P_i \left(
				D_0 ^{k} | X_0 = i
				\right) P_i (\tau_m = n)
			\\
			}
			\only<13->{
			        & = \sum_{k=1}^{\infty} f_{ii} ^{(k)}
			\underbrace{ P_i (\tau_m < \infty) }_{ \left(
			f_{ii} ^{*}
			\right) ^{m} }
			= \left(
			f_{ii} ^{*}
			\right) ^{m+1}
			}
		\end{align*}
		\only<13->{
			\qquad \qquad \qed
		}
	}
\end{frame}

\begin{frame}[t]
	\frametitle{Kriterium für Rekurrenz - Rekurrenzsatz}
	Wir schauen uns noch einmal die Gleichung von eben an:
	\[
		P_i \left(
		B_i \geq m
		\right) = \left(
		f_{ii} ^{*}
		\right) ^{m}
	\]
	\only<2->{
		Wir überlegen uns, dass ein Zustand rekurrent ist genau dann wenn
		\[
			f_{ii} ^{*} = 1,
		\]
		denn nun klappt die obige Gleichung für alle $m \in \mathbb{N}$. \\
	}
	\only<3->{
		\vspace{5mm}
		Dies ist aber äquivalent zu
		\[
			p_{ii} ^{*} = \infty.
		\]
		\\
		Dass wir also bei Start in $i$, den Zustand $i$ unendlich oft wieder besuchen.
	}
\end{frame}
\subsection{Beispiel mehrdimensionale Irrfahrt}
\begin{frame}[t]
	\frametitle{Mehrdimensionale Irrfahrt}
	\only<-3>{
		Sei
		\[
			Y_n = \left(
			Y_{n_1}, Y_{n_2}, ..., Y_{nd}
			\right)
		\]
		eine Folge von unabhängigen $d$-dimensionalen ZV in $\mathbb{Z}$. \\
	}
	\only<2-3>{
		\vspace{5mm}
		Wir fragen uns ob der Zustand
		\[
			\left(
			0, 0, ..., 0
			\right) \in \mathbb{Z}^{d}
		\]
		rekurrent ist oder transient. \\
	}
	\only<3-5>{
		\vspace{5mm}
		Wir betrachten
		\[
			p_{(0,...,0), (0,...,0)} ^{(2n)} = \left(
			\begin{pmatrix} 2n \\ n \\ \end{pmatrix} 2 ^{-2n}
			\right) ^{d}
			\sim \left(
			\frac{ 1 }{ \sqrt{\pi n}  }
			\right) ^{d}
		\]
	}
	\only<4->{
		die Wahrscheinlichkeit nach $2n$ Schritten vom Ursprung zum Ursprung
		zurückzukehren. \\
	}
	\only<5->{
		\vspace{5mm}
		Die Markowkette ist für
		\begin{align*}
			d \leq 2 & \text{ rekurrent und } \\
			d \geq 3 & \text{ transient. }
		\end{align*}
	}
\end{frame}

\begin{frame}[t]
	\frametitle{Warum?}
	Mit dem Rekurrenzsatz prüfen wir wann $p_{ii} ^{*} < \infty$
	\begin{align*}
		p_{ii} ^{*} & = \sum_{n=1}^{\infty} p_{ii} ^{(n)}                          \\
		            & = \frac{ 1 }{ (\sqrt{\pi}) ^{d} } \sum_{n=1}^{\infty} \left(
		\frac{ 1 }{ n }
		\right) ^{\frac{ d }{ 2 }}
	\end{align*}
	Und diese Summe divergiert nur für $d \leq 2$.
\end{frame}

\end{document}
