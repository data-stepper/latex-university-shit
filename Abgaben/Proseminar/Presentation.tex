\documentclass[]{beamer}
%Information to be included in the title page:
\title{Langzeitverhalten von Markowketten}
\author{Nogarbek Sharykpayev, Bent Müller}
\institute{Universität Hamburg}
\date{15.06.2021}

\setbeamertemplate{headline}[default]
\setbeamertemplate{navigation symbols}{}
\mode<beamer>{\setbeamertemplate{blocks}[rounded][shadow=true]}
\setbeamercovered{transparent}
\setbeamercolor{block body example}{fg=blue, bg=black!20}

\useoutertheme[subsection=false]{miniframes}
\usetheme{default}

\begin{document}

\frame{\titlepage}
	\begin{frame}
		\frametitle{Struktur der Präsentation}
		\tableofcontents
	\end{frame}

	\section{Konvergenz der Übergangsmatrix}
	\subsection{Invariante Wahrscheinlichkeitsverteilung}
	\subsection{Beweis der Konvergenz}

	\begin{frame}
		\frametitle{Convergence}
		Yo
	\end{frame}
	\begin{frame}
		Die konvergiert
	\end{frame}

	\section{Periodizität}
	\subsection{Kommunizierende Zustände}
	\begin{frame}
		\frametitle{Notation}
		$i, j \in I$ seien Zustände
		\[
			i \rightsquigarrow j
		\]
		heißt, dass wir von $i$ nach $j$ gelangen können.

		Wir sagen $i$ kommuniziert mit $j$.
		Des weiteren gilt:
		\[
		i \leftrightsquigarrow j \Leftrightarrow i \rightsquigarrow j
		\text{ und } j \rightsquigarrow i
		\] 
	\end{frame}
	
	\subsection{Periodische Markowketten}
	\subsection{Zerlegung in Teilgruppen}
	\begin{frame}
		\frametitle{Kommunizierende Zustände}
		Wir benutzen Folgende Notation:
	\end{frame}

	\section{Rekurrenz und Transienz}
	\subsection{Definition Rekurrenz}
	\subsection{Beispiel mehrdimensionale Irrfahrt}
	\begin{frame}
		\frametitle{Rekurrente Ketten}
		
	\end{frame}
	
\end{document}
