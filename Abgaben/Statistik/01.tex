\documentclass[12pt]{article}
\usepackage{setspace}
\usepackage{amssymb}
\usepackage{amsmath}
\title{Mathematische Statistik \\ Sommersemester 2021 \\ Abgabe Blatt 1 \\}
\author{Bent Müller (7302332) \\ Steven Zinger (7309728) \\ Ferdinand Cramm (7318883)}
\doublespace
\newcommand{\notab}{\hspace*{-4em}}
\newcommand{\tab}{\hspace*{-1em}}
\newcommand{\dtab}{\hspace*{2em}}
\newcommand{\ttab}{\hspace*{5em}}
\newcommand{\n}{$\:$\\}
\date{April 16, 2021}
\setcounter{page}{1}
\begin{document}
\maketitle
\textbf{Aufgabe 1} (a) \\\\Aufgabe war es die Dichte der F-Verteilung zu berechnen, welche wie folgt definiert ist:
\[
	X \sim \chi^2_m, Y \sim \chi^2_n, F = \frac{nX}{mY}
\] 
Und wir erinnern uns, dass eine Zufallsvariable $A \sim \chi^2_k, k \in \mathbb{N}$ die Dichte
\[
	f_A(y) = y^{\frac{n}{2} - 1} e^{-\frac{y}{2}} \cdot
	\frac{1}{
		\Gamma(\frac{n}{2}) 2^{\frac{n}{2}}
	}
\]
besitzt.
Nun berechnen wir die Dichte von $a \cdot A, a \in \mathbb{R}_{>0}$ per Dichtetransformation wie folgt:

Wir setzen:
$
	g(x) = a \cdot x,
	g'(x) = a,
	g ^{-1} (y) = \frac{y}{a},
	|g' (g ^{-1} (y))| = |a| = a
$

\begin{align*}
	f_{a\cdot A}(y) &= \frac{ f_A ( g ^{-1} (y)) }{ | g' ( g ^{-1} (y) ) | }
	= \frac{ 1 }{ a } \left( \frac{ y }{ a } \right) ^{ \frac{ k }{ 2 } - 1 }
	e ^{- \frac{ y }{ 2a }} \frac{ 1 }{ \Gamma(\frac{ k }{ 2 }) 2 ^{\frac{ k }{ 2 }}}\\
					&= \frac{y ^{ \frac{ k }{ 2 } -1} e ^{ - \frac{ y }{ 2a }}}{
						a ^{ \frac{ k }{ 2 } } \Gamma(\frac{ k }{ 2 }) 2 ^{ \frac{ k }{ 2 } }
					}
\end{align*}

Also wissen wir die Dichten von $nX$ und $mY$:
\[
	f_{nX}(y) = \frac{ y ^{ \frac{ m }{ 2 } -1} e ^{ - \frac{ y }{ 2n }} }{ 
		n ^{ \frac{ m }{ 2 } } \Gamma(\frac{ m }{ 2}) 2 ^{\frac{ m }{ 2 }}
	},\;\;\;\;\;\;
	f_{mY}(y) = \frac{ y ^{ \frac{ n }{ 2 } -1} e ^{ - \frac{ y }{ 2m }} }{ 
		m ^{ \frac{ n }{ 2 } } \Gamma(\frac{ n }{ 2}) 2 ^{\frac{ n }{ 2 }}
	}
\] 

Jetzt berechnen wir die gemeinsame Dichte $f(x, y)$ von $(nX, mY)$

\begin{align*}
	f(x, y) &= f_{nX}(x) \cdot f_{mY}(y) \;\;\;\;\; \textit{(Produktdichte)} \\
			&= \frac{ 
				\left( x ^{\frac{ m }{ 2 }-1} e ^{- \frac{ x }{ 2n }} \right) 
				\left( y ^{\frac{ n }{ 2 }-1} e ^{- \frac{ y }{ 2m }} \right) 
				}{  
				\left( n ^{\frac{ m }{ 2 }} \Gamma(\frac{ m }{ 2 }) 2 ^{\frac{ m }{ 2 }} \right) 
				\left( m ^{\frac{ n }{ 2 }} \Gamma(\frac{ n }{ 2 }) 2 ^{\frac{ n }{ 2 }} \right)
			}
			= \frac{ 
				x ^{\frac{ m }{ 2 } - 1} y ^{\frac{ n }{ 2 }-1} e ^ {-\frac{ x }{ 2n } -\frac{ y }{ 2m }}
				}{  
				\Gamma(\frac{ m }{ 2 }) \Gamma(\frac{ n }{ 2 })
				2 ^{\frac{ m+n }{ 2 }} 
				n ^{\frac{ m }{ 2 }} m ^{\frac{ n }{ 2 }}
			}
\end{align*}

Nach Satz A.2 berechnen wir endlich die Dichte von $\frac{nX}{mY}$

\begin{align*}
	f_{ \frac{nX}{mY} }(z) &= \int_{0}^{\infty} f(z \cdot t, t) | t | \;\; dt
	= \int_{0}^{\infty} 
			 \frac{ 
				\left( zt \right)  ^{\frac{ m }{ 2 } - 1} t ^{\frac{ n }{ 2 }-1} e ^ {-\frac{ zt }{ 2n } -\frac{ t }{ 2m }}
				}{  
				\Gamma(\frac{ m }{ 2 }) \Gamma(\frac{ n }{ 2 })
				2 ^{\frac{ m+n }{ 2 }} 
				n ^{\frac{ m }{ 2 }} m ^{\frac{ n }{ 2 }}
			} \; |t| \;\; dt \\
	&= \frac{ z ^{\frac{ m }{ 2 }-1} }{ 
		\Gamma(\frac{ m }{ 2 }) \Gamma(\frac{ n }{ 2 })
		2 ^{\frac{ m+n }{ 2 }} 
		n ^{\frac{ m }{ 2 }} m ^{\frac{ n }{ 2 }}
	}
	\int_{0}^{\infty} t ^{ \frac{m+n}{2} - 1} e ^{ - \frac{ zt }{ 2n } - \frac{ t }{ 2m } } \;\; dt \\
\end{align*}

Substituiere wie folgt:
\begin{align*}
	u &= \frac{ zt }{ 2n } + \frac{ t }{ 2m } = t\left( \frac{ z }{ 2n } + \frac{ 1 }{ 2m } \right) ,
	t = \frac{ u }{ \left( \frac{ z }{ 2n } + \frac{ 1 }{ 2m } \right)},\\
	du &=   \left(
		\frac{ z }{ 2n } + \frac{ 1 }{ 2m }
	\right)  dt, \;
	dt = \frac{ du }{ \left( \frac{ z }{ 2n } + \frac{ 1 }{ 2m } \right) }
\end{align*}

Wohlgemerkt ändern sich hier keine Integralgrenzen, da die substituierte Funktion an den
betroffenen Stellen die selben Werte annimmt wie die vorherige Funktion.
Somit erhalten wir folglich:

\begin{align*}
	\int_{0}^{\infty} t ^{ \frac{m+n}{2} - 1} e ^{ - \frac{ zt }{ 2n } - \frac{ t }{ 2m } } \;\; dt
	&= \int_{0}^{\infty} \left( 
		\frac{ u }{ 
			\frac{ z }{ 2n } + \frac{ 1 }{ 2m }
		}
		\right) ^{\frac{m+n}{2} -1} e ^{-u} \;\; \frac{ du }{ 
		\left(  
		\frac{ z }{ 2n } + \frac{ 1 }{ 2m }
		\right) 
	}\\
	&= \left( 
		\frac{ 1 }{ 
			\frac{ z }{ 2n } + \frac{ 1 }{ 2m }
		}
		\right) ^{
		\frac{ m+n }{2}
	}
	\int_{0}^{\infty}  u ^{ \frac{ m+n }{ 2 } -1} e^{ -u }\;\; du\\
	&= \left( 
		\frac{ 1 }{ 
			\frac{ z }{ 2n } + \frac{ 1 }{ 2m }
		}
		\right) ^{
		\frac{ m+n }{2}
	} \cdot \Gamma \left( 
	\frac{ m+n }{ 2 }
\right) 
\end{align*}

Wenn wir das jetzt in die vorherige Gleichung einsetzen, bekommen wir:

\begin{align*}
	f_{ \frac{nX}{mY} }(z) &= \frac{
		\Gamma \left( 
	\frac{ m+n }{ 2 }
\right) z ^{\frac{ m }{ 2 }-1} }{ 
				\Gamma(\frac{ m }{ 2 }) \Gamma(\frac{ n }{ 2 })
						2 ^{\frac{ m+n }{ 2 }} 
								n ^{\frac{ m }{ 2 }} m ^{\frac{ n }{ 2 }}
									}
									\left( 
												\frac{ 1 }{ 
																\frac{ z }{ 2n } + \frac{ 1 }{ 2m }
																		}
																				\right) ^{
																						\frac{ m+n }{2}
																							} \\
	&= \frac{
		\Gamma \left( 
	\frac{ m+n }{ 2 }
\right) z ^{\frac{ m }{ 2 }-1} }{ 
				\Gamma(\frac{ m }{ 2 }) \Gamma(\frac{ n }{ 2 })
								n ^{\frac{ m }{ 2 }} m ^{\frac{ n }{ 2 }}
									}
								\left( 
									\frac{ 1 }{ 
								\frac{ z }{ n } + \frac{ 1 }{ m }
							}
						\right) ^{
					\frac{ m+n }{2}
				} \Bigg \vert \cdot \left( \frac{ \frac{ m }{ n } }{ \frac{ m }{ n } } \right) ^{\frac{ m }{ 2 }}\\
	&= \frac{ z ^{ \frac{ m }{ 2 } -1} \left( 
			\frac{ m }{ n }
			\right) ^{ \frac{ m }{ 2 } }  }{ B \left( 
	\frac{ m }{ 2 }, \frac{ n }{ 2 }\right) \left( 
	1+ \frac{ m }{ n }z
	\right) ^{ \frac{ m+n }{ 2 } } } \\
\end{align*}

	\begin{description}
			\item{\textbf{Aufgabe 1} (b)\\\\ Seien $X \thicksim \mathcal{N}(0,1), \quad Y \thicksim \chi_n^2$ \;unabhängig und $T = \frac{\sqrt{n}X}{\sqrt{Y}}.$
			}
			\item{\textit{Behauptung}\\ $T$ hat die Dichte $$f(t) = \frac{\Gamma(\frac{n+1}{2})}{\Gamma(\frac{n}{2})\sqrt{n\pi}} \,\Big(1 + \frac{t^2}{n}\Big)^{-\frac{n+1}{2}},\qquad t \in \R.$$
			}
			\item{\textit{Beweis}\\ Sei $A := \sqrt{n}\cdot X$. \quad $f_X (x) = \frac{1}{\sqrt{2\pi}} e^{-\frac{x^2}{2}}$.\\Wende die Dichtetransformation (Satz 10.10, Math. Stoch.) für $A$ an: $$h: \R \longrightarrow \R,\quad h(x) = \sqrt{n}\cdot x,\quad h^{-1} (x) = \frac{x}{\sqrt{n}}, \quad (h^{-1}(x))^{'} = \frac{1}{\sqrt{n}} \quad (n\in \N)$$ 
			\begin{equation*}\begin{split}\Longrightarrow\qquad f_A(a) &= f_X(h^{-1}(a)) \;\Big|\det D(h^{-1}) (a)  \Big| \;I_\R(a) 
			\\&= \frac{1}{\sqrt{2\pi}} \exp\Big(-\frac{(\frac{a}{\sqrt{n}})^2}{2}\Big)\; \Big|\frac{1}{\sqrt{n}}  \Big| 
			\\&= \frac{1}{\sqrt{2\pi n}} \exp\Big(-\frac{(\frac{a}{\sqrt{n}})^2}{2}\Big).
			\end{split}\end{equation*}\\\\
			Sei $B := \sqrt{Y}$. \quad $f_Y(y) = \frac{1}{2^{n/2} \Gamma(\frac{n}{2})} \;e^{-\frac{y}{2}} \;y^{\frac{n}{2}-1} \; I_{(0,\infty)} (y)$.
			\\Wende die Dichtetransformation (Satz 10.10, Math. Stoch.) für $B$ an: $$g: \R_{\geq 0} \longrightarrow \R_{\geq 0},\quad g(x) = \sqrt{x},\quad g^{-1} (x) = x^2, \quad (g^{-1}(x))^{'} = 2x$$ 
			\begin{equation*}\begin{split}\Longrightarrow\qquad f_B(b) &= f_Y(g^{-1}(b)) \;\Big|\det D(g^{-1}) (b)  \Big| \;I_{\R_{\geq 0}}(b) 
			\\&= \frac{1}{2^{n/2} \Gamma(\frac{n}{2})} e^{-\frac{b^2}{2}} \,b^{2^{\frac{n}{2}-1}} \; \big|2b\big| \;I_{(0,\infty)}(b)
			\\&= \frac{1}{2^{n/2} \Gamma(\frac{n}{2})} e^{-\frac{b^2}{2}} \,b^{n-2} \; \big|2b\big| \;I_{(0,\infty)}(b)
			\\&= \frac{2}{2^{n/2} \Gamma(\frac{n}{2})} e^{-\frac{b^2}{2}} \,b^{n-1} \;  \;I_{(0,\infty)}(b).\\\\ 
			\end{split}\end{equation*} 
					\begin{equation*}\begin{split}
							f_{A,B} (a,b) = f_A(a) \cdot f_B(b) &= \frac{1}{\sqrt{2\pi n}} \exp\Big(-\frac{(\frac{a}{\sqrt{n}})^2}{2}\Big) \cdot \frac{2}{2^{n/2} \Gamma(\frac{n}{2})} \exp\Big(-\frac{b^2}{2}\Big) \,b^{n-1} \;  \;I_{(0,\infty)}(b)
							\\&= \frac{2}{\sqrt{2\pi n}\;2^{n/2}\; \Gamma(\frac{n}{2})} \cdot \exp\Big(-\frac{a^2}{2n}-\frac{b^2}{2}\Big) \cdot b^{n-1} \;I_{(0,\infty)}(b)
					\end{split}\end{equation*} \\ Die Dichte von $T = \frac{\sqrt{n}X}{\sqrt{Y}}$ beträgt nach Satz 10.11 (c) (Math. Stoch.):
			\begin{equation*}\begin{split}
				k_T(z) &= \int_{\mathbb{R}} \; f_{A,B} (za,a)\;|a|\; da 
					 \\&= \int_{\mathbb{R}} \frac{2}{\sqrt{2\pi n} \; 2^{n/2}\; \Gamma(\frac{n}{2})} \; \exp\Big(-\frac{(za)^2}{2n} -\frac{a^2}{2}  \Big) \; a^{n-1} |a| \; I_{(0,\infty)}\; da 
					\\&= \frac{2}{\sqrt{2\pi n} \; 2^{n/2}\; \Gamma(\frac{n}{2})} \; \int_0^\infty \; \exp\Big(a^2\big(-\frac{z^2}{2n} -\frac{1}{2}\big) \Big)\; a^n \; da
			\end{split}\end{equation*}
			}
	\end{description}

Jetzt substituieren wir folgendermaßen:

\begin{align*}
	& u = a ^{2} \left(
		\frac{z^2}{2n} + \frac{ 1 }{ 2 }
	\right) \;,
	a = \sqrt{ \frac{ u }{  \frac{z^2}{2n} + \frac{ 1 }{ 2 }} }\; ,
	du = 2a \left(
		\frac{z^2}{2n} + \frac{ 1 }{ 2 }
	\right) da \\
	&= \left(
		\frac{ z^2 }{ n } + 1
		\right) \sqrt{
		\frac{ u }{ \frac{ z^2 }{ 2n } + \frac{ 1 }{ 2 } }
	} \;\; da
	\Leftrightarrow da = \frac{ du }{ \left(
		\frac{ z^2 }{ n } + 1
		\right) \sqrt{
		\frac{ u }{ \frac{ z^2 }{ 2n } + \frac{ 1 }{ 2 } }
	} } \text{ und erhalten } \\
	k_T (z) &= \frac{ 2 }{ 
		\sqrt{2 \pi n}\;  2 ^{\frac{ n }{ 2 }} \Gamma( \frac{ n }{ 2 } )
	}
	\int_{0}^{\infty} \left(
		\sqrt{ \frac{ u }{ \frac{ z^2 }{ 2n } + \frac{ 1 }{ 2 } } } 
	\right) ^{n}
	e ^{-u}
	\frac{ du }{ \left(
		\frac{ z^2 }{ n } + 1
		\right) \sqrt{\frac{ u }{ 
			\frac{ z^2 }{ 2n } + \frac{ 1 }{ 2 }
	}} 
} \\
	&= \frac{ \sqrt{
			\frac{ z^2 }{ n } + 1
		}  }{ 
		\sqrt{\pi n}\;  2 ^{\frac{ n }{ 2 }} \Gamma( \frac{ n }{ 2 } )
		\left(
			\frac{ z^2 }{ n } + 1
		\right) ^{\frac{ n }{ 2 } + 1}
	}
	\int_{0}^{\infty} u ^{\frac{n-1}{2}} e ^{-u} du \\
	&= \frac{ 
		\Gamma(\frac{n+1}{2})
	}{ \sqrt{\pi n} \;\;  \Gamma(\frac{ n }{ 2 })}
	\left(
		\frac{ z^2 }{ n } + 1
	\right) ^{ - 
		\frac{n+1}{2}
	} \qquad = f(z) \text{  (aus der Aufgabe)} \qed
\end{align*}

 \begin{description}
                \item{\textbf{Aufgabe 2a}\\
                Betrachte die Dichten der $N(\vartheta,\vartheta^2)$-Verteilung mit $\vartheta\in\R\,\setminus \{0\}$:
                \begin{align*}
                    f_{\vt}(x)=\frac{1}{\sqrt{2\pi\vt^2}}\exp\left(-\frac{(x-\vt)^2}{2\vt^2}\right) , \hspace{1mm} x\in\R
                \end{align*}
                }   
                \item{\textit{Behauptung}\\
                Die Familie von Dichten $\{f_{\vt}|\vt\in\Theta=\R\setminus\{0\}\}$ bildet eine $2$-parametrige Exponentielle Familie mit den Statistiken:
                \begin{align*}
                    T_1(x)=-x^2 \\
                    T_2(x)=2x 
                \end{align*}
                und natürlichem Parameterraum $Z^*=\R_{+}\times\R$
                }
                \item{\textit{Beweis}\\
            Wir setzen:
            \begin{align*}
                D(\vt)=\log\left(\frac{1}{\sqrt{2\pi\vt^2}}\right) \Rightarrow c(\vt)=\frac{1}{\sqrt{2\pi\vt^2}}
            \end{align*}
                Betrachten wir den Exponenten:
                \begin{align*}
                    -\frac{(x-\vt)^2}{2\vt^2}=-\frac{1}{2\vt}(x^2-2\vt x +\vt^2)=-\frac{1}{2\vt^2}x^2+\frac{1}{\vt}\cdot x-\frac{1}{2}
                \end{align*}
                Wir können setzen:
                \begin{align*}
                    Q_1(\vt)=\frac{1}{2\vt^2}  \hspace{4mm} Q_2(\vt)=\frac{1}{\vt} \\
                    T_1(x)=-x^2  \hspace{4mm} T_2(x)=x  \hspace{4mm} S(x)=-\frac{1}{2}
                \end{align*}
                Und erhalten dann die Darstellung:
                \begin{align*}
                    f_{\vt}(x)=\exp\left(Q_1(\vt)T_1(\vt)+Q_2(\vt)T_2(\vt)+D(\vt)+S(x)\right) \\
                    =\exp(D(\vt))\cdot \exp\left(Q_1(\vt)T_1(\vt)+Q_2(\vt)T_2(\vt)+S(x)\right) \\
                    =\frac{1}{\sqrt{2\pi\vt^2}}\exp\left(-\frac{x^2}{2\vt^2}+\frac{x}{\vt}-\frac{1}{2}\right) 
                    = \frac{1}{\sqrt{2\pi\vt^2}}\exp\left(-\frac{(x-\vt)^2}{2\vt^2}\right)
                \end{align*}
                Das Bild von $Q(\vt)=(Q_1(\vt),Q_2(\vt))$ ist $Q(\Theta)=(Q_1(\Theta),Q_2(\Theta))=(\R_{+}\times\R)\setminus\{(0,0)\}=Z$
                \\Setze $A=\text{supp}(f_{\vt})=\R$. Wir wollen jetzt den natürlichen Parameteraum bestimmen, dafür betrachte:
                \begin{align*}
                    \frac{1}{c(\xi)}=\int_{A}\exp(S(x))\exp(\xi_1T_1(x)+\xi_2T_2(x))dx \\
                    =\exp\left(-\frac{1}{2}\right)\int_{-\infty}^{\infty}\exp(-\xi_1x^2+\xi_2 x)dx \\
                    =\exp\left(-\frac{1}{2}\right)\cdot \left(\left.\frac{1}{-2\xi_1x+\xi_2}\cdot\exp(-\xi_1x^2+\xi_2 x)\right|_{-\infty}^{\infty}\right) 
                \end{align*}
                Wir wollen jetzt also die 2 verschiedenen Grenzwerte $x\to\infty$ und $x\to-\infty$ ausrechnen:
                \begin{align*}
                    \lim_{x\to\infty}\frac{1}{-2\xi_1x+\xi_2}\cdot\exp(-\xi_1x^2+\xi_2 x) \\
                    \lim_{x\to\infty}\frac{1}{-2\xi_1x+\xi_2}\cdot \lim_{x\to\infty}\exp(-\xi_1x^2+\xi_2x)
                \end{align*}
                Der erste Faktor konvergiert für $x\to\infty$ gegen $\frac{1}{\xi_2}$, weil $\frac{1}{x}\xrightarrow{x\to\infty}0$
                \\ Also haben wir:
                \begin{align*}
                    \lim_{x\to\infty}\frac{1}{-2\xi_1x+\xi_2}\cdot \lim_{x\to\infty}\exp(-\xi_1x^2+\xi_2x) \\
                    =\frac{1}{\xi_2}\lim_{x\to\infty}\exp(-\xi_1x^2) \cdot \lim_{x\to\infty}\exp(\xi_2x)
                \end{align*}
                Wir bemerken , dass $\exp(-\xi_1x^2)\to 0$ für $x\to \infty$ und $\exp(\xi_2x)\to \infty$ für $x\to\infty$. Da aber $x^2$ schneller wächst als $x$ ist das Produkt der Grenzwerte $0$ (also geht $\exp(-\xi_1x^2)$ schneller gegen $0$ als $\exp(\xi_2x)$ gegen $\infty$ geht)
                \\ Betrachten wir jetzt den Fall $x\to-\infty$:
                \begin{align*}
                    \lim_{x\to-\infty}\frac{1}{-2\xi_1x+\xi_2}\cdot\exp(-\xi_1x^2+\xi_2 x)
                \end{align*}
                Beachte , dass wir auch $x\to\infty$ schreiben können , wenn wir jedes $x$ mit $-1$ multiplizieren,also:
                \begin{align*}
                    \lim_{x\to-\infty}\frac{1}{-2\xi_1x+\xi_2}\cdot\exp(-\xi_1x^2+\xi_2 x) \\
                    =\lim_{x\to\infty}\frac{1}{2\xi_1 x+\xi_2}\cdot \lim_{x\to\infty}\exp(-\xi_1x^2-\xi_2x)
                \end{align*}
                Der erste Faktor geht, mit dem selben Argument wie davor, gegen $0$, also bleibt:
                \begin{align*}
                    \lim_{x\to\infty}\exp(-\xi_1x^2)\cdot\lim_{x\to\infty}\exp(-\xi_2x)
                \end{align*}
                beide Exponenten haben aber ein negatives Vorzeichen, weshalb auch hier jeder der Faktoren gegen $0$ konvergiert.
                \\ Wir bemerken, dass die Konvergenz nur gewährleistet ist wenn $\xi_1>0$, weil sich sonst das Vorzeichen von $\exp(-\xi_1 x^2)$ ändern würde (was zur Divergenz führen würde).
                \\ Alles in allem erhalten wir also:
                \begin{align*}
                    \frac{1}{c(\xi)}=\exp\left(-\frac{1}{2}\right)\int_{-\infty}^{\infty}\exp(-\xi_1x^2+\xi_2 x)dx<\infty
                    \Leftrightarrow \xi_1>0 \hspace{2mm} \xi_2\in\R
                \end{align*}
                Damit ist unser natürlicher Parameterraum:
                \begin{align*}
                    Z^*=\{\xi=(\xi_1,\xi_2)\in\R^2|\frac{1}{c(\xi)}<\infty\}=\R_{+}\times\R\supset Z
                \end{align*}
                \xtab \qed\\\\\\
                }
        \end{description}
        \begin{description}
            \item{\textbf{Aufgabe 2b}\\
            Betrachte die Dichtenfunktionen für $\vt>0$:
            \begin{align*}
                f_{\vt}(x)=I_{(0,\vt)}(x)\exp(-2\log(\vt)+log(2x)),\hspace{2mm} x\in\R 
            \end{align*}
            }
            \item{\textit{Behauptung}\quad\\
            Die Familie $\{f_{\vt}|\vt\in\Theta=(0,\infty)\}$ bildet keine exponentielle Familie.
            }
            \item{\textit{Beweis}\quad\\
            Der Träger von $f_{\vt}$ ist gegeben durch:
            \begin{align*}
                A=\text{supp}(f_{\vt})=\{x\in\R|f_{\vt}(x)>0\}=\R\setminus(0,\vt)=(0,\vt)^C
            \end{align*}
            Das bedeutet aber, dass der Träger abhängig vom Parameter $\vt$ macht, also liegt hier keine exponentielle Familie vor.
            \\ \xtab \qed
            }
            \item{\textbf{Aufgabe 2c}\\ Betrachte die Zähldichten der Multinomialverteilung $$P(X = x) = {N \choose x_1\, .\,.\,.\, x_n} \, \prod_{i = 1}^n \, p_i^{x_i},$$  $x = (x_1,\,.\,.\,., x_n), \quad x_i \in \N_0, \quad\sum_{i=1}^n \,x_i = N, \quad p_i \in (0,1), \quad \sum_{i=1}^n p_i = 1.$
        }
            \item{\textit{Behauptung} \\ Die Familie von Dichten $\{f_{\vt}\,|\,\vt\in\Theta = (0,1)^n\}$ bildet eine $n$-parametrige Exponentielle Familie mit den Statistiken $$T_i (x) = x_i$$
        und natürlichem Parameterraum $Z^* = \R^n.$\\\\
        }
        \item{\textit{Beweis}
        \begin{equation*}\begin{split} 
        P_\vt(X = x) &= {N \choose x_1\, .\,.\,.\, x_n} \cdot (p_1^{x_1} \cdot p_2^{x_2} \,.\,.\,. \, \cdot p_n^{x_n}) \\&= \underbrace{\frac{N!}{x_1! \cdot x_2! \cdot \cdot \cdot x_n!}}_{=: K} \cdot (p_1^{x_1} \cdot p_2^{x_2} \,.\,.\,. \, \cdot p_n^{x_n}) \\&= \exp\Big(\log(K) + \log(p_1^{x_1} \cdot p_2^{x_2} \,.\,.\,. \, \cdot p_n^{x_n})\Big) \\&= \exp\Big(\log(K) + x_1 \log(p_1) + x_2 \log(p_2) + \,.\,.\,. \, + x_n \log(p_n)\Big) \\&=: \exp\Big( S(x) + T_1(x) Q_1(\vt) + T_2(x) Q_2(\vt) + \,.\,.\,. + T_n(x) Q_n(\vt) \Big),
        \end{split}\end{equation*} mit \quad$S(x) = \log(K), \quad D(\vt) = 0, \quad T_i(x) = x_i,\quad Q_i(\vt) = \log(\vt_i), \;\;\vt_i \in (0,1).$ \\\\
         $Q(\vt) = (Q_1(\vt), .\,.\,.\,, Q_n(\vt)) \,\Longrightarrow \, Q(\Theta) = (-\infty, 0)^n = Z \ni \xi.$\qquad(nat. Parameter) \\ $A = \{x \in \N_0 \, | \, \sum_{i=1}^n \,x_i = N\}.$ \qquad(Träger) 
        \begin{equation*}\begin{split} {1 \over c(\xi)} &= \sum_{x \in A} \,h(x) \, \exp\Bigg(\sum_{i = 1}^n \,\xi_i \, T_i (x)\Bigg) \\&= \sum_{x \in A} \,{\rm e}^{\log(K)} \, \exp\Bigg(\sum_{i = 1}^n \,\xi_i \, x_i\Bigg) \\&= \sum_{x \in A} \, \underbrace{\frac{N!}{x_1! \cdot x_2! \cdot \cdot \cdot x_n!}}_{ K} \, \exp\Bigg(\sum_{i = 1}^n \,\xi_i \, x_i\Bigg) 
        \end{split}\end{equation*} Beide Summen sind sichtbar endlich, wodurch also ${1 \over c(\xi)} < \infty$ gelten muss. Daraus folgt, dass $Z^\star = \R^n. \;$ $k$ ist sichtbar gleich $n$. \\\xtab\qed
        }
        \item{\textbf{Aufgabe 2d}\\ Betrachte die Dichten der Gamma-Verteilung $$\Gamma(\alpha,\sigma): f_{\alpha,\sigma} (x) = \frac{x^{\alpha -1}e^{-x/\sigma}}{\sigma^\alpha \Gamma(\alpha)} \,I\{x > 0\},\quad \alpha, \sigma > 0, \;\;\vt := (\alpha,\sigma).$$ 
        }
                \item{\textit{Behauptung} \\ Die Familie von Dichten $\{f_{\vt}\,|\,\vt\in\Theta = \R^+ \times \,\R^+ \}$ bildet eine $2$-parametrige Exponentielle Familie mit den Statistiken $$T_1 (x) = -x, \;\; T_2 (x) = \log(x)$$
        und natürlichem Parameterraum $Z^* = \R^2.$
        }
        \item{\textit{Beweis} 
        \begin{equation*}\begin{split}
                f_{\alpha,\sigma} (x) &= \frac{x^{\alpha -1}e^{-x/\sigma}}{\sigma^\alpha \Gamma(\alpha)} \,I\{x > 0\} \\&= \frac{1}{\sigma^\alpha \Gamma(\alpha)} \exp\Bigg(-\frac{x}{\sigma} + \log(x^{\alpha -1}) \Bigg)\,I\{x > 0\} \\&= \exp\Bigg(-x \cdot \frac{1}{\sigma} + \underbrace{\log(x^{\alpha -1})}_{= (\alpha - 1)\log(x)}  + \log\Big(\frac{1}{\sigma^{\alpha}\Gamma(\alpha)}\Big)  \Bigg)\,I\{x > 0\} \\&=: \exp\Bigg(\sum_{i=1}^2 Q_i(\vt)T_i(x) + D(\vt) + S(x)  \Bigg),
        \end{split}\end{equation*} mit \quad$S(x) = 0, \quad D(\vt) = \log\Big(\frac{1}{\sigma^{\alpha}\Gamma(\alpha)}\Big),\\ T_1(x) = -x, \quad T_2(x) = \log(x), \quad Q_1(\vt) =  \frac{1}{\sigma}, \quad Q_2(\vt) = \alpha - 1.$\\\\$A = \R^+$, aufgrund der Indikatorfunktion $I\{x > 0\}$ von $f_{\alpha,\sigma} (x). \;\;\; h(x) = e^{S(x)} = 1.$ 
        \begin{equation*}\begin{split}
                                {1 \over c(\xi)} &= \int_A h(x) \exp\Big(\sum_{i=1}^2 \xi_i T_i(x) \Big) dx 
                                \\&= \int_0^\infty 1\cdot \exp\Big(-\xi_1 x + \xi_2 \log(x)   \Big) dx 
                                \\&= \frac{1}{-\xi_1 +\xi_2 \frac{1}{x}} \exp\Big(-\xi_1 x + \xi_2 \log(x)   \Big) \Bigg|_0^\infty
                                 \\&= \frac{1}{-\xi_1 +\xi_2 \frac{1}{x}} \exp\Big(-\xi_1 x\Big) x \exp\Big(\xi_2 \Big) \Bigg|_0^\infty \quad \longrightarrow 0 \; < \; \infty,
        \end{split}\end{equation*} da $e^{x}$ schneller als $x$ gegen $\infty$ divergiert, konvergiert demnach $e^{-x}$ schneller gegen 0 als $x$ divergiert. Daraus folgt, dass $Z^\star = \R^2$.\; $k$ ist sichtbar gleich 2. \xtab\qed
        }
        \end{description}
        \begin{description}
                         \item{\textbf{Aufgabe 3}\\
                                Sei $X$ eine diskrete Zufallsvariable mit Zähldichte $f\in\{f_{\vt}|\vt\in\Theta\}=\Sigma$ , wobei $\Sigma$ eine einparametrige exponentielle Familie bildet. 
                                \\ $f$ ist von der Form:
                                \begin{align*}
                                        f(k)=P(X=k)=\exp(Q(\vt)T(x)+D(\vt)+S(x))
                                \end{align*}
                                Es bezeichne $A$ den Träger von $f$
                                }
                                \item{\textit{Behauptung}\\
                                Die Zähldichten der diskreten Zufallsvariable $T(X)$ bilden eine einparametrige exponentielle Familie der Form:
                                \begin{align*}
                                        \exp(Q(\vt)x+D(\vt)+S^*(x))
                                \end{align*}
                                }
                                \item{\textit{Beweis}\\
                                Da $X$ diskret ist , ist insbesondere das Urbild von $T$ unter einem $k\in A$ eine diskrete Menge (endlich oder abzählbar unendlich) , da das Urbild aber nicht eindeutig ist (wir wissen nicht ob $T$ invertierbar ist) folgt für unsere Transformation: 
                                \begin{align*}
                                        P(T(X)=x)=P(X\in T^{-1}(x))=\sum_{k\in T^{-1}(x)}f(k) \\
                                        =\sum_{k\in T^{-1}(x)} \exp(Q(\vt)T(k)+D(\vt)+S(k))
                                \end{align*}
                                Da wir über alle $k\in T^{-1}(x)$ summieren , ist insbesondere dann $T(k)=x$, weiter gilt dann also:
                                \begin{align*}
                                        P(T(X)=x)=\exp(Q(\vt)\cdot x+D(\vt)) \cdot \sum_{k\in T^{-1}(x)}\exp(S(k))
                                \end{align*}
                                Da $f$ eine Zähldichte ist und $A\supset T^{-1}(x)$, gilt insbesondere:
                                \begin{align*}
                                        \sum_{k\in A}f(x)=\sum_{k\in A}\exp(Q(\vt)T(k)+D(\vt)+S(k))=1\geq 
                                \end{align*}
                                und damit:
                                \begin{align*}
                                        \sum_{k\in T^{-1}(x)} \exp(Q(\vt)T(k)+D(\vt)+S(k)) \\
                                        = \exp(Q(\vt)\cdot x+D(\vt)) \cdot \sum_{k\in T^{-1}(x)}\exp(S(k)) \leq 1
                                \end{align*}
                                also divergiert die Summe im Fall eines abzählbar unendlichen Urbildes nicht , setze:
                                \begin{align*}
                                        \delta(x)=\log\left(\sum_{k\in T^{-1}(x)}\exp(S(k)\right)
                                \end{align*}
                                und wir erhalten die gewünschte Darstellung, denn dann ist:
                                \begin{align*}
                                        P(T(X)=x)=\exp(Q(\vt)x+D(\vt)+\delta(x)).
                                \end{align*}}
                \end{description}
\end{document}
