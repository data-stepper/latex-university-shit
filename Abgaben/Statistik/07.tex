\documentclass[a4paper]{article}

\usepackage[margin=1in]{geometry} 
\usepackage{amsmath,amsthm,amssymb, graphicx, multicol, array}


\pdfminorversion=7
\pdfsuppresswarningpagegroup=1

\newcommand{\R}{\mathbb{R}}
\newcommand{\N}{\mathbb{N}}
\newcommand{\Z}{\mathbb{Z}}
\newcommand{\beh}{\textit{Behauptung. }}

\setlength{\parindent}{0pt}
\newenvironment{Aufgabe}[2][Aufgabe]{\begin{trivlist}
\item[\hskip \labelsep {\bfseries #1}\hskip \labelsep {\bfseries #2.}]}{\end{trivlist}}

\begin{document}
	\begin{theorem} % Aufgabe #1
	\begin{Aufgabe}{1} % #1
		Fahrkartenkontrolleur
	\end{Aufgabe}

	\textbf{(b)} 

	\begin{proof}[Maximum-Likelihood-Schätzung]
		Als erstes erinnern wir uns an die Dichtefunktion der geometrischen
		Verteilung, welche wie folgt aussieht:
		\[
			P(X= k) = f_\vartheta (k) = \vartheta \cdot (1- \vartheta) ^{k}
			\text{ wobei } k \in \mathbb{N}_0
			\text{ und } \vartheta \in (0, 1)
		\]
		Als erstes bestimmen wir unseren Maximum-Likelihood-Schätzer.
		Hierfür bestimmen wir die log-Likelihood-Funktion welche nämlich
		wie folgt aussieht:
		\begin{align*}
			l (\vartheta, x_1, ..., x_{n}) &= \log \left(
				\underbrace{ \prod_{i=1}^{n} \vartheta (1-\vartheta) ^{x_{i}}  }_{ \text{ gemeinsame Dichte } } 
			\right) \\
			   &= \log \left(
				   \vartheta ^{n} (1 - \vartheta) ^{
						   \sum_{i=1}^{n} x_{i}
				   }
			   \right) \\
			   &= n \cdot \log (\vartheta) + 
			   \left(
				   \sum_{i=1}^{n} x_{i} 
			   \right) \cdot \log (1 - \vartheta)
		\end{align*}
		Nun berechnen wir das Maximum dieser Funktion, indem wir sie ableiten:
		\begin{align*}
			\frac{ \partial l(\vartheta, x_1, ..., x_{n}) }{ \partial \vartheta } &= 
			\frac{ \partial  }{ \partial \vartheta } \left(
			   n \cdot \log (\vartheta) + 
			   \left(
				   \sum_{i=1}^{n} x_{i} 
			   \right) \cdot \log (1 - \vartheta)
			\right) \\
			  &= \frac{ n }{ \vartheta } - \left(
			  	\sum_{i=1}^{n} x_{i}
			  \right) \cdot \frac{ 1 }{ 1 - \vartheta } \\
			\frac{ \partial l(\vartheta, x_1, ..., x_{n}) }{ \partial \vartheta } = 0
			  & \Leftrightarrow 
			  \frac{ n }{ \vartheta } = \frac{ 1 }{ 1- \vartheta } \sum_{i=1}^{n} x_{i} \\
			  & \Leftrightarrow \frac{ 1 - \vartheta }{ \vartheta }
			  = \frac{ 1 }{ \vartheta  } - 1
			  = \frac{ 1 }{ n } \sum_{i=1}^{n} x_{i} \\
			  & \Leftrightarrow \vartheta = \frac{ 1 }{ 1 + \frac{ 1 }{ n } \sum_{i=1}^{n} x_{i} }
		\end{align*}
		Somit brauchen wir nur noch das Stichprobenmittel errechnen und dann in den obigen
		Maximum-Likelihood-Schätzer einsetzen:
		\[
		\overline{X}_n = \frac{ 1 }{ 10 } \left(
			48 + 46 + 42 + 68 + 36 + 30 + 64 + 40 + 50 + 42
		\right) = 46,6
		\] 
		Also schätzen wir nun unseren Parameter $\vartheta$ wie folgt:
		\[
			\vartheta = \frac{ 1 }{ 1 + 46,6 }
			\approx 0.021008403361344536...
		\] 
		Wir können jetzt also sagen, dass nach unserer Maximum-Likelihood-Schätzung der
		Fahrkartenkontrolleur bei jeder Person mit einer Wahrscheinlichkeit von ca. $2.1\%$
		erwarten kann, dass diese keine Fahrkarte besitzt.
	\end{proof}
	\end{theorem}
\end{document}

