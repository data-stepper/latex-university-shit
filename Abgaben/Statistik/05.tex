\documentclass[a4paper]{article}

\usepackage[margin=1in]{geometry} 
\usepackage{amsmath,amsthm,amssymb, graphicx, multicol, array}


\pdfminorversion=7
\pdfsuppresswarningpagegroup=1

\newcommand{\R}{\mathbb{R}}
\newcommand{\N}{\mathbb{N}}
\newcommand{\Z}{\mathbb{Z}}
\newcommand{\beh}{\textit{Behauptung. }}

\setlength{\parindent}{0pt}
\newenvironment{Aufgabe}[2][Aufgabe]{\begin{trivlist}
\item[\hskip \labelsep {\bfseries #1}\hskip \labelsep {\bfseries #2.}]}{\end{trivlist}}

\begin{document}

\begin{theorem} % Aufgabe #2
\begin{Aufgabe}{2} % #2
\end{Aufgabe}

(Teil 1)
\beh $T(X) = \sum_{i=1}^{n} X_i ^2$ ist suffizient und vollständig für $\vartheta$

\begin{proof}[Beweis]
	Um die Suffizienz von $T(X)$ zu zeigen, wenden wir das Faktorisierungskriterium an wie folgt:
	\begin{align*}
		f_\vartheta (x) &= \frac{ x }{ \vartheta } x ^{ \frac{ - x ^2 }{ 2 \vartheta } }
		I_{ ( 0, \infty ) } (x) \\
			&= \left(
				\frac{ 1 }{ \vartheta } e ^{ \frac{ 
						\frac{ 1 }{ n } T(x)
				}{ 2 \vartheta } }
			\right) \cdot x \\
		\text{mit } T(x) &= \sum_{i=1}^{n} x_i ^2
	\end{align*}
	Im Folgenden zeigen wir die Vollständigkeit von $T(X)$:
	Als erstes berechnen wir die Dichte der Summanden in $T(X)$, also die Dichte von
	$X_i ^2$:
	Wir wenden also die Dichtetransformation aus A.1 an, indem wir folgendes setzen:
	\[
		g(x) = x ^2, g ^{-1} (y) = \sqrt{y} , g' (x) = 2 x,
		\frac{ 1 }{ | g'(g ^{-1} (y)) | } = \frac{ 1 }{ | 2 \sqrt{y} | }
	\] 
	Wohlgemerkt ist $g(x) = x ^2$ ein Diffeomorphismus, da wir nur positive Werte für $x$ betrachten.
	Da wir ja die Dichte $f_\vartheta (x)$ von $X$ selbst auch kennen, können die transformierte Dichte ermitteln:
	\begin{align*}
		h(y) & \overset{A.1} = f_\vartheta (g ^{-1} (y)) \cdot \frac{ 1 }{ | g' (g ^{-1} (y)) | }
		= \frac{ \sqrt{y} }{ \vartheta } e ^{ - \frac{ \sqrt{y} ^2 }{ 2 \vartheta } }
		I_{ ( 0, \infty ) } ( \sqrt{y} ) \cdot
		\frac{ 1 }{ 2 \sqrt{y} } \\
			 &= \frac{ 1 }{ 2 \vartheta } e ^{ - \frac{ y }{ 2 \vartheta } } I_{ ( 0, \infty ) } (y) 
	\end{align*}
	Jetzt berechnen wir rekursiv die Dichte von $T(X)$, indem wir zuerst die Dichte $p_{\vartheta, 2}(z)$
	von $X_1 ^2 + X_2 ^2$ berechnen. Da $X_1 ^2$ und $X_2 ^2$ $iid$ sind ist die gemeinsame
	Dichte von $(X_1 ^2, X_2 ^2)$ einfach die Produktdichte.
	Hierzu können aus dem Anhang A.2 (a) verwenden:
	\begin{align*}
		p_{\vartheta, 2} (z) & \overset{A.2} = \int f(z - t, t) dt
		= \int \left(
			\frac{ 1 }{ 2 \vartheta } e ^{ \frac{ - (z - t) }{ 2 \vartheta } }
			I_{ ( 0, \infty ) } (z - t) 
		\right) \cdot
		\left(
			\frac{ 1 }{ 2 \vartheta } e ^{ - \frac{ t }{ 2 \vartheta } }
			I_{ ( 0, \infty ) } (t) 
		\right) dt \\
			 &= \left(
				\frac{ 1 }{ 2 \vartheta }
			 \right) ^2
			 \int \exp \left(
				\frac{ 1 }{ 2 \vartheta } \left(
					 - (z - t) - t
			 \right)  )
			 \right) 
			 I_{ ( 0, \infty ) } (z - t) \cdot I_{ ( 0, \infty ) } (t) 
			 \; dt \\
			 &= \left(
			 	\frac{ 1 }{ 2 \vartheta }
			 \right) ^2 \int \exp 
			 \left(
				 \frac{ - z }{ 2 \vartheta } 
			 \right) I_{ ( 0, \infty ) } (z - t) I_{ ( 0, \infty ) } (t) \; dt \\
			 &= \left(
			 	\frac{ 1 }{ 2 \vartheta }
			\right) ^2 \exp \left(
				\frac{ -z }{ 2 \vartheta }
			\right) \int 1 \cdot \; I_{ ( 0, z ) } (t) \; dt \\
			 &= \left(
			 	\frac{ 1 }{ 2 \vartheta }
			\right) ^2 \exp \left(
				\frac{ -z }{ 2 \vartheta }
			\right) \cdot z \cdot \; I_{ ( 0, \infty ) } (z) 
	\end{align*}
	Nun berechnen nach unserem rekursiven Ansatz die Dichte $p_{\vartheta, 3}$ von $X_1 ^2 + X_2 ^2 + X_3 ^2$.
	Wohlgemerkt ist unser Zufallsvariablentupel hier $(X_3 ^2, X_1 ^2 + X_2 ^2)$, sodass wir
	die im vorigen ermittelte Dichte verwenden können.
	\begin{align*}
		p_{\vartheta, 3} (z) &= \int \left(
			\frac{ 1 }{ 2 \vartheta } e ^{ \frac{ - (z - t) }{ 2 \vartheta }}
			I_{ ( 0, \infty ) } (z - t) 
		\right) \cdot \left(
			\left(
				\frac{ 1 }{ 2 \vartheta }
			\right) ^2
			e ^{ \frac{ -t }{ 2 \vartheta } } \cdot t \cdot \;
			I_{ ( 0, \infty ) } (t) 
		\right) dt \\
			 &= \left(
			 	\frac{ 1 }{ 2 \vartheta }
			 \right) ^3
			 \int \exp \left(
			 	\frac{ 1 }{ 2 \vartheta }
				( - (z - t) - t )
			 \right) t \; \cdot I_{ ( 0, z ) } (t) \; dt \\
			 &= \left(
			 	\frac{ 1 }{ 2 \vartheta }
			 \right) ^3 \cdot \exp \left(
			 	\frac{ -z }{ 2 \vartheta }
			 \right) \cdot \int t \cdot I_{ ( 0, z ) } (t) \; dt \\
			 &= \left(
			 	\frac{ 1 }{ 2 \vartheta }
			 \right) ^3 \cdot \exp \left(
			 	\frac{ -z }{ 2 \vartheta }
			\right) \cdot \frac{ z ^2 }{ 2 } \cdot I_{ ( 0, \infty ) } (z) 
	\end{align*}
	Wir erkennen also, dass sich das Muster wiederholt und können so auch die Dichte bestimmen
	wenn wir $n$ viele unserer $X_i ^2$ summieren. Worauf wir nun aber achten müssen ist, dass
	obwohl zwar aus unseren Integralen alle Faktoren herausfallen immer einer übrig bleibt
	über welchen wir dann integrieren. Dieser ist dann jeweils $t ^{k}$ (bei der Summe bis $k$)
	und dessen Integral über den jeweiligen Indikator ist dann $\frac{ 1 }{ k+1 } z ^{k + 1}$.
	Dieser Vorfaktor bleibt allerdings, sodass wir am Ende einen Vorfaktor von $\frac{ 1 }{ (n-1)! }$
	erhalten, da wir ja alle vorherigen miteinander multiplizieren.
	Folglich erhalten wir die Dichte $p_{\vartheta, n}$ von $T(X)$:
	\[
		p_{\vartheta, n} (z) = \left(
			\frac{ 1 }{ 2 \vartheta }
		\right) ^{n} \cdot \exp \left(
			\frac{ -z }{ 2 \vartheta }
		\right) \cdot \frac{ z ^{ n - 1 } }{ (n-1) ! } \cdot I_{ ( 0, \infty ) } (z) 
	\] 
	Nun wollen wir hiermit zeigen, dass $T(X)$ vollständig ist. Wir betrachten also:
	\[
	E \left[
		g(T(X))
	\right] = \left(
		\frac{ 1 }{ 2 \vartheta }
	\right) ^{n} \cdot \frac{ 1 }{ (n-1)! }
	\int g(t) \exp \left(
		\frac{ -t }{ 2 \vartheta }
	\right) \cdot t ^{n-1} \, dt = 0
	\] 
	Dies muss für alle messbaren $g$ implizieren, dass $g$ jeweils überall $0$ ist.
	Allerdings darf $g$ aber nicht von $\vartheta$ abhängen. Nehmen wir also an es gibt
	ein $g \neq 0$ sodass die obige Bedingung erfüllt ist. Dann überlegen wir uns, dass
	diese Funktion dann in irgendeiner Form periodisch sein muss, da ja die Dichte selbst
	überall positiv ist. Außerdem müssen sich jeweils alle positiven Bereiche mit den
	negativen ausgleichen. Die Höhe der Kurve wird aber unter anderem durch den
	$\exp$ -Term bestimmt und somit auch durch $\vartheta$. Wir sehen also, dass die einzige
	Möglichkeit für $g$ die Bedingung zu erfüllen und nicht von $\vartheta$ abzuhängen $g=0$ ist.
\end{proof}

(Teil 2) \textit{Bestimmen} sie einen für $\vartheta$ erwartungstreuen Schätzer der nur von
$T$ abhängt.

\begin{proof}[Berechnung]
	Zuerst berechnen wir einen den Erwartungswert von $T(X)$, indem wir den Erwartungswert von
	$X_i ^2$ berechnen per Transformationsformel:
	\begin{align*}
		E \left[
			X_1 ^2
		\right] &= \int x ^2 \left(
			\frac{ x }{ \vartheta } \exp \left(
				\frac{ -x ^2 }{ 2 \vartheta }
			\right) 
		\right) \cdot I_{ ( 0, \infty ) } (x) \; dx \\
			&= \frac{ 1 }{ \vartheta } \int_{0}^{\infty} x ^3 \exp \left(
				\frac{ -x ^2 }{ 2 \vartheta }
			\right) dx
	\end{align*}
	Wir substituieren nun wie folgt:
	\[
	u = \frac{ -x ^2 }{ 2 \vartheta },\; du = \frac{ -x }{ \vartheta } dx,\;
	dx = \frac{ \vartheta }{ -x } du,\; x = \sqrt{- 2 \vartheta u},\;
	x ^2 = - 2\vartheta u
	\] 
	Nun erhalten wir aus dem vorigen:
	\begin{align*}
		E \left[
			X_1 ^2
		\right] &= \frac{ 1 }{ \vartheta }
		\int_{0}^{- \infty} x \cdot (- 2 \vartheta u) \exp (u) \frac{ \vartheta }{ -x } du \\
				&= 2 \vartheta \int_{0}^{- \infty} u \exp(u) \; du \\
				&= 2 \vartheta \left(
					- \int_{0}^{- \infty} \exp(u) \; du + \left[
						u \exp(u)
					\right]_0^{- \infty}
				\right) \\
				&= 2 \vartheta \left[
					(u - 1) \exp(u)
				\right]_0^{- \infty} \\
				&= 2 \vartheta \left[
					\left(
						\frac{ - x ^2 }{ 2 \vartheta } - 1
					\right) 
					 \exp \left(
						\frac{ -x ^2 }{ 2 \vartheta }
					\right) 
				\right]_0^{\infty} \\
				&= 2 \vartheta
	\end{align*}
	Jetzt berechnen wir den Erwartungswert von $T(X)$:
	\begin{align*}
		E \left[
			T(X)
		\right] &= E \left[
			\sum_{i=1}^{n} X_i ^2
		\right] 
		= \sum_{i=1}^{n} E \left[
			X_i ^2
		\right] \overset{iid.} = n \cdot E \left[
			X_1 ^2
		\right] = 2 \vartheta n
	\end{align*}
	Ganz simpel können wir jetzt $T(X)$ mit $\frac{ 1 }{ 2 n }$ multiplizieren und erhalten
	so einen erwartungstreuen Schätzer für $\vartheta$.
	\[
		S (x) = \frac{ x }{ 2 n } \implies E \left[
			S(T(X)) 
		\right] = E \left[
			\frac{ T(X) }{ 2 n }
		\right] = \vartheta
	\] 
\end{proof}
\end{theorem}

\end{document}

