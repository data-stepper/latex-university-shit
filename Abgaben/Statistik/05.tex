\documentclass[a4paper]{article}

\usepackage[margin=1in]{geometry} 
\usepackage{amsmath,amsthm,amssymb, graphicx, multicol, array}


\pdfminorversion=7
\pdfsuppresswarningpagegroup=1

\newcommand{\R}{\mathbb{R}}
\newcommand{\N}{\mathbb{N}}
\newcommand{\Z}{\mathbb{Z}}
\newcommand{\beh}{\textit{Behauptung. }}

\setlength{\parindent}{0pt}
\newenvironment{Aufgabe}[2][Aufgabe]{\begin{trivlist}
\item[\hskip \labelsep {\bfseries #1}\hskip \labelsep {\bfseries #2.}]}{\end{trivlist}}

\begin{document}

\begin{theorem} % Aufgabe #2
\begin{Aufgabe}{2} % #2
\end{Aufgabe}

(Teil 1)
\beh $T(X) = \sum_{i=1}^{n} X_i ^2$ ist suffizient und vollständig für $\vartheta$

\begin{proof}[Beweis]
	Um die Suffizienz von $T(X)$ zu zeigen verwenden wir das Faktorisierungskriterium,
	indem wir die gemeinsame Dichte $f_\vartheta (x_1, ..., x_{n})$ von unseren
	$X_i$ bestimmen:
	\begin{align*}
		f_\vartheta (x_1, ..., x_{n}) &= \prod_{i=1}^{n} f_\vartheta (x_{i}) \\
				  &= \prod_{i=1}^{n} \left(
				  	\frac{ x_{i} }{ \vartheta } e ^{\frac{ - x_{i} ^2 }{ 2 \vartheta }}
					\cdot I_{ ( 0, \infty ) } (x_{i}) 
				  \right) \\
				  &= \frac{ 1 }{ \vartheta ^{n} } e ^{ \frac{ - 1 }{ 2 \vartheta } \cdot 
				  \sum_{i=1}^{n} x_{i} ^2
			  }
			  \prod_{i=1}^{n} x_{i} \cdot I_{ ( 0, \infty ) } (x_{i})  
	\end{align*}
	Jetzt können wir erkennen, dass unsere Statistik das Faktorisierungskriterium erfüllt, indem wir
	aus der Definition folgendes setzen:
	\[
		g_\vartheta (x) = \frac{ 1 }{ \vartheta ^{n} } e ^{ \frac{ -x }{ 2 \vartheta } },
		\; h(x_1, ..., x_{n}) = \prod_{i=1}^{n} x_{i} \cdot I_{ ( 0, \infty ) } (x_{i})  
	\] 
	Wobei offensichtlich der Parameter von $g_\vartheta$ $T(X)$ ist. Es folgt dass $T(X)$ suffizient ist,
	da:
	\[
		f_\vartheta (x) = g_\vartheta (T(x)) \cdot h(x), \;
		\text{mit } x = (x_1, ..., x_{n})
	\] 
	Um die Vollständigkeit von $T(X)$ zu zeigen schreiben wir die gemeinsame Dichte
	$f_\vartheta(x_1, ..., x_{n})$ so um, dass wir Satz 5.11 anwenden können. Im Folgenden
	sei $x = ( x_1 , ..., x_{n} )$.
	\begin{align*}
		f_\vartheta(x) &= \frac{ 1 }{ \vartheta ^{n} } e ^{ \frac{ - T(X) }{ 2 \vartheta } }
		\cdot \prod_{i=1}^{n} x_{i} I_{ ( 0, \infty ) } (x_{i}) \\
			   &= \exp \left(
				   - n \cdot \ln (\vartheta)
				   - \frac{ T(X) }{ 2 \vartheta }
				   + \sum_{i=1}^{n} \ln x_{i}
			   \right) 
	\end{align*}
	Und jetzt können wir auch erkennen, dass unsere Dichte Teil der exponentiellen Familie ist
	mit Folgendem:
	\[
		Q(\vartheta) = \frac{ -1 }{ 2 \vartheta }, \;
		T(x) = T(x) = \sum_{i=1}^{n} x_{i} ^2, \;
		D(\vartheta) = - n \ln \vartheta, \;
		S(x) = \sum_{i=1}^{n} \ln x_{i}
	\] 
	Um den Satz 5.11 jetzt aber benutzen zu dürfen benötigen wir immernoch eine offene Teilmenge
	reeller Zahlen in der Menge der $Q(\vartheta)$ aus unserem Parameterraum.
	Diese können wir aber wie folgt finden:
	\begin{align*}
		\mathcal{Q} &= \{
			Q(\vartheta) = \frac{ -1 }{ 2 \vartheta } \; \Big | \; \vartheta > 0
		\} \\
					&= (- \infty, 0)
	\end{align*}
	Somit sind also die Bedingungen für Satz 5.11 erfüllt und wir haben gezeigt, dass $T(X)$
	auch vollständig ist.
\end{proof}

(Teil 2) \textit{Bestimmen} sie einen für $\vartheta$ erwartungstreuen Schätzer der nur von
$T$ abhängt.

\begin{proof}[Berechnung]
	Zuerst berechnen wir einen den Erwartungswert von $T(X)$, indem wir den Erwartungswert von
	$X_i ^2$ berechnen per Transformationsformel:
	\begin{align*}
		E \left[
			X_1 ^2
		\right] &= \int x ^2 \left(
			\frac{ x }{ \vartheta } \exp \left(
				\frac{ -x ^2 }{ 2 \vartheta }
			\right) 
		\right) \cdot I_{ ( 0, \infty ) } (x) \; dx \\
			&= \frac{ 1 }{ \vartheta } \int_{0}^{\infty} x ^3 \exp \left(
				\frac{ -x ^2 }{ 2 \vartheta }
			\right) dx
	\end{align*}
	Wir substituieren nun wie folgt:
	\[
	u = \frac{ -x ^2 }{ 2 \vartheta },\; du = \frac{ -x }{ \vartheta } dx,\;
	dx = \frac{ \vartheta }{ -x } du,\; x = \sqrt{- 2 \vartheta u},\;
	x ^2 = - 2\vartheta u
	\] 
	Nun erhalten wir aus dem vorigen:
	\begin{align*}
		E \left[
			X_1 ^2
		\right] &= \frac{ 1 }{ \vartheta }
		\int_{0}^{- \infty} x \cdot (- 2 \vartheta u) \exp (u) \frac{ \vartheta }{ -x } du \\
				&= 2 \vartheta \int_{0}^{- \infty} u \exp(u) \; du \\
				&= 2 \vartheta \left(
					- \int_{0}^{- \infty} \exp(u) \; du + \left[
						u \exp(u)
					\right]_0^{- \infty}
				\right) \\
				&= 2 \vartheta \left[
					(u - 1) \exp(u)
				\right]_0^{- \infty} \\
				&= 2 \vartheta \left[
					\left(
						\frac{ - x ^2 }{ 2 \vartheta } - 1
					\right) 
					 \exp \left(
						\frac{ -x ^2 }{ 2 \vartheta }
					\right) 
				\right]_0^{\infty} \\
				&= 2 \vartheta
	\end{align*}
	Jetzt berechnen wir den Erwartungswert von $T(X)$:
	\begin{align*}
		E \left[
			T(X)
		\right] &= E \left[
			\sum_{i=1}^{n} X_i ^2
		\right] 
		= \sum_{i=1}^{n} E \left[
			X_i ^2
		\right] \overset{iid.} = n \cdot E \left[
			X_1 ^2
		\right] = 2 \vartheta n
	\end{align*}
	Ganz simpel können wir jetzt $T(X)$ mit $\frac{ 1 }{ 2 n }$ multiplizieren und erhalten
	so einen erwartungstreuen Schätzer für $\vartheta$.
	\[
		S (x) = \frac{ x }{ 2 n } \implies E \left[
			S(T(X)) 
		\right] = E \left[
			\frac{ T(X) }{ 2 n }
		\right] = \vartheta
	\] 
\end{proof}
\end{theorem}

\end{document}

