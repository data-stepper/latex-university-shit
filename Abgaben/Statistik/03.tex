\documentclass[a4paper]{article}

\usepackage[margin=1in]{geometry} 
\usepackage{amsmath,amsthm,amssymb, graphicx, multicol, array}


\pdfminorversion=7
\pdfsuppresswarningpagegroup=1

\newcommand{\R}{\mathbb{R}}
\newcommand{\N}{\mathbb{N}}
\newcommand{\Z}{\mathbb{Z}}
\newcommand{\beh}{\textit{Behauptung. }}

\setlength{\parindent}{0pt}
\newenvironment{Aufgabe}[2][Aufgabe]{\begin{trivlist}
\item[\hskip \labelsep {\bfseries #1}\hskip \labelsep {\bfseries #2.}]}{\end{trivlist}}

\begin{document}

\begin{theorem} % Aufgabe #2
\begin{Aufgabe}{2} % #2
\end{Aufgabe}

\textbf{a)}
\beh $T \left(
	X_1, ..., X_{n}
\right) = \frac{ 6 }{ n( n+1 ) ( 2n +1 ) }
\sum_{n=1}^{n} i ^2 X_{i}$ schätzt konsistent $\mu$.

\begin{proof}[Beweis]
	Aus dem Skript wissen wir, dass ein Schätzer genau dann konsistent ist wenn
	er erwartungstreu (1) ist und seine Varianz im Grenzfall gegen $0$ geht (2).
	\begin{align*}
	E \left[
		T \left(
			X_1, ..., X_n
		\right)
	\right] &= E \left[
	\frac{ 6 }{ n( n+1 )( 2n +1 ) }
		\sum_{i=1}^{n} i ^2 X_i
	\right] \\
			&= \frac{ 6 }{n ( n+1 ) ( 2n +1)}
			\sum_{i=1}^{n} i ^2 E \left[
				X_i
			\right] \\
			&= \frac{ 6 }{ n( n+1 ) ( 2n +1) } E \left[
				X_1
			\right] \sum_{i=1}^{n} i ^2 \\
			&= \frac{ 6 }{ n( n+1 ) ( 2n +1) } E \left[
				X_1
			\right] \left(
			\frac{ n (n+ 1 ) ( 2n +1)}{ 6 }
			\right) = \mu
	\end{align*}
	Oben haben wir die Gaußsche Summenformel für Quadrate benutzt. Nun ist nur noch zu
	zeigen, dass die Varianz unseres Schätzers im Grenzfall gegen $0$ geht. Um dies zu
	zeigen berechnen wir zuerst die Varianz selbst:
	\begin{align*}
		Var ( 
		T \left(
			X_1, ..., X_n
		\right)
		) &= E \left[
			\left(
				T \left(
					X_1, ..., X_n
				\right)
			\right) ^2
		\right] - E \left[
			T \left(
				X_1, ..., X_n
			\right)
		\right] ^2 \\
		  &= E \left[
		  	\left(
				\frac{ 6 }{ n(n+1)(2n+1) }
				\sum_{i=1}^{n} i ^2 X_i
		  	\right) ^2
		  \right] - \mu ^2 \\
		  &= \frac{ 36 }{ n ^2 (n+1) ^2 (2n + 1) ^2 }
		  E \left[
		  	\left(
		  		\sum_{i=1}^{n} i ^2 X_i
		  	\right) ^2
		  \right] - \mu ^2 \\
		  \text{Bem.: } E \left[
		  	\left(
		  		\sum_{i=1}^{n} i ^2 X_i
		  	\right) ^2
		\right] &= E \left[
			\left(
				\sum_{i=1}^{n} i ^2 X_i
			\right) 
			\left(
				\sum_{j=1}^{n} j ^2 X_j
			\right) 
		\right] \\
				&= E \left[
					\sum_{i=i}^{n} 
					\sum_{j=1}^{n} i ^2 j ^2 X_i X_j
				\right] \\
				&= \sum_{i=i}^{n} 
					\sum_{j=1}^{n} i ^2 j ^2 E \left[
						X_i X_j
					\right] \\
				&= \frac{ n ^2 (n+1) ^2 (2n +1) ^2 }{ 36 } E \left[
					X_1
				\right]  ^2 \\
				& \implies Var(
					T \left(
						X_1, ..., X_n
					\right)
				) = \mu ^2 - \mu ^2 = 0
	\end{align*}
	Nun sehen wir also, dass die Varianz unseres Schätzers sowieso $0$ ist. Also folgt
	dies auch insbesondere im Grenzfall.
\end{proof}
\end{theorem}

\begin{theorem} % Aufgabe #2
\textbf{b)}

\beh $S_n (X_1, ..., X_{n}) = \left(
	\prod_{i=1}^{n} X_{i}
\right) ^{\frac{ 1 }{ n }} $ ist konsistenter Schätzer
für $\frac{ \vartheta }{ e }$ 

\begin{proof}[Beweis]
	Aus dem Skript wissen wir, dass es reicht die folgende Aussage zu zeigen:
	\[
		\left(
			\prod_{i=1}^{n} X_{i}
		\right) ^{\frac{ 1 }{ n }} 
		= S_n (X ^{(n)}) \overset{P}
		\to \psi(\vartheta) = \frac{ \vartheta }{ e }
	\] 
	Zuerst beobachten wir was mit unserem Schätzer passiert wenn wir diesen
	logarithmieren:
	\[
		\log \left(
			\left(
				\prod_{i=1}^{n} X_{i}
			\right) ^{\frac{ 1 }{ n }} 
		\right) =
			\frac{ 1 }{ n }
			\sum_{i=1}^{n} \log(X_i)
	\] 
	Nun wissen wir aber nach dem Zentralen Grenzwertsatz gerade folgendes:
	\[
		\frac{ 1 }{ n } \sum_{i=1}^{n} \log(X_i)
		\overset{P} \to E \left[
			\log(X_1)
		\right] 
	\] 
	Diesen Erwartungswert können wir jetzt einfach nach der Transformationsformel
	berechnen, da wir ja wissen, dass unsere Zufallsvariable $X_1$ gleichverteilt ist:
	\begin{align*}
		E \left[
			\log(X_1)
		\right] &= \frac{ 1 }{ \vartheta }
		\int_{0}^{\vartheta} \log(x) dx \\
				&= \frac{ 1 }{ \vartheta }  \left[
					x \log(x) - x
				\right]_0 ^{\vartheta} \\
				&= \frac{ 1 }{ \vartheta } \left(
					\vartheta \log ( \vartheta ) - \vartheta
				\right) \\
				&= \log(\vartheta) - 1
	\end{align*}
	Jetzt können wir wie folgt den ursprünglichen Schätzer wieder konstruieren, indem
	wir den logarithmierten Schätzer exponentieren:
	\begin{align*}
		\left(
			\prod_{i=1}^{n} X_{i}
		\right) ^{\frac{ 1 }{ n }} &=
		\exp 
			\left(
				\log 
				\left(
					\left(
						\prod_{i=1}^{n} X_{i}
					\right) ^{\frac{ 1 }{ n }} 
				\right) 
			\right) \\
								   &=
		\exp \left(
			\frac{ 1 }{ n } \sum_{i=1}^{n} \log(X_i)
		\right) \overset{P} \to \exp(
			E \left[
				\log(X_1)
			\right] 
		) \\
								   &=
			\exp (\log (\vartheta) - 1) 
			= \frac{ \vartheta }{ e }
	\end{align*}
	Das obige gilt offensichtlich, da exponentieren eine stetige Transformation ist.
\end{proof}
\end{theorem}


\end{document}

