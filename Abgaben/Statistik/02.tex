\documentclass[a4paper]{article}

\usepackage[margin=1in]{geometry} 
\usepackage{amsmath,amsthm,amssymb, graphicx, multicol, array}


\pdfminorversion=7
\pdfsuppresswarningpagegroup=1

\newcommand{\R}{\mathbb{R}}
\newcommand{\N}{\mathbb{N}}
\newcommand{\Z}{\mathbb{Z}}
\newcommand{\beh}{\textit{Behauptung. }}

\setlength{\parindent}{0pt}
\newenvironment{Aufgabe}[2][Aufgabe]{\begin{trivlist}
\item[\hskip \labelsep {\bfseries #1}\hskip \labelsep {\bfseries #2.}]}{\end{trivlist}}

\begin{document}

\begin{document}
\maketitle

%% Start aufgabe 1c

\begin{theorem} % Aufgabe #1
\begin{Aufgabe}{1} % #1
	c)
\end{Aufgabe}

\beh $\hat{S_n} ^2 \overset{P}\longrightarrow Var (X_1)$ 

\begin{proof}[Beweis]
	Wir benutzen die Tschebyschev-Ungleichung aus 5.12 b) aus dem Stochastik Skript.
	In dieser setzen wir $X = \hat{S_n ^2}$ und erhalten folgendes:
	\[
		P(| \hat{S_n ^2} - E \left[
			\hat{S_n ^2} 
			\right] | \geq \epsilon) \leq \frac{ 
			Var (\hat{S_n ^2} )
		}{ \epsilon ^2 }
	\] 
	Allerdings kennen wir aus der vorherigen Aufgabe (1b) $Var (\hat{S_n ^2} )$ und
	wissen auch, dass diese bei $n \to \infty$ gegen $0$ geht. Also gilt:
	\[
		Var (\hat{S_n ^2} ) = \frac{ 1 }{ n } E \left[
			(X_1 - E \left[
				X_1
			\right] ) ^{4}
		\right] \frac{ 3-n }{ n(n-1) } (Var (X_1))^2
	\] 
	Und hier sehen wir eindeutig, dass dieser Term bei $n \to \infty$ gegen 0 geht, da
	beide Summanden den Faktor $\frac{ 1 }{ n }$ haben. Also wissen wir aus der
	Tschebyschev-Ungleichung, dass auch der linke Term der Ungleichung gegen 0
	geht.
	\[
		P(| \hat{S_n ^2}  - E \left[
			\hat{S_n ^2} 
		\right] | \ge \epsilon) \le \frac{ 0 }{ \epsilon ^2 }
		\text{ (bei $n \to  \infty$ ) }
	\] 
	Nun ist aber die Aussage von stochastischer Konvergenz genau die, dass
	diese Warscheinlichkeit eben im Grenzfall $n \to  \infty$ genau 0 ist.
	Somit ist die Behauptung bewiesen.
\end{proof}
\end{theorem}

\begin{theorem} % Aufgabe #2
\begin{Aufgabe}{2} % #2
	Konvergenz in Verteilung des 4. Stichprobenmoments zeigen.
\end{Aufgabe}

\beh $\sqrt{n} (a_{4,n} - E \left[
	X_1 ^{4}
\right] ) \overset{D} \longrightarrow \mathcal{N} (0, \tau ^2)$

\begin{proof}[Beweis]
	Mit Hilfe des Zentralen Grenzwertsatzes und Korollar 14.2 aus dem Stochastik Skript
	wissen wir direkt:
	\begin{align}
	\frac{ 1 }{ \sqrt{n}  }\left(
		\sum_{i=1}^{n} X_i ^{4} - n E \left[
			X_1 ^{4}
		\right] 
	\right) \overset{D} \longrightarrow \mathcal{N} (0, Var (X_1 ^{ 4 }))
	\end{align}
	Nun formen wir wie folgt um und erhalten:
	\begin{align*}
		&\frac{ 1 }{ \sqrt{n}  } \left(
			\sum_{i=1}^{n} X_i ^{4} - n E \left[
				X_1 ^{4}
			\right] 
		\right) \\
		&= \frac{ 1 }{ \sqrt{n} }\; n \left(
			\frac{ 1 }{ n } \sum_{i=1}^{n} X_i ^{4} - E \left[
				X_1 ^{4}
			\right] 
		\right) \\
		&= \sqrt{n} \left(
			a_{4, n} - E \left[
				X_1 ^{4}
			\right] 
		\right) 
	\end{align*}
	Nun wissen wir auch direkt aus Gleichung (1), dass $Var (X_1 ^{4}) = \tau ^2$.
	Diese Gleichung können wir nun weiter umformen und erhalten:
	\begin{align*}
		\tau ^2 &= Var (X_1 ^{4}) = E \left[
			X_1 ^{8}
		\right] - E \left[
			X_1 ^{4}
		\right] ^2 \\
			&= \mu ^{8} + 28 \mu ^{6} \sigma ^2 + 210 \mu ^{4} \sigma ^{4}
			+ 420 \mu ^2 \sigma ^{6} + 105 \sigma ^{8}
			- \left(
				\mu ^{4} + 6 \mu ^2 \sigma ^2 + 3 \sigma ^{4}
			\right) ^2 \\
			&= 16 \mu ^{6} \sigma ^2 + 168 \mu ^{4} \sigma ^{4} + 384 \mu ^2
			\sigma ^{6} + 96 \sigma ^{8}
	\end{align*}
\end{proof}
\end{theorem}

\end{document}
