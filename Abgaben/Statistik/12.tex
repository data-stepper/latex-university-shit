\documentclass[a4paper]{article}

\usepackage[margin=1in]{geometry} 
\usepackage{amsmath,amsthm,amssymb, graphicx, multicol, array}

\usepackage{tikz}
\usetikzlibrary{automata, positioning}

\pdfminorversion=7
\pdfsuppresswarningpagegroup=1

\newcommand{\R}{\mathbb{R}}
\newcommand{\N}{\mathbb{N}}
\newcommand{\Z}{\mathbb{Z}}
\newcommand{\beh}{\textit{Behauptung. }}

\setlength{\parindent}{0pt}
\newenvironment{Aufgabe}[2][Aufgabe]{\begin{trivlist}
\item[\hskip \labelsep {\bfseries #1}\hskip \labelsep {\bfseries #2.}]}{\end{trivlist}}

\begin{document}
\title{ \textbf{Statistik Blatt \# 12} }
\author{Bent Müller, Ferdinand Cramm, Steven Zinger}
\date{30.06.2021}
\maketitle
	\begin{Aufgabe}{2}
		Genotypen
	\end{Aufgabe}

	\begin{proof}[Beweis]
		Zuerst beobachten wir, dass wir unser Experiment mit Hilfe einer
		Multinomialverteilung modellieren können.
		\\

		Die Zähldichte können wir somit wie folgt aufstellen:
		\[
			P(X_j = x_{j} ,j \in \{
				1, 2, ..., m
				\}) = \frac{ n! }{ 
				\prod_{k = 1}^{m} x_k!
			}
			\prod_{k = 1}^{m} p_k ^{x_k} 
		\]
		In unserem Fall ist $m=3$, da wir drei verschiedene
		Ausgangssituationen vorliegen haben ($AA, Aa, aa$).
		Da wir allerdings noch mehr über die Wahrscheinlichkeiten in unserem
		Modell wissen, können wir die Zähldichte noch präziser formulieren.
		\begin{align*}
			P(X_j = x_{j}, j \in \{
				1, 2, 3
			\} ) &= 
		\end{align*}
		Für den Maximum-Likelihood-Schätzer schauen wir uns die folgende
		log-Likelihood Funktion an, für welche wir dann auch direkt das
		Maximum finden.
		\begin{align*}
			l (p, )
		\end{align*}
	\end{proof}
\end{document}
