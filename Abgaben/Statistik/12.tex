\documentclass[a4paper]{article}

\usepackage[margin=1in]{geometry} 
\usepackage{amsmath,amsthm,amssymb, graphicx, multicol, array}

\usepackage{tikz}
\usetikzlibrary{automata, positioning}

\pdfminorversion=7
\pdfsuppresswarningpagegroup=1

\newcommand{\R}{\mathbb{R}}
\newcommand{\N}{\mathbb{N}}
\newcommand{\Z}{\mathbb{Z}}
\newcommand{\beh}{\textit{Behauptung. }}

\setlength{\parindent}{0pt}
\newenvironment{Aufgabe}[2][Aufgabe]{\begin{trivlist}
\item[\hskip \labelsep {\bfseries #1}\hskip \labelsep {\bfseries #2.}]}{\end{trivlist}}

\begin{document}
\title{ \textbf{Statistik Blatt \# 12} }
\author{Bent Müller, Ferdinand Cramm, Steven Zinger}
\date{30.06.2021}
\maketitle
	\begin{Aufgabe}{2}
		Genotypen
	\end{Aufgabe}

	\begin{proof}[Beweis]
		Zuerst beobachten wir, dass wir unser Experiment mit Hilfe einer
		Multinomialverteilung modellieren können.
		\\

		Die Zähldichte können wir somit wie folgt aufstellen:
		\[
			P(X_j = x_{j} ,j \in \{
				1, 2, ..., m
				\}) = \frac{ n! }{ 
				\prod_{k = 1}^{m} x_k!
			}
			\prod_{k = 1}^{m} p_k ^{x_k} 
		\]
		In unserem Fall ist $m=3$, da wir drei verschiedene
		Ausgangssituationen vorliegen haben ($AA, Aa, aa$).
		Da wir allerdings noch mehr über die Wahrscheinlichkeiten in unserem
		Modell wissen, können wir die Zähldichte noch präziser formulieren.
		\[
			P(X_j = x_{j}, j \in \{
				1, 2, 3
				\} ) &= \frac{ n! }{ 
				x_1 \cdot x_2 \cdot x_3
			} \left(
				p ^2
			\right) ^{x_1} \cdot
			\left(
				2p(1-p)
			\right) ^{x_2} \cdot
			\left(
				1 - p ^2
			\right) ^{x_3}
		\] 
		Wichtig hier ist, dass $x_j$ jeweils diskret die Anzahl der 
		Genotypen beschreibt.
		Für den Maximum-Likelihood-Schätzer schauen wir uns die folgende
		log-Likelihood Funktion an, für welche wir dann auch direkt das
		Maximum finden.
		\begin{align*}
			l (p, x=n) &=
			\log \left(
				L(p, x_1 = n_1, x_2 = n_2, x_3 = n_3)
			\right) \\
			   &= 
			   \log
			\left(
			    \frac{ n! }{ 
				x_1 \cdot x_2 \cdot x_3
			} \left(
				p ^2
			\right) ^{x_1} \cdot
			\left(
				2p(1-p)
			\right) ^{x_2} \cdot
			\left(
				1 - p ^2
			\right) ^{x_3}
			\right) \\
				&= \log (n!) - \log \left(
					x_1 x_2 x_3
				\right) + 2 x_1 \log \left(
					p
				\right) + x_2 \log \left(
					2p (1-p)
				\right) + x_3 \log \left(
					1 - p ^2
				\right) \\
				&= \log (n!) - \log \left(
					x_1 x_2 x_3
				\right) + 2 x_1 \log \left(
					p
				\right) + x_2 \log \left(
				2p - 2p ^2
				\right) + x_3 \log \left(
					1 - p ^2
				\right) \\
				\Rightarrow \frac{ \partial l (p, x=n) }{ \partial p } &= 
				\frac{ 2 x_1 }{ p } +
				\left(
					2 - 4p
				\right) 
				\frac{ x_2 }{ 2 (p - p ^2) }
				\; - 2p \frac{ x_3 }{ 1 - p ^2 } = 0 \\
				   & \Leftrightarrow \frac{ x_1 }{ p }
				   + \frac{ x_2 - 2p x_2 }{ 2(p - p ^2) }
				   - \frac{ p x_3 }{ 1 - p ^2 } = 0 \\
				   & \overset{p \neq 0} \Longleftrightarrow 
				   x_1 + \frac{ x_2 (1 - 2p) }{ 2 (1- p) }
				   - \frac{ p ^2 x_3 }{ 1 - p ^2 } = 0 \\
				   & \overset{p \neq 1} \Longleftrightarrow 
				   (1 - p) x_1 + \frac{ x_2 (1 - 2p) }{ 2 }
				   - \frac{ p ^2 x_3 }{ 1 + p } = 0 \\
				   & \Leftrightarrow x_1 + p \left(
				   		- x_1 - x_2 
				   \right) - \frac{ p ^2 x_3 }{ 1 + p }
				   + \frac{ x_2 }{ 2 } = 0 \qquad \big \vert \cdot (1+p) \\
				   & \Leftrightarrow x_1 + p \left(
				   		- x_2 
				   \right)
				   + p ^2 \left(
				   		- x_1 - x_2 - x_3
				   \right)
				   + \frac{ (1+p) x_2 }{ 2 } = 0 \\
				   & \Leftrightarrow 
				   \underbrace{ x_1 + \frac{ x_2 }{ 2 } }_{ c } 
				   + p \underbrace{ \left(
					   \frac{ - x_2 }{ 2 } 
				   \right) }_{ b }
					+ p ^2 
				   \underbrace{ 
					   \left(
							- x_1 - x_2 - x_3
					   \right)
				   }_{ a } 
				   = 0 \\
				   & \Rightarrow
				   p_{1,2} = \frac{ 
					   \frac{ x_2 }{ 2 } \pm \sqrt{
						   \frac{ x_2 ^2 }{ 4 } - 4 \left(
						   	- x_1 - x_2 - x_3
						   \right) \left(
						   	x_1 + \frac{ x_2 }{ 2 }
						   \right) 
					   }
					   }{ 
					   2 x_1 + x_2
				   } \\
				   & \Rightarrow
				   p_{1,2} = \frac{ 
					   \frac{ x_2 }{ 2 } \pm \sqrt{
						   \frac{ x_2 ^2 }{ 4 } + 4 \left(
							   x_1 ^2 + x_1 x_2 + x_1 x_3
							   \right) 
							   + 2
							   \left(
								   x_1 x_2 + x_2 ^2 + x_2 x_3
						   \right) 
					   }
					   }{ 
					   2 x_1 + x_2
				   } \\
				   & \Rightarrow
				   p_{1,2} = \frac{ 
					   \frac{ x_2 }{ 2 } \pm \sqrt{
						   (2 + \frac{ 1 }{ 4 }) x_2 ^2
						   + 4 x_1 ^2 + 6 x_1 x_2 
						   + 4 x_1 x_3 + 2 x_2 x_3
					   }
					   }{ 
					   2 x_1 + x_2
				   } \\
		\end{align*}
	\end{proof}
\end{document}
