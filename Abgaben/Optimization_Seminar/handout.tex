\documentclass[11pt,a4paper]{article}

\usepackage[margin=1in]{geometry}
\usepackage{amsmath,amsthm,amssymb, graphicx, multicol, array}

\usepackage{tikz}
\usetikzlibrary{automata, positioning}

\usepackage{amsmath}
\usepackage{amssymb}
\usepackage{amsthm}

\usepackage{mathrsfs}

\usepackage{color}
\usepackage{epsfig}

%--------------------------------------------------------------------------
% Maths Macros
%--------------------------------------------------------------------------
\def\Z{{\mathbb Z}}
\def\N{{\mathbb N}}
\def\Q{{\mathbb Q}}
\def\R{{\mathbb R}}
\def\C{{\mathbb C}}
\def\S{{\mathbb S}}
\def\K{{\mathbb K}}
\def\T{{\mathbb T}}

\def\cA{{\mathcal A}}
\def\cF{{\mathcal F}}
\def\cG{{\mathcal G}}
\def\cM{{\mathcal M}}
\def\cN{{\mathcal N}}
\def\cP{{\mathcal P}}
\def\cS{{\mathcal S}}

%
\def\sC{\mathscr{C}}

\newtheorem{algorithm}{Algorithm}[section]
\newtheorem{theorem}{Theorem}[section]
\newtheorem{proposition}{Proposition}[section]
\newtheorem{lemma}{Lemma}[section]

\theoremstyle{definition} % Makes my definitions non italic
\newtheorem{definition}{Definition}[section]

\def\qed{\hfill{\rule{1.5ex}{1.5ex}}}
\def\eop{\hfill{$\Box$}}

\begin{document}
\title{ \textbf{Handout - Automatic Differentiation} }
\author{Bent Mueller}
\date{7.6.2022}
\maketitle

\section{Overview - What is Auto-Diff?}

Given a \textit{twice differentiable} function $f: \R^n \to \R$, the
Auto-Diff Algorithm \textit{efficiently} computes gradient and Hessian
and achieves good \textit{numerical stability}:

\[
	\nabla f(x) := \begin{pmatrix}
		\frac{\partial f(x)}{\partial x_1} \\
		\frac{\partial f(x)}{\partial x_2} \\
		\vdots                             \\
		\frac{\partial f(x)}{\partial x_n}
	\end{pmatrix} ;
	\quad
	\nabla ^2 f(x) :=
	\begin{pmatrix}
		\frac{\partial^2 f(x)}{\partial x_1^2}            &
		\frac{\partial^2 f(x)}{\partial x_1 \partial x_2} &
		\vdots                                            &
		\frac{\partial^2 f(x)}{\partial x_1 \partial x_n}   \\
		\frac{\partial^2 f(x)}{\partial x_2 \partial x_1} &
		\frac{\partial^2 f(x)}{\partial x_2 \partial x_2} &
		\vdots                                            &
		\frac{\partial^2 f(x)}{\partial x_2 \partial x_n}   \\
		\vdots                                            &
		\vdots                                            &
		\vdots                                            &
		\vdots                                              \\
		\frac{\partial^2 f(x)}{\partial x_n \partial x_1} &
		\frac{\partial^2 f(x)}{\partial x_n \partial x_2} &
		\vdots                                            &
		\frac{\partial^2 f(x)}{\partial x_n \partial x_n}
	\end{pmatrix}
\]

\end{document}
