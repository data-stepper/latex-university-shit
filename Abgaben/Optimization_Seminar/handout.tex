\documentclass[11pt,a4paper]{article}

\usepackage[margin=1in]{geometry}
\usepackage{amsmath,amsthm,amssymb, graphicx, multicol, array}

\usepackage{tikz}
\usetikzlibrary{automata, positioning}

\usepackage{amsmath}
\usepackage{amssymb}
\usepackage{amsthm}
\usepackage{enumerate}

\usepackage{mathrsfs}

\usepackage{color}
\usepackage{epsfig}

%--------------------------------------------------------------------------
% Maths Macros
%--------------------------------------------------------------------------
\def\Z{{\mathbb Z}}
\def\N{{\mathbb N}}
\def\Q{{\mathbb Q}}
\def\R{{\mathbb R}}
\def\C{{\mathbb C}}
\def\S{{\mathbb S}}
\def\K{{\mathbb K}}
\def\T{{\mathbb T}}

\def\cA{{\mathcal A}}
\def\cF{{\mathcal F}}
\def\cG{{\mathcal G}}
\def\cM{{\mathcal M}}
\def\cN{{\mathcal N}}
\def\cP{{\mathcal P}}
\def\cS{{\mathcal S}}

%
\def\sC{\mathscr{C}}

\newtheorem{algorithm}{Algorithm}[section]
\newtheorem{theorem}{Theorem}[section]
\newtheorem{proposition}{Proposition}[section]
\newtheorem{lemma}{Lemma}[section]

\theoremstyle{definition} % Makes my definitions non italic
\newtheorem{definition}{Definition}[section]

\def\qed{\hfill{\rule{1.5ex}{1.5ex}}}
\def\eop{\hfill{$\Box$}}

\begin{document}
\title{ \textbf{Handout - Automatic Differentiation} }
\author{Bent Mueller}
\date{7.6.2022}
\maketitle

\section{Overview}

Given a \textit{twice differentiable} function $f: \R^n \to \R$, the
Auto-Diff Algorithm \textit{efficiently} computes gradient and Hessian
while achieving good \textit{numerical stability}:

\[
	\nabla f(x) := \begin{pmatrix}
		\frac{\partial f(x)}{\partial x_1} \\
		\frac{\partial f(x)}{\partial x_2} \\
		\vdots                             \\
		\frac{\partial f(x)}{\partial x_n}
	\end{pmatrix} ;
	\qquad
	\nabla ^2 f(x) :=
	\begin{pmatrix}
		\frac{\partial^2 f(x)}{\partial x_1^2}            &
		\frac{\partial^2 f(x)}{\partial x_1 \partial x_2} &
		\vdots                                            &
		\frac{\partial^2 f(x)}{\partial x_1 \partial x_n}   \\
		\frac{\partial^2 f(x)}{\partial x_2 \partial x_1} &
		\frac{\partial^2 f(x)}{\partial x_2 \partial x_2} &
		\vdots                                            &
		\frac{\partial^2 f(x)}{\partial x_2 \partial x_n}   \\
		\vdots                                            &
		\vdots                                            &
		\vdots                                            &
		\vdots                                              \\
		\frac{\partial^2 f(x)}{\partial x_n \partial x_1} &
		\frac{\partial^2 f(x)}{\partial x_n \partial x_2} &
		\vdots                                            &
		\frac{\partial^2 f(x)}{\partial x_n \partial x_n}
	\end{pmatrix}
\]

The Hessian is the \textit{Jacobian} of the gradient.

\subsection{Characterizing Sequence}

We only consider functions $f$ which can be decomposed into:

\begin{itemize}
	\item Constant functions $f(x) = c \in \R$, we say $f \in \mathcal{C}$
	\item Unary functions $f(x) \in \R$, we say $f \in \mathcal{U}$
	\item Binary functions $f(x_1, x_2) \in \R$, we say $f \in \mathcal{B}$
\end{itemize}

Note that a constant function takes no argument, a unary function takes one
and a binary function takes two arguments.
Thus we can decompose $f$ as follows:

% Char sequence for f(x)
\begin{enumerate}[(1)]
	\item $f_i = x_i$ for $i \in \{
		      1, \ldots, n
		      \} $
	\item $f_{i + n} = \begin{cases}
			      \omega_i                    & \text{ if } \omega_i \in \mathcal{C} \\
			      \omega_i (f_{k_i})          & \text{ if } \omega_i \in \mathcal{U} \\
			      \omega_i (f_{k_i}, f_{l_i}) & \text{ if } \omega_i \in \mathcal{B}
		      \end{cases}$
	      \qquad for $i \in \{
		      1, \ldots, m
		      \}$
	\item $f_{m + n} = f(x)$
	\item $k_i, l_i < i + n$  \text{ and }
	\item $\{
		      n+1, \ldots, n+m - 1
		      \} \subset \bigcup_{i=1}^{m} \{
		      k_i, l_i
		      \}$
\end{enumerate}

We define two sets of indices, $I := \{
	1, \ldots, m
	\} $ and $J := \{
	1, \ldots, n+m-1
	\} $.
Then we say that a sequence $S$ of tuples is a \textit{characterizing sequence}
for $f$ if and only if:

\[
	S = \left(
	(\omega_i, k_i, l_i)
	\right)_{i \in I}
	\in \left(
	\left(
		\mathcal{C} \times \{
		0
		\}^2
		\right) \cup
	\left(
		\mathcal{U} \times J \times \{
		0
		\} \right) \cup
	\left(
		\mathcal{B} \times J^2
		\right)
	\right) ^m
\]

Such that $S$ fulfills conditions 1-5 from the above.
$\Rightarrow$ $S$ computes $f(x)$ in $m$ steps.
Now we can use $S$ to also compute gradient and Hessian of $f$.

\subsection{Computing Gradient $\nabla f(x)$}

We define the sequence for the gradient as follows:

\[
	g_j = e_j \text{ if } j \in \{
	1, \ldots, n
	\}
	\text{ and }
	g_{i+n}
\]

\end{document}
