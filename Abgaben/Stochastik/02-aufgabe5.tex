\documentclass[a4paper]{article}

\usepackage[margin=1in]{geometry} 
\usepackage{amsmath,amsthm,amssymb, graphicx, multicol, array}

\usepackage{tikz}
\usetikzlibrary{automata, positioning}

\pdfminorversion=7
\pdfsuppresswarningpagegroup=1

\newcommand{\R}{\mathbb{R}}
\newcommand{\N}{\mathbb{N}}
\newcommand{\Z}{\mathbb{Z}}
\newcommand{\beh}{\textit{Behauptung. }}

\setlength{\parindent}{0pt}
\newenvironment{Aufgabe}[2][Aufgabe]{\begin{trivlist}
\item[\hskip \labelsep {\bfseries #1}\hskip \labelsep {\bfseries #2.}]}{\end{trivlist}}

\begin{document}
\maketitle
	\begin{Aufgabe}{5}
		Als erstes schauen wir uns Fall \textit{(ii)} an.
		\[
			\forall i, j \in \{
				1, ..., n
			\}:
			P(A_i \cap A_j) = \frac{ \# A }{ \# \Omega } =
			2^{-n} \cdot \# \{
				(w_1, ..., w_n) \in \Omega \; \vert \; 
				w_i = 1, w_j = 1
			\} = 2^{-n} \cdot 2^{n-2} = \frac{ 1 }{ 4 } \\
		\] 
		Es gilt aber auch:
		\[
			P(A_i) = 2^{-n} \cdot \# \{
				(w_1, ..., w_n) \in \Omega \; \vert \; 
				w_i = 1
			\} = 2^{-n} \cdot 2^{n-1} = \frac{ 1 }{ 2 }
			\qquad (= P(A_j))
		\] 
		Und $P(A_i \cap A_j) = \frac{ 1 }{ 4 } = \frac{ 1 }{ 2 } \cdot \frac{ 1 }{ 2 } = P(A_i) P(A_j)
		\Rightarrow A_i, A_j$ sind stochastisch unabhaengig.
		Da jeweils aber immer gilt $i\neq j$ koennen wir das fuer beliebige Teilmengen machen. Auch bei
		allen anderen Teilmengen $I \subset \{
			1, ..., n
		\} $ gilt dann immer die Unabhaengigkeit, da die Indize immer disjunkt sind.
		Insbesondere gilt auch:
		\[
			P\left(
				\bigcap_{i=1}^{n-1} A_i \cap A_n
			\right) = 2 ^{-n} \cdot \# \{
				(w_1, ..., w_n) \in \Omega \; \vert \;
				w_1 = 1, ..., w_n = 1
			\} = 2^{-n}
		\] 
		\[
			P\left(
				\bigcap_{i=1}^{n-1} A_i
			\right) = 2 ^{-n} \cdot \# \{
				(w_1, ..., w_n) \in \Omega \; \vert \;
				w_1 = 1, ..., w_{n-1} = 1
			\} = 2 ^{1-n}
		\] 
		\[
		P\left(
			A_n
		\right) = 2 ^{-n} \cdot \# \{
			(w_1, ..., w_n) \in \Omega \; \vert \;
			w_n = 1
		\} = 2^{-1}
		\] 
		Wir sehen also auch bei der gesamten Indexmenge die Unabhaengigkeit.
		Es folgt also, dass $\{
			A_1, ..., A_n
		\} $ stochastisch unabhaengig ist.
		\\
		
		Aehnlich koennen wir im Fall \textit{(iii)} vorgehen:
		\begin{align*}
			P\left(
				A_j \cap A_{n+1}
				\right) &= 2^{-n} \cdot \# \{
				(w_1, ..., w_n) \in \Omega \; \vert \;
				w_j = 1, w_1 + \cdots + w_n \text{ ungerade }
			\} \\
				&= 2^{-n} \cdot \# \{
					(w_1, ..., w_n) \in \Omega \; \vert \;
					w_1 + \cdots w_{j-1} + w_{j+1} + \cdots + w_n
					\text{ gerade }
				\} = 2^{-n} \cdot 2^{n-2} = \frac{ 1 }{ 4 }
		\end{align*}
		Im obigen verwenden wir den Hinweis, aber mit einer $n-1$-elementigen Teilmenge (da ja $w_j$ nicht
		Teil der Summe ist). Wie bei \textit{(ii)} koennen wir beliebige Indexmengen hier vereinigen
		und wir schauen auch, dass im Fall der gesamten Indexmenge klappt:
		\begin{align*}
			P\left(
				\bigcap_{i=2}^n A_i \cap A_{n+1}
			\right) &= 2^{-n} \cdot \# \{
				(w_2, ..., w_n) \in \Omega \; \vert \;
				w_2 = 1, ..., w_n = 1, w_1 + \cdots + w_n \text{ ungerade}
			\} = 2 ^{-n} \\
			P\left(
				\bigcap_{i=2}^n A_i
		\right) &= 2^{1-n} \\
			P \left(
				A_{n+1}
			\right) &= 2 ^{-n} \cdot \# \{
					(w_1, ..., w_n) \in \Omega \; \vert \;
					w_1 + \cdots + w_n \text{ ungerade}
				\} = 2^{-n} \cdot 2^{n-1} = \frac{ 1 }{ 2 }\\
					& \frac{ 1 }{ 2 } \cdot 2^{1-n} = 2^{-n}
					\Rightarrow \text{Die Familie in \textit{(iii)} ist stochastisch unabhaengig.}
		\end{align*}
		Nun faellt uns aber auf, dass genau der letzte Schritt im Fall von \textit{(i)} schief geht.
		Sind naemlich alle $w_i = 1$ dann gibt es keine Moeglichkeit mehr, dass die Summe ungerade ist.
		Dann gilt also:
		\[
			P\left(
				\bigcap_{i=1}^n A_i \cap A_{n+1}
			\right) = 0 \neq P\left(
				\bigcap_{i=1}^n A_i
			\right) \cdot P \left(
				A_{n+1}
			\right) > 0
		\] 
		Also ist die Familie im Fall \textit{(i)} \textbf{nicht} stochastisch unabhaengig.
	\end{Aufgabe}
\end{document}
