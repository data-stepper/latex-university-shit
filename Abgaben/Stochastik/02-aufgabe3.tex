\documentclass[a4paper]{article}

\usepackage[margin=1in]{geometry} 
\usepackage{amsmath,amsthm,amssymb, graphicx, multicol, array}

\usepackage{tikz}
\usetikzlibrary{automata, positioning}

\pdfminorversion=7
\pdfsuppresswarningpagegroup=1

\newcommand{\R}{\mathbb{R}}
\newcommand{\N}{\mathbb{N}}
\newcommand{\Z}{\mathbb{Z}}
\newcommand{\beh}{\textit{Behauptung. }}

\setlength{\parindent}{0pt}
\newenvironment{Aufgabe}[2][Aufgabe]{\begin{trivlist}
\item[\hskip \labelsep {\bfseries #1}\hskip \labelsep {\bfseries #2.}]}{\end{trivlist}}

\begin{document}
\maketitle
	\begin{Aufgabe}{3}
	\end{Aufgabe}

	\begin{itemize}
		\item[a)] Nein, denn für genau dieses Paar gilt ja dann:
			\[
				0 = P(A_i \cap A_j) \neq P(A_i) \cdot P(A_j) > 0
			\] 
			Insbesondere ist dann die Indexteilemenge $I = \{
				i, j
			\} \subset \{
				1, ..., n
			\} $ für welche die Unabhängigkeitsaussage nicht gilt.

		\item[b)] {
				\begin{enumerate}
					\item[(i)] {
							\begin{align*}
								P \left(
									\bigcup_{i=1}^n A_i
								\right) &= 
								P \left(
									\left(
										\bigcap_{i=1}^n A_i ^{c}
									\right) 
									^{c}
								\right) =
								1 - P \left(
									\bigcap_{i=1}^n A_i ^{c}
								\right) \\
								&= 1 - \prod_{i=1}^{n} P(A_i ^{c}) \\
								&= 1 - \prod_{i=1}^{n} \left(
									1 - P(A_i)
								\right)
							\end{align*}
						}
				\end{enumerate}
			}
	\end{itemize}

\end{document}
