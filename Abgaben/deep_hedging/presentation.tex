\documentclass{beamer}

% Information to be included in the title page:

\title{
    Deep Hedging
}

\author{Bent Müller}
\institute{University of Hamburg, Department of Mathematics}
\date{12.6.2023}

% Standard beamer class setup, configure as needed

\usepackage{tikz}
\usetikzlibrary{automata, positioning}

\setbeamertemplate{headline}[default]
\setbeamertemplate{navigation symbols}{}
\mode<beamer>{\setbeamertemplate{blocks}[rounded][shadow=true]}
\setbeamercovered{transparent}
\setbeamercolor{block body example}{fg=blue, bg=black!20}

\useoutertheme[subsection=true]{miniframes}
\usetheme{Frankfurt}

\def\Z{{\mathbb Z}}
\def\N{{\mathbb N}}
\def\Q{{\mathbb Q}}
\def\R{{\mathbb R}}
\def\C{{\mathbb C}}
\def\S{{\mathbb S}}
\def\K{{\mathbb K}}
\def\T{{\mathbb T}}

\def\cA{{\mathcal A}}
\def\cF{{\mathcal F}}
\def\cG{{\mathcal G}}
\def\cM{{\mathcal M}}
\def\cN{{\mathcal N}}
\def\cP{{\mathcal P}}
\def\cS{{\mathcal S}}

\begin{document}

\begin{frame}
    \titlepage
\end{frame}

\section{Introduction}
\subsection{Table of Contents}
\begin{frame}
    \tableofcontents
\end{frame}

\subsection{General problem layout}

\begin{frame}
    \frametitle{What is ``Deep Hedging''?}

    Given a function $f: \R^{n} \longrightarrow \R$ that is
    \textit{twice differentiable}, we want to
    \textit{efficiently}
    and with good \textit{numerical stability} compute
    \vspace{5mm}

    \begin{itemize}
        \item $f(x)$, the value of our function at a point $x$ in $\R^{n}$,
        \item $\nabla f(x)$, the gradient of our function at $x$, and
        \item $\nabla^2 f(x)$, the Hessian of our function at $x$.
    \end{itemize}

    \vspace{5mm}
    Where $\nabla$ is the \textit{Differential Operator}, also called
    the \textit{nabla} Operator or sometimes just \textit{Del}.
\end{frame}

\section{}

\end{document}
