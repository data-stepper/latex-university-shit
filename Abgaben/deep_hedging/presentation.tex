\documentclass[serif]{beamer}

% Information to be included in the title page:

\title{Deep Hedging}

\author{Bent Müller}
\institute{University of Hamburg, Department of Mathematics}
\date{12.6.2023}

% Standard beamer class setup, configure as needed

\usepackage{tikz}
\usetikzlibrary{automata, positioning}

\setbeamertemplate{headline}[default]
\setbeamertemplate{caption}[numbered]
\setbeamertemplate{navigation symbols}{}
\mode<beamer>{\setbeamertemplate{blocks}[rounded][shadow=true]}
\setbeamercovered{transparent}
\setbeamercolor{block body example}{fg=blue, bg=black!20}

\useoutertheme[subsection=true]{miniframes}
\usetheme{Frankfurt}

% Definition of new commands
\def\R{{\mathbb R}}
\def\P{{\mathbb P}}
\def\E{{\mathbb E}}
\def\N{{\mathbb N}}
\def\O{{\Omega}}

\def\cF{{\mathcal F}}
\def\x{{\times}}

\def\cNN{{\mathcal N \mathcal N}}
\def\cB{{\mathcal B}}
\def\cP{{\mathcal P}}
\def\cG{{\mathcal G}}
\def\cX{{\mathcal X}}
\def\cY{{\mathcal Y}}

\def\BM{{\text{BM}}}
\def\RN{{\text{RN}}}
\def\Var{{\text{Var}}}
\def\Cov{{\text{Cov}}}
\def\riskNeutralQ{{\mathbb Q}}
\def\F{{\mathbb F}}

\def\vs{{\vspace{0.5cm}}}

% These are specifically for deep hedging
\def\Hu{{\mathcal H^u}}
\def\H{{\mathcal H}}
\def\L{{\mathcal L}}

% New additions for complete market model
\def\ps{{(\O, \cF, \P)}}
\def\fm{{(\O, \cF, \P, \F, S)}}

\begin{document}

\begin{frame}
    \titlepage
    \footnote{
        Original Paper: \cite[Deep Hedging]{buehler2018deep}
    }
\end{frame}

% TOC I sent to supervisor
% 1. The general problem of hedging a portfolio of derivatives (recap)
% 2. Setup in the paper
% 2.1 Different (convex) risk measures
% 2.2 From risk measures to hedging strategies
% 3. Neural Networks Introduction
% 3.1 Universal Approximation Theorem
% 3.2 Learning Optimal Hedging Strategies
% 3.3 Neural Network Architectures (recurrent and simple feed forward)
% 4. Numerical Results from the Paper
% 4.1 Hedging in the Heston Model
% 4.2 Hedging in a Multi-Asset Market (underlying + variance swap)
% 4.3 Choosing risk measures to optimize for

\section{Table of Contents}

\begin{frame}
    \frametitle{Table of Contents}
    \tableofcontents
\end{frame}

\section{Hedging}

\subsection{Recap from the Lecture}
\begin{frame}
    \frametitle{Notation and Setup}
    \begin{itemize}
        \only<1>{
        \item $\O := \{ \omega_1, \omega_2, \dots, \omega_N \}$ is our \textbf{discrete} set of outcomes.
        \item $\cF := 2^\O$ the $\sigma$-algebra of all subsets of $\O$, so that is
              $\fm$ our financial market.
        \item $\cX := \{X: \O \to \R\}$ is the set of all real-valued random variables.
        \item $\rho: \cX \to \R$ is a risk measure.
              }
              \only<2>{
        \item $l: \R \to \R$ \textbf{continuous, convex and non-decreasing} is called a loss function.
        \item $\rho_l: \cX \to \R$, where $\rho_l (X) := \inf_{w \in \R} \{ w + \E [ l (-X -w) ] \}$ defines
              a convex risk measure, a so-called \textbf{Optimized Certainty Equivalent} (OCE) risk measure
              (Lemma 3.16).
        \item $S^i_t : \O \to \R_{\geq 0}$
              is the $\cF_t$-measurable price of the $i$-th risky asset
              at time $t$
              for $0 \leq t \leq T$ and $0 \leq i \leq d$.
              }
              \only<3>{
        \item $\Hu := \{ \phi : \O \to \R^{d+1} \; | \; \forall 0 \leq t \leq T:
                  \phi_{t+1} \text{ is } \cF_t \text{-measurable and } \phi_{-1} = \phi_T = 0 \} $ is the set of
              unconstrained hedging strategies.
        \item $\H \subset \Hu$ is the set of \textbf{admissible} hedging strategies which
              are constrained by e.g. market hours or emittance of certain hedging instruments.
        \item For simplicity, let $r=0$.
              }
    \end{itemize}
\end{frame}

\subsection{Hedging under Transaction Costs and Market Frictions}

\begin{frame}[t]
    \frametitle{Trading with Transaction Costs}
    Let $\delta \in \H$ be a hedging strategy, then we define
    $$C_T (\delta) := \sum_{t=0}^T c_t (\delta_t - \delta_{t-1})$$
    as the \textbf{cumulative transaction cost} of
    trading using $\delta$ up to time $T$.
    $c_t : \R^{d+1} \to \R^+$ is a non-negative \textbf{adapted cost process}
    satisfying $\forall t: c_t (0) = 0$.
    \vs

    This makes different transaction costs possible, like:
    \begin{itemize}
        \only<1>{
        \item Proportional Transaction Costs:
              $$c_t (n) = \sum_{i=0}^d c_t^i S_t^i | n^i |$$
              for a transaction $n \in \R^{d+1}$.
              }
              \only<2>{
        \item Fixed Transaction Costs:
              $$c_t (n) = \sum_{i=0}^d c_t^i 1_{|n^i| > 0}$$
              for a transaction $n \in \R^{d+1}$.
              }
              \only<3>{
        \item Or even volatility dependent transaction costs which we won't consider here.
              In this case $c_t$ might increase with high volatility.
              }
              \only<4>{
        \item If the agent is allowed to trade in other options to hedge $-Z$,
              there might be additional costs for high volatility.
              This is called the \textbf{cost of volatility} and can be modeled
              if we allow $c_t$ to depend on the Black \& Scholes \textbf{Vega}
              of each traded asset.
              }
    \end{itemize}
\end{frame}

\begin{frame}
    \frametitle{Fundamental Problem}
    A hedging agent then wishes to achieve
    the optimal hedge for a given liability $-Z$.
    \[
        \pi (-Z) := \inf_{\delta \in \H} \rho (
        -Z + (\delta \cdot S)_T - C_T (\delta)
        )
    \]
    $\pi (-Z)$ is the risk of the \textbf{optimal hedge} possible when hedging
    using strategies in $\H$.
    Then
    $$p(Z) = \pi (-Z) - \pi (0)$$
    is the \textbf{indifference price} at which the agent would be
    indifferent between holding $-Z$ and not holding any position.
\end{frame}

\begin{frame}
    \frametitle{Fundamental Problem}
    At time $T$, the agent's terminal wealth is given by
    $$p_0 - Z + (\delta \cdot S)_T - C_T (\delta)$$
    where $p_0$ is the amount she received for selling $Z$ in the first place. \\
    \vs
    Note that $p_0$ may also be given externally and must not be the
    indifference price $p(Z)$.
\end{frame}

\subsection{Conditional Value at Risk}

\begin{frame}
    \frametitle{Conditional Value at Risk}
    \begin{definition}[CVaR and VaR]
        For any $X \in \L^1 (\O, \cF, \P)$ and $\alpha \in (0,1)$,
        the \textbf{Conditional Value at Risk} (CVaR)
        is defined as
        $$\text{CVaR}_\alpha (X) := \frac{1}{\alpha} \int_0^\alpha VaR_u (X) du$$
        where $VaR_\lambda (X)$ is the \textbf{Value at Risk} at level $\lambda$ defined as
        $$VaR_\lambda (X) := - \inf \{ x \in \R \; | \; \P (X \leq x) \geq \lambda \}.$$
    \end{definition}
\end{frame}

\begin{frame}
    \frametitle{Conditional Value at Risk of S\&P 500 Daily Returns}
    \begin{figure}
        \includegraphics[width=1.0\textwidth]{./images/cvar_sp500_example.png}
        \caption{Conditional Value at Risk of S\&P 500 Daily Returns since 1993-02-01;
            mirrored on the $0$ for representability
        }
    \end{figure}
\end{frame}

\section{Neural Networks}

\begin{frame}
    \frametitle{Feed Forward Networks}
    For $d_i, d_h, d_o \in \N$
    and activation function $\sigma$,
    we define a single hidden layer \textbf{feed forward network} (FFN) as
    \[
        \text{FFN}_{W_2, b_2, W_1, b_1}^{d_i, d_h, d_o} : \R^{d_i} \to \R^{d_o}, \;
        x \mapsto W_2 \cdot \sigma (W_1 \cdot x + b_1) + b_2.
    \]
    Where
    \begin{itemize}
        \item $d_i, d_h, d_o$ are the dimensions of the input, hidden and output layer respectively,
        \item $W_1 \in \R^{d_h \times d_i}, W_2 \in \R^{d_o \times d_h}$ are weight matrices and
        \item $b_1 \in \R^{d_h}, b_2 \in \R^{d_o}$ are bias vectors.
    \end{itemize}
    We shorten the parameters and write $\text{FFN}_\theta^{d_i, d_h, d_o}$.
\end{frame}

\begin{frame}
    \frametitle{Common Activation Functions}
    \begin{figure}
        \includegraphics[width=0.8\textwidth]{./images/activation_functions.png}
        \caption{Common Activation Functions}
    \end{figure}
\end{frame}

\subsection{Universal Approximation Theorem}

\begin{frame}[t]
    \frametitle{Universal Approximation Theorem}
    Let $\sigma$ be non-constant and bounded, $d \in \N$
    and $\cNN_\infty := \{ \text{FFN}_{\theta}^{d, L, 1} \; | \; L \in \N \}$
    be the set of FFNs with arbitrary hidden layer size.
    Then we have: \\ \vs
    \begin{enumerate}
        \only<1->{
        \item For any finite measure $\mu$ on $(\R^{d}, \cB(\R^{d}))$
              and $p \in [1, \infty)$, the set $\cNN_\infty$ is dense in
              $L^p (\R^{d}, \mu)$.
              }
              \only<2->{
        \item If further $\sigma$ is continuous
              then $\cNN_\infty$ is dense in $C(\R^{d})$ in the sense that
              for any $f \in C(\R^{d})$ there exists a sequence of
              neural networks $\varphi_n \in \cNN_\infty$ that converges
              uniformly on any compact set $K \subset \R^{d}$ to $f$.
              }
    \end{enumerate}
    \only<2>{
        \vs The Universal Approximation Theorem goes back to
        \cite{nn_function_approximators}.
    }
    \only<3>{
        \vs Therefore, we may approximate any $\delta \in \H$
        arbitrarily well using a FFN.
    }
\end{frame}

\section{Deep Hedging}

\subsection{Heston Model}

\begin{frame}
    \frametitle{Heston Model}
    \begin{definition}[Heston Model]
        \only<1>{
            In a Heston model, the dynamics of the underlying risky asset $S$ are given
            by
        }
        \begin{align*}
            dS_t & = \mu S_t dt + \sqrt{V_t} S_t dB_t             \\
            dV_t & = \alpha (b - V_t) dt + \sigma \sqrt{V_t} dW_t
        \end{align*}
        % TODO: Add here where the other parameters come from
        \only<1>{
            where $B, W$ are two correlated Brownian motions with
            correlation $\rho \in [-1, 1]$.
        }
        \only<2>{
            \vs
            In our case, we are modeling a `typical equity market' using
            $$S_0 = 100, \alpha = 1, b = V_0 = 0.04, \rho = -0.7, \mu = 0 \text{ and } \sigma = 2.$$
        }
    \end{definition}
\end{frame}

\begin{frame}[t]
    \frametitle{GBM vs Heston}
    \begin{figure}[t]
        \centering
        \includegraphics[width=1.0\textwidth]{./images/gbm_heston_example.png}
        \caption{
            Simulated stock prices by GBM and Heston as well as the real S\&P 500;
            Below each price we can see the corresponding (14 days rolling) volatility.
        }
    \end{figure}
\end{frame}

\subsection{Deep Hedging as Presented in The Paper}

\begin{frame}
    \frametitle{Idealized Variance Swap}
    In the paper, the hedging agent is allowed to trade in
    \begin{itemize}
        \item $S^{(1)}_t$, the underlying risky asset and
        \item an idealized variance swap which is defined as
              \begin{align*}
                  S^{(2)}_t & := \E_{\mathbb{Q}} \Big [
                      \int_0^T V_s \; ds \; \Big | \; \cF_t
                  \Big ]                                \\
                            & = \int_0^t V_s \; ds
                  + \underbrace{
                      \frac{V_t - b}{\alpha} (
                      1 - e^{-\alpha(T-t)}
                      ) + b(T-t)
                  }_{:= L(t, V_t)}.
              \end{align*}
    \end{itemize}
\end{frame}

\begin{frame}
    \frametitle{How this looks like}
    \begin{figure}
        \includegraphics[width=0.7\textwidth]{./images/vol_swap_payoff.png}
        \caption{
            Histograms from $10^7$ simulations
            of $S^{(1)}_T$ and $S^{(2)}_T$ in the Heston model
            for $T=30$
            and parameters as in the paper
        }
    \end{figure}
\end{frame}

\begin{frame}
    \frametitle{How this looks like}
    \begin{figure}
        \includegraphics[width=0.8\textwidth]{./images/vol_swap_scatter.png}
        \caption{
            Scatter plot showing how $S^{(1)}_T$ and volatility swap
            payoff $S^{(2)}_T$ (or log thereof in the bottom plot) interact
        }
    \end{figure}
\end{frame}

\begin{frame}
    \frametitle{How this looks like}
    \begin{figure}
        \includegraphics[width=0.7\textwidth]{./images/call_put_payoff.png}
        \caption{
            Expected Payoffs from Calls and Puts at $T=30$ with strikes
            on the x-Axis; Assuming $r=0$, these are the `fair' option prices
        }
    \end{figure}
\end{frame}

\begin{frame}
    \frametitle{How this looks like}
    \begin{figure}
        \includegraphics[width=0.7\textwidth]{./images/var_swap_price1.png}
        \caption{
            Price of a variance swap throughout a hedging period
        }
    \end{figure}
\end{frame}

\begin{frame}
    \frametitle{How this looks like}
    \begin{figure}
        \includegraphics[width=0.7\textwidth]{./images/var_swap_price.png}
        \caption{
            Price of a variance swap throughout a hedging period
        }
    \end{figure}
\end{frame}

\subsection{Hedging using Deep Neural Networks}

\begin{frame}
    \frametitle{Learning Hedging Strategies}
    \begin{definition}[Information Set]
        Let $I_t$ be the $\F$-adapted information
        set with all the information we want to provide our
        hedging agent with. \\ \vs
        In our case, let
        \[
            I_t := (S^{(1)}_t, S^{(2)}_t)
        \]
        since our market model is \emph{markovian}.
    \end{definition}
    \only<2>{
        We may also allow additional information to be included
        in $I_{t-1}$, like upcoming earnings or market sentiment
        if such information can be built into our model somehow.
    }
    \only<3>{
        When hedging in a \emph{non-markovian} market, we might want to define
        \[
            \tilde{I}_t := (
            S^{(1)}_t, S^{(2)}_t, S^{(1)}_{t-1}, S^{(2)}_{t-1}, \dots, S^{(1)}_0, S^{(2)}_0
            ).
        \]
    }
\end{frame}

\begin{frame}
    \frametitle{Learning Hedging Strategies}
    \begin{definition}[Deep Hedging Strategy]
        For $L \in \N, d \in \N$, $\sigma: \R \to \R$
        an activation function,
        $I_t$ our information set at time $t$ and
        $\text{FFN}_\theta^{d+k, L, d} : \R^{d+k} \to \R^d$,
        a feed forward neural network, we define
        our \textbf{deep hedging strategy} as
        \begin{align*}
            \delta^\theta_t:
            \;\; & \R^{d+k} \to \R^d                                   \\
            I_{t-1} \mapsto \delta^\theta_t (I_{t-1})
                 & := \text{FFN}_\theta^{d+k, L, d} (I_{t-1})          \\
                 & = W_2 \cdot \sigma (W_1 \cdot I_{t-1} + b_1) + b_2.
        \end{align*}
    \end{definition}
\end{frame}

\begin{frame}
    \frametitle{Learning Hedging Strategies}
    \only<1>{
        Observe how our previous optimization problem
        \begin{equation*}
            \pi (-Z) := \inf_{\delta \in \H} \rho (
            -Z + (\delta \cdot S)_T - C_T (\delta)
            )
        \end{equation*}
        becomes, for $\delta^\theta \in \H$,
    }
    \begin{equation*}
        \pi (-Z) := \inf_{\theta \in \Theta} \rho (
        -Z + (\delta^\theta \cdot S)_T - C_T (\delta^\theta)
        )
    \end{equation*}
    \only<1>{
        just the problem of finding the best parameters $\theta$ for our
        neural network.
    }
    \only<2>{
        Further, if $\rho$ and $c_t$ are differentiable, we can compute
        $$\nabla_\theta
            \rho (
            -Z + (\delta^\theta \cdot S)_T - C_T (\delta^\theta)
            )
        $$
        in order to update our parameters $\theta$.
    }
\end{frame}

\begin{frame}
    \frametitle{The case for OCE Risk Measures}
    If $\rho = \rho_l$ is an OCE risk measure, then the problem
    becomes:
    \begin{align}
        \pi (-Z) & = \inf_{\theta \in \Theta} \inf_{w \in \R}
        w + \E [ l (
            Z - (\delta^\theta \cdot S)_T + C_T (\delta^\theta)
            -w
        ) ]                                                                      \\
                 & = \inf_{\tilde{\theta} \in \tilde{\Theta}}
        w + \E [ l (
            Z - (\delta^\theta \cdot S)_T + C_T (\delta^\theta)
            -w
        ) ]                                                                      \\
                 & = \inf_{\tilde{\theta} \in \tilde{\Theta}} J (\tilde{\theta})
    \end{align}
    \only<1>{
        For $\tilde{\Theta} := \Theta \x \R$
        and of course $\tilde{\theta} := (\theta, w)$.
    }
    \only<2>{
        We can now compute the gradient of $J$ with respect to $\tilde{\theta}$
        and use stochastic gradient descent to find the optimal parameters $\hat{\theta}$.
    }
    \only<3>{
        Specifically, we sample
        $\omega_1, \ldots, \omega_{N_{\text{batch}}} \in \O$,
        estimate
        \[
            \tilde{J} (\tilde{\theta}) = w +
            \sum_m^{N_{\text{batch}}} l (
            Z(\omega_m) - (\delta^{\theta} \cdot S)_T (\omega_m)
            + C_T (\delta^{\theta}) (\omega_m)
            - w
            ) \frac{N \cdot
                \P (\{ \omega_m \})
            }{N_{\text{batch}}}
        \]
        from which we immediately get
        $\nabla_{\tilde{\theta}} \tilde{J} (\tilde{\theta})$
        using autodiff.
    }
\end{frame}

\subsection{Numerical Results from the Paper}

\begin{frame}
    \frametitle{Neural Network Architecture in the Paper}
    \begin{figure}
        \includegraphics[height=0.7\textheight]{./images/neural_net_screenshot.png}
        \caption{
            (Recurrent) Neural Network Architecture used in the paper
        }
    \end{figure}
\end{frame}

\begin{frame}
    \frametitle{Learned and Model Delta}
    \begin{figure}
        \includegraphics[width=0.8\textwidth]{./images/model_delta_screenshot.png}
        \caption{
            Learned $\delta^{\theta, 1}$ (the position in the underlying)
            and Heston model delta $\delta$
            as a function of $(S^{(1)}_t, V_t)$ for $t=15$ days.
        }
    \end{figure}
\end{frame}

\subsection{Wrapping Up}

\begin{frame}
    \frametitle{Key Takeaways}
    \begin{itemize}
        \item Deep hedging is a \textbf{model-independent} approach to hedging
              which can be applied to any market model.
        \item It is \textbf{computationally efficient} and can be used to
              hedge in \textbf{high-dimensional} (i.e. multi-asset) markets.
        \item It is \textbf{flexible} in the sense that we can choose
              \begin{itemize}
                  \item the risk measure we want to optimize for,
                  \item transaction cost and trading restrictions,
                  \item the information set we provide to the agent and
                  \item the hedging instruments we allow trading in.
              \end{itemize}
        \item The approach is \textbf{easy to adapt} to more sophisticated
              derivatives or to the hedging of multiple liabilities
              simultaneously.
    \end{itemize}
\end{frame}

\begin{frame}
    \frametitle{Criticism of the Paper}
    \begin{itemize}
        \item Optimization via
              $\nabla_{\tilde{\theta}} (\tilde{J}) (\tilde{\theta})$
              directly is actually really hard and rather primitive
              compared to more modern reinforcement learning approaches.
    \end{itemize}
\end{frame}

\begin{frame}
    \centering
    \Large{
        Thank you for your attention!
    }
\end{frame}

% And make the references slide using bibtex
\section{References}
\begin{frame}
    \frametitle{References}
    \bibliographystyle{apalike}
    \bibliography{references}
\end{frame}

\end{document}
