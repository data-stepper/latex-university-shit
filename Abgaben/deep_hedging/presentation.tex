\documentclass[serif]{beamer}

% Information to be included in the title page:

\title{Deep Hedging}

\author{Bent Müller}
\institute{University of Hamburg, Department of Mathematics}
\date{12.6.2023}

% Standard beamer class setup, configure as needed

\usepackage{tikz}
\usetikzlibrary{automata, positioning}

\setbeamertemplate{headline}[default]
\setbeamertemplate{navigation symbols}{}
\mode<beamer>{\setbeamertemplate{blocks}[rounded][shadow=true]}
\setbeamercovered{transparent}
\setbeamercolor{block body example}{fg=blue, bg=black!20}

\useoutertheme[subsection=true]{miniframes}
\usetheme{Frankfurt}

% Definition of new commands
\def\R{{\mathbb R}}
\def\P{{\mathbb P}}
\def\E{{\mathbb E}}
\def\N{{\mathbb N}}
\def\O{{\Omega}}

\def\cF{{\mathcal F}}

\def\cP{{\mathcal P}}
\def\cG{{\mathcal G}}
\def\cX{{\mathcal X}}
\def\cY{{\mathcal Y}}

\def\BM{{\text{BM}}}
\def\RN{{\text{RN}}}
\def\Var{{\text{Var}}}
\def\Cov{{\text{Cov}}}
\def\riskNeutralQ{{\mathbb Q}}
\def\F{{\mathbb F}}

\def\vs{{\vspace{0.5cm}}}

% These are specifically for deep hedging
\def\Hu{{\mathcal H^u}}

% New additions for complete market model
\def\ps{{(\O, \cF, \P)}}
\def\fm{{(\O, \cF, \F, X, \P)}}

\begin{document}

\begin{frame}
    \titlepage
    \footnote{
        Original Paper: \cite[Deep Hedging]{bühler2018deep}
    }
\end{frame}

% TOC I sent to supervisor
% 1. The general problem of hedging a portfolio of derivatives (recap)
% 2. Setup in the paper
% 2.1 Different (convex) risk measures
% 2.2 From risk measures to hedging strategies
% 3. Neural Networks Introduction
% 3.1 Universal Approximation Theorem
% 3.2 Learning Optimal Hedging Strategies
% 3.3 Neural Network Architectures (recurrent and simple feed forward)
% 4. Numerical Results from the Paper
% 4.1 Hedging in the Heston Model
% 4.2 Hedging in a Multi-Asset Market (underlying + variance swap)
% 4.3 Choosing risk measures to optimize for

\section{Table of Contents}
\begin{frame}
    \tableofcontents
\end{frame}

\section{Hedging a Portfolio of Derivatives}

\subsection{Recap from the Lecture}
\begin{frame}
    \frametitle{Notation}
    \begin{itemize}
        \item $\O := \{ \omega_1, \omega_2, \dots, \omega_N \}$ is our \textbf{discrete} set of outcomes.
        \item $\cF := 2^\O$ the $\sigma$-algebra of all subsets of $\O$, so that $\ps$ is our model probability space.
        \item $\cX := \{X: \O \to \R\}$ is the set of all real-valued random variables.
        \item $\rho: \cX \to \R$ is a (possibly convex) risk measure.
        \item $l: \R \to \R$ \textbf{continuous, convex and non-decreasing} is called a loss function.
        \item $\rho_l: \cX \to \R$, where $\rho_l (X) := \inf_{w \in \R} \{ w + \E [ l (-X -w) ] \}$ defines
              a convex risk measure, a so-called \textbf{Optimized Certainty Equivalent} (OCE) risk measure
              (Lemma 3.16).
    \end{itemize}
\end{frame}

\begin{frame}[t]
    \frametitle{Trading with Transaction Costs}
    For $\delta$ an arbitrary trading strategy, we define
    $$C_T (\delta) := \sum_{t=0}^T c_t (\delta_t - \delta_{t-1})$$
    as the \textbf{cumulative cost} of
    trading using $\delta$ up to time $T$.
    $c_t : \R^+ \to \R^+$ is non-negative adapted cost function satisfying $\forall t: c_t (0) = 0$.
    \vs

    This makes different transaction costs possible, e.g.:
    \begin{itemize}
        \item Proportional Transaction Costs:
    \end{itemize}
\end{frame}

\begin{frame}
    \frametitle{Fundamental Problem}
    Let $\fm$ be a financial market, $Z: \O \mapsto \R_{\geq 0}$ a contingent claim on an asset
    $S_t: \O \mapsto \R$ with a payout at time $T$ and
    $\rho: \cX \mapsto \R$ be a given risk measure.
    A hedging agent then wishes to minimize
    \[
        \pi (Z) := \rho (
        Z - (\delta \cdot S)_T - C_T (\delta)
        )
    \]
\end{frame}

\subsection{Conditional Value at Risk}

\subsection{Constrained Hedging Strategies}
\subsection{Hedging under Transaction Costs and Market Frictions}
\subsection{Universal Approximation Theorem}

\section{Introduction to Neural Networks}
\subsection{Universal Approximation Theorem}

\section{Deep Hedging}

\subsection{Why Use Neural Networks?}
\subsection{Deep Hedging as Presented in The Paper}
\subsection{Pricing using Deep Hedging}

% And make the references slide using bibtex
\section{References}
\begin{frame}
    \frametitle{References}
    \bibliographystyle{apalike}
    \bibliography{references}
\end{frame}

\end{document}
