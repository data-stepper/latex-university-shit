\documentclass[a4paper]{article}

\usepackage[margin=1in]{geometry} 
\usepackage{amsmath,amsthm,amssymb, graphicx, multicol, array}

\usepackage{tikz}
\usetikzlibrary{automata, positioning}

\pdfminorversion=7
\pdfsuppresswarningpagegroup=1

\newcommand{\R}{\mathbb{R}}
\newcommand{\N}{\mathbb{N}}
\newcommand{\Z}{\mathbb{Z}}
\newcommand{\beh}{\textit{Behauptung. }}

\setlength{\parindent}{0pt}
\newenvironment{Aufgabe}[2][Aufgabe]{\begin{trivlist}
\item[\hskip \labelsep {\bfseries #1}\hskip \labelsep {\bfseries #2.}]}{\end{trivlist}}

\begin{document}
\title{ \textbf{Praesenzaufgaben} }
\author{Bent Müller}
\date{22. Juli 2021}
\maketitle

\section*{Blatt 1}

\textbf{(1)} Dichte der Binomialverteilung:

\begin{align*}
	P(X = k) &= \begin{pmatrix} n \\ k \end{pmatrix} 
	\cdot p ^{k} (1-p) ^{n-k} \\
			 &= \exp \left(
				 \log \left(
					 \frac{ n! }{ k! (n-k)! }
				\right) 
				+ k \cdot \log (p) + (n-k) \cdot \log (1-p)
			 \right) \\
			 &= \exp \left(
				 \log \left(
					 n!
				\right) 
				- \log (k! (n-k)!)
				+ k \cdot \log (p) + n \cdot \log (1-p) -
				k \cdot \log (1-p)
			 \right) \\
\end{align*}

\textbf{(2)}

\begin{align*}
	f_{\vartheta , \rho} (x) &= \frac{ \vartheta ^{\rho} }{ \Gamma (\rho) }
	x ^{\rho - 1} e ^{- \vartheta x} \cdot I_{ [ 0, \infty ) } (x) \\
		 &= 
\end{align*}

\end{document}
