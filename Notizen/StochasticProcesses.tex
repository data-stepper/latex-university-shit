\documentclass[a4paper]{article}

\usepackage[margin=1in]{geometry} 
\usepackage{amsmath,amsthm,amssymb, graphicx, multicol, array}

\usepackage{tikz}
\usetikzlibrary{automata, positioning}

\pdfminorversion=7
\pdfsuppresswarningpagegroup=1

\newcommand{\R}{\mathbb{R}}
\newcommand{\N}{\mathbb{N}}
\newcommand{\Z}{\mathbb{Z}}
\newcommand{\beh}{\textit{Behauptung. }}

\setlength{\parindent}{0pt}
\newenvironment{Aufgabe}[2][Aufgabe]{\begin{trivlist}
\item[\hskip \labelsep {\bfseries #1}\hskip \labelsep {\bfseries #2.}]}{\end{trivlist}}

\begin{document}
\title{ \textbf{Cheat Sheet Stochastic Processes} }
\author{Bent Mueller}
\date{04.04.2022}
\maketitle

\section{Kolmogorov Erweiterungssatz}
\[
	\forall J \subset T: \forall B \in \mathcal{B}_s: P(\pi^{-1}_{T, J}(B))
	= P_J (B)
\] 

Means that for every transition $T$ in the state space $T$ and every state $B$ in the set of states $B_s$ the probability of the transition $T$ is the same as the probability of the state $B$.

In other words, the probability of the transition $T$ is the same as the probability of the state $B$.

\end{document}
