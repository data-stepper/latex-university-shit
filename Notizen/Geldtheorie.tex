\documentclass[a4paper]{article}

\usepackage[margin=1in]{geometry}
\usepackage{amsmath,amsthm,amssymb, graphicx, multicol, array}

% Use spacings
\usepackage{setspace}

% And multi cols
\usepackage{multicol}

\usepackage{tikz}
\usetikzlibrary{automata, positioning}

\pdfminorversion=7
\pdfsuppresswarningpagegroup=1

\newcommand{\R}{\mathbb{R}}
\newcommand{\N}{\mathbb{N}}
\newcommand{\Z}{\mathbb{Z}}
\newcommand{\beh}{\textit{Behauptung. }}

\setlength{\parindent}{0pt}
\onehalfspacing

\begin{document}

\tableofcontents
\pagebreak

\section{
  Terminologie
 }

\begin{table}[htpb]
	\centering
	\label{tab:label}
	\begin{tabular}{ll}
		Sichteinlagen   &
		Demand Deposits, Einlagen auf die man schnell zugreifen kann. \\
		Sparbuch        &
		Time (sometimes Term) Deposits                                \\
		Saldo           &
		Differenz (besonders bei Bilanz)                              \\
		Ceteris Paribus &
		Alle anderen Faktoren konstant halten.                        \\
	\end{tabular}
\end{table}

\section{
  Bonds
 }

Der aktuelle Wert einer Anleihe entspricht wegen dem Law of One Price
der Summe der zukünftigen Zahlungen nach diskontiertem Zinssatz.
Diesen Wert nennt man Barwert (engl. Present Value).

Sagen wir Bond A zahlt 5 EUR jedes Jahr und nach 5 Jahren zahlt er 100 EUR
(den Nennwert $N$) zurueck. Falls der Marktzins ($r$) 7\% sind, gilt:

\begin{align*}
	PV(A) =
	\frac{ 5 }{ 1 + r } +
	\frac{ 5 }{ (1 + r)^{2} } +
	\frac{ 5 }{ (1 + r)^{3} } +
	\frac{ 5 }{ (1 + r)^{4} } +
	\frac{ 100 }{ (1 + r)^{5} }
	\approx 88,23 \text{ EUR}
\end{align*}

Und der Bond sollte bei keinem anderen Preis gehandelt werden als dieser.
Andersherum kann man den Marktzins auch aus dem Bondpreis errechnen mit der
selben Formel. Generell gilt:

\[
	PV(B) = \sum_{n=1}^{N} \frac{ i_C }{ \left(
		1 + r
		\right) ^n }
\]

Wobei $i_C$ die Kouponzahlung des Bonds ist. Wichtig ist, dass dieser Preis
zuerst nur fuer Risikofreie Bonds gilt, da ein Risiko den Preis weiter senkt.

\section{
  Bilanzierung
 }

Unterteilung der Bilanz in Aktiva und Passiva:

\begin{itemize}
	\item Aktiva: Mittelverwendung, also alle Gegenstände, in die
	      investiert wurde. Worin liegen die Mittel aktuell? Hat die Firma
	      viel Bargeld, viele Aktien und so weiter.

	\item Passiva: Mittelherkunft, also alle Ursprünge der Mittel.
	      Wichtig sind hier auch die Unterteilung in Fremdkapital, welches
	      zusätzliche Verpflichtungen darstellt, und Eigenkapital.
\end{itemize}

Insgesamt muss die Bilanz aufgehen,
also beide Seiten summieren sich auf das selbe.
Wichtig ist auch, dass jeweils immer die Anfangsbestände eingetragen werden.
Folgende Terme sind wichtig wenn sich die Bilanz aendert.

\begin{itemize}
	\item Aktivtausch: Tausch innerhalb der Aktiva, z.B. Aktien gegen
	      Aktien oder Bargeld gegen Aktien.

	\item Passivtausch: Tausch innerhalb der Passiva, z.B. Kredite gegen
	      Anleihen.

	\item Aktiv-Passiv Mehrung: Erhöhung der Aktiva und Passiva
	      durch Aufnahme von Fremdkapital. Manchmal auch Verlaengerung
	      der Bilanz.

	\item Aktiv-Passiv Minderung, Bilanzverkuerzung: Wie oben nur andersherum.
\end{itemize}

Beispiel: Eine Firma hat 1000 EUR Eigenkapital und 1000 EUR Fremdkapital.
Sie investiert 500 EUR in Aktien, diese steigen um 100 EUR an Wert.

\begin{table}
	\centering
	\label{tab:label}
	\caption{Jahres Anfangsbilanz}
	\begin{tabular}{l|r}
		A                & P                     \\
		\hline                                   \\
		500 EUR Aktien   & 1000 EUR Eigenkapital \\
		1500 Bargeld EUR & 1000 EUR Fremdkapital \\
	\end{tabular}
\end{table}

\begin{table}
	\centering
	\caption{Jahres Endbilanz}
	\label{tab:label}
	\begin{tabular}{l|r}
		A                & P                     \\
		\hline                                   \\
		600 EUR Aktien   & 1100 EUR Eigenkapital \\
		1500 Bargeld EUR & 1000 EUR Fremdkapital \\
	\end{tabular}
\end{table}

Wir sehen, die Aktien haben insgesamt die Bilanz vergrößert, da beide Seiten
um genau die 100 EUR Gewinn gestiegen sind.

\section{
  Formen von Vermögen
 }

\begin{itemize}
	\item Bruttogeldvermögen (BGV): alle Forderungen, die man 'schnell' zu Geld machen
	      kann. Diese werden unterteilt in Zahlungsmittel
	      (Bargeld, Sichteinlagen) und in weitere Geldforderungen.

	\item Sachvermögen (SV): alle denkbaren Sachwerte.

	\item Nettogeldvermögen (NGV): BGV minus alle Verbindlichkeiten (Geldschulden)
	      die gezahlt werden müssen.

	\item Nettovermögen, Reinvermögen (NV): SV + NGV = gesamt Vermögen.
\end{itemize}

Bei gesamtwirtschaftlicher Buchhaltung, betrachten wir generell die
Vermögensbildung, da diese von der Produktivitaet abhaengt.
\\

Vermögensbildungen werden immer mit $\Delta <Vermögen>$ bezeichnet. Also
z.B. $\Delta SV$ fuer die Sachvermögensbildung. Dies meint die Differenz
zwischen dem Anfangs- und Endbestand innerhalb einer Periode des Sachvermögens.

\end{document}
