\documentclass[a4paper]{article}

\usepackage[margin=1in]{geometry}
\usepackage{amsmath,amsthm,amssymb, graphicx, multicol, array}

% Use spacings
\usepackage{setspace}

\usepackage{tikz}
\usetikzlibrary{automata, positioning}

\pdfminorversion=7
\pdfsuppresswarningpagegroup=1

\newcommand{\R}{\mathbb{R}}
\newcommand{\N}{\mathbb{N}}
\newcommand{\Z}{\mathbb{Z}}
\newcommand{\beh}{\textit{Behauptung. }}

\setlength{\parindent}{0pt}
\onehalfspacing

\begin{document}

\tableofcontents
\pagebreak

\section{
  Bonds
 }

Der aktuelle Wert einer Anleihe entspricht wegen dem Law of One Price
der Summe der zukünftigen Zahlungen nach diskontiertem Zinssatz.
Diesen Wert nennt man Barwert (engl. Present Value).

Sagen wir Bond A zahlt 5 EUR jedes Jahr und nach 5 Jahren zahlt er 100 EUR
(den Nennwert $N$) zurueck. Falls der Marktzins ($r$) 7\% sind, gilt:

\begin{align*}
	PV(A) =
	\frac{ 5 }{ 1 + r } +
	\frac{ 5 }{ (1 + r)^{2} } +
	\frac{ 5 }{ (1 + r)^{3} } +
	\frac{ 5 }{ (1 + r)^{4} } +
	\frac{ 100 }{ (1 + r)^{5} }
	\approx 88,23 \text{ EUR}
\end{align*}

Und der Bond sollte bei keinem anderen Preis gehandelt werden als dieser.
Andersherum kann man den Marktzins auch aus dem Bondpreis errechnen mit der
selben Formel. Generell gilt:

\[
	PV(B) = \sum_{n=1}^{N} \frac{ i_C }{ \left(
		1 + r
		\right) ^n }
\]

Wobei $i_C$ die Kouponzahlung des Bonds ist. Wichtig ist, dass dieser Preis
zuerst nur fuer Risikofreie Bonds gilt, da ein Risiko den Preis weiter senkt.

\section{
  Bilanzierung
 }

Unterteilung der Bilanz in Aktiva und Passiva:

\begin{itemize}
	\item Aktiva: Mittelverwendung, also alle Gegenstände, in die
	      investiert wurde. Worin liegen die Mittel aktuell? Hat die Firma
	      viel Bargeld, viele Aktien und so weiter.

	\item Passiva: Mittelherkunft, also alle Ursprünge der Mittel.
	      Wichtig sind hier auch die Unterteilung in Fremdkapital, welches
	      zusätzliche Verpflichtungen darstellt, und Eigenkapital.
\end{itemize}

Insgesamt muss die Bilanz aufgehen,
also beide Seiten summieren sich auf das selbe.
Wichtig ist auch, dass jeweils immer die Anfangsbestände eingetragen werden.
Folgende Terme sind wichtig wenn sich die Bilanz aendert.

\begin{itemize}
	\item Aktivtausch: Tausch innerhalb der Aktiva, z.B. Aktien gegen
	      Aktien oder Bargeld gegen Aktien.

	\item Passivtausch: Tausch innerhalb der Passiva, z.B. Kredite gegen
	      Anleihen.

	\item Aktiv-Passiv Mehrung: Erhöhung der Aktiva und Passiva
	      durch Aufnahme von Fremdkapital.
\end{itemize}

\section{
  Formen von Vermögen
 }

Terminologie: Sichteinlagen = Demand Deposits und
Sparbuch = Time (oder manchmal Term) Deposits.

\begin{itemize}
	\item Bruttogeldvermögen (BGV): alle Forderungen, die man 'schnell' zu Geld machen
	      kann. Diese werden unterteilt in Zahlungsmittel
	      (Bargeld, Sichteinlagen) und in weitere Geldforderungen.

	\item Sachvermögen (SV): alle denkbaren Sachwerte.

	\item Nettogeldvermögen (NGV): BGV minus alle Verbindlichkeiten (Geldschulden)
	      die gezahlt werden müssen.

	\item Nettovermögen, Reinvermögen (NV): SV + NGV = gesamt Vermögen.
\end{itemize}

\end{document}
