\documentclass[a4paper]{article}

\usepackage[margin=1in]{geometry}
\usepackage{amsmath,amsthm,amssymb, graphicx, multicol, array}

% Use spacings
\usepackage{setspace}

% And multi cols
\usepackage{multicol}

\usepackage{tikz}
\usetikzlibrary{automata, positioning}

\pdfminorversion=7
\pdfsuppresswarningpagegroup=1

\newcommand{\R}{\mathbb{R}}
\newcommand{\N}{\mathbb{N}}
\newcommand{\Z}{\mathbb{Z}}
\newcommand{\beh}{\textit{Behauptung. }}

\setlength{\parindent}{0pt}
\onehalfspacing

\begin{document}

\tableofcontents
\pagebreak

\section{
  Terminologie
 }

\begin{table}[htpb]
	\centering
	\label{tab:label}
	\begin{tabular}{ll}
		Sichteinlagen   &
		Demand Deposits, Einlagen auf die man schnell zugreifen kann. \\
		Sparbuch        &
		Time (sometimes Term) Deposits                                \\
		Saldo           &
		Differenz (besonders bei Bilanz)                              \\
		Ceteris Paribus &
		Alle anderen Faktoren konstant halten.                        \\
	\end{tabular}
\end{table}

\section{
  Bonds
 }

Der aktuelle Wert einer Anleihe entspricht wegen dem Law of One Price
der Summe der zukünftigen Zahlungen nach diskontiertem Zinssatz.
Diesen Wert nennt man Barwert (engl. Present Value).

Sagen wir Bond A zahlt 5 EUR jedes Jahr und nach 5 Jahren zahlt er 100 EUR
(den Nennwert $N$) zurueck. Falls der Marktzins ($r$) 7\% sind, gilt:

\begin{align*}
	PV(A) =
	\frac{ 5 }{ 1 + r } +
	\frac{ 5 }{ (1 + r)^{2} } +
	\frac{ 5 }{ (1 + r)^{3} } +
	\frac{ 5 }{ (1 + r)^{4} } +
	\frac{ 100 }{ (1 + r)^{5} }
	\approx 88,23 \text{ EUR}
\end{align*}

Und der Bond sollte bei keinem anderen Preis gehandelt werden als dieser.
Andersherum kann man den Marktzins auch aus dem Bondpreis errechnen mit der
selben Formel. Generell gilt:

\[
	PV(B) = \sum_{n=1}^{N} \frac{ i_C }{ \left(
		1 + r
		\right) ^n }
\]

Wobei $i_C$ die Kouponzahlung des Bonds ist. Wichtig ist, dass dieser Preis
zuerst nur fuer Risikofreie Bonds gilt, da ein Risiko den Preis weiter senkt.

\section{
  Bilanzierung
 }

Unterteilung der Bilanz in Aktiva und Passiva:

\begin{itemize}
	\item Aktiva: Mittelverwendung, also alle Gegenstände, in die
	      investiert wurde. Worin liegen die Mittel aktuell? Hat die Firma
	      viel Bargeld, viele Aktien und so weiter.

	\item Passiva: Mittelherkunft, also alle Ursprünge der Mittel.
	      Wichtig sind hier auch die Unterteilung in Fremdkapital, welches
	      zusätzliche Verpflichtungen darstellt, und Eigenkapital.
\end{itemize}

Insgesamt muss die Bilanz aufgehen,
also beide Seiten summieren sich auf das selbe.
Wichtig ist auch, dass jeweils immer die Anfangsbestände eingetragen werden.
Folgende Terme sind wichtig wenn sich die Bilanz aendert.

\begin{itemize}
	\item Aktivtausch: Tausch innerhalb der Aktiva, z.B. Aktien gegen
	      Aktien oder Bargeld gegen Aktien.

	\item Passivtausch: Tausch innerhalb der Passiva, z.B. Kredite gegen
	      Anleihen.

	\item Aktiv-Passiv Mehrung: Erhöhung der Aktiva und Passiva
	      durch Aufnahme von Fremdkapital. Manchmal auch Verlaengerung
	      der Bilanz.

	\item Aktiv-Passiv Minderung, Bilanzverkuerzung: Wie oben nur andersherum.
\end{itemize}

Beispiel: Eine Firma hat 1000 EUR Eigenkapital und 1000 EUR Fremdkapital.
Sie investiert 500 EUR in Aktien, diese steigen um 100 EUR an Wert.

\begin{table}
	\centering
	\label{tab:label}
	\caption{Jahres Anfangsbilanz}
	\begin{tabular}{l|r}
		A                & P                     \\
		\hline                                   \\
		500 EUR Aktien   & 1000 EUR Eigenkapital \\
		1500 Bargeld EUR & 1000 EUR Fremdkapital \\
	\end{tabular}
\end{table}

\begin{table}
	\centering
	\caption{Jahres Endbilanz}
	\label{tab:label}
	\begin{tabular}{l|r}
		A                & P                     \\
		\hline                                   \\
		600 EUR Aktien   & 1100 EUR Eigenkapital \\
		1500 Bargeld EUR & 1000 EUR Fremdkapital \\
	\end{tabular}
\end{table}

Wir sehen, die Aktien haben insgesamt die Bilanz vergrößert, da beide Seiten
um genau die 100 EUR Gewinn gestiegen sind.

\subsection{
	Veraenderungen in der Bilanz
}

\begin{itemize}
	\item Einzahlung / Auszahlung $\Leftrightarrow$ Zahlungsmittel (e.g. Bargeld)
	\item Einnahmen / Ausgaben $\Leftrightarrow$ Nettogeldvermoegen (Aktien, Anleihen, ...)
	\item Ertrag (= Einkommen) / Aufwand (= Konsum) $\Leftrightarrow$
	      Nettogeldvermoegen / Reinvermoegen
\end{itemize}

Nettogeldvermoegen ist \textbf{global} gleich 0.

\section{Formen von Vermögen}

\begin{itemize}
	\item Bruttogeldvermögen (BGV): alle Forderungen, die man 'schnell' zu Geld machen
	      kann. Diese werden unterteilt in Zahlungsmittel
	      (Bargeld, Sichteinlagen) und in weitere Geldforderungen.

	\item Sachvermögen (SV): alle denkbaren Sachwerte.

	\item Nettogeldvermögen (NGV): BGV minus alle Verbindlichkeiten (Geldschulden)
	      die gezahlt werden müssen.

	\item Nettovermögen, Reinvermögen (NV): SV + NGV = gesamt Vermögen.
\end{itemize}

Bei gesamtwirtschaftlicher Buchhaltung, betrachten wir generell die
Vermögensbildung, da diese von der Produktivitaet abhaengt.
\\

Verm\"ogensbildungen werden immer mit $\Delta$ Verm\"ogen bezeichnet. Also
z.B. $\Delta SV$ fuer die Sachvermögensbildung (welche durchaus auch negativ
sein kann). Dies meint die Differenz
zwischen dem Anfangs- und Endbestand innerhalb einer Periode des Sachvermögens.

\section{Clearingsysteme}

\textbf{Clearing} nennt man den Prozess der digitalen Schuldenbegleichung
mit Hilfe dritter (insbesondere dem Clearing Haus).
Banken leihen sich untereinander Geld. \\

Vorteile eines multilateralen Clearing Haus:

\begin{itemize}
	\item Zahlungsschwierigkeiten werden von allen Mitgliedern (Banken) getragen
	      und nicht nur vom Handelspartner wie beim bilateralen Clearing. Dadurch
	      kann auch mit aussenstehenden gehandelt werden, da nun mehr Sicherheiten
	      vorhanden sind.

	\item Die Banken haben eine bessere Liquiditaet, da sie nicht nur auf ihre
	      eigenen Reserven angewiesen sind, sondern sich einfach weitere
	      Kreditzertifikate ausgeben lassen koennen.

	\item Alle Mitglieder werden durch das Clearing Haus diszipliniert, da
	      haufiges Ueberziehen teuer wird fuer die Banken.

	\item Es werden weniger Transaktionen durchgefuehrt, da das Clearing Haus
	      direkt netting durchfuehrt.
\end{itemize}

\subsection{Federal Reserve}

Die Banken muessen ihre Mindestreserve halten, falls ein Bank Run auftreten
sollte. Die Mindestreserve war frueher bei 25\% der Einlagen gesetzt. Eine
feste Mindestreserve ist aber problematisch, da dann die Zinsen periodisch
schwanken sobald mehr Nachfrage nach Krediten besteht. \\

Die FED dient also tatsaechlich als 'Reserve' der Banken, undzwar genau dann
wenn die Nachfrage nach Krediten steigt. Banken koennen sich bei der FED sog.
FR (Federal Reserve Notes) ausleihen. Aus diesem Grund wird das Clearing
heutzutage auch ueber die Zentralbank abgewickelt.

Schema:

\[
	\text{Kunde}
	\overset{\text{nimmt Kredit auf}} \longrightarrow
	\text{Bank}
	\overset{\text{leiht sich Reserven (Innertages kredit oder overnight)}}
	\longrightarrow
	\text{Zentralbank}
\]

Rechteres passiert allerdings nur falls die Bank nicht genug Reserveguthaben
bereitstehend hat, ansonsten kann sie auch direkt,
solange sie noch die gesetzliche Mindestreserve erfuellt,
die Einlagen der anderen Kunden als Kredite verleihen. \\

$\Rightarrow$ Die Zentralbank kann die Zinsen in einem 'Korridor' halten.

\subsection{Einfaches Geldmarktmodell}

\begin{itemize}
	\item $MR$ Mindestreserve, $r = \frac{ MR }{ D }$ Mindestreservesatz (Anteil)
	\item $D$ Einlagen
	\item $B$ Bargeld
	\item $V$ Summe Privatsektor, sodass $V = D + B$
	\item $b = \frac{ B }{ V }$, Anteil an Bargeld im Privatsektor
	\item $i_M$ Zinsen auf Kundeneinlagen
	\item $i_L$ Zinsen auf Unternehmenskredite
	\item $i_{ZB}$ Zinsen der Zentralbank
\end{itemize}

\[
	\Rightarrow
	D = V \cdot (1 - b),
	B = V \cdot b,
	MR = r \cdot D
\]

Banken halten nur die Mindestreserve, welche nicht verzinst wird.
Berechnen wir nun $i_L$, die Zinsen auf Unternehmenskredite:

\[
	i_L (D + B) = \text{Profit der Bank}
	\overset{!} =
	\text{Kosten der Bank}
	=
	i_M D + i_{ZB} (B + MR)
\]

Oberes folgt direkt sobald man die Bilanzen von Zentralbank und Geschaeftsbank
aufschreibt aus der doppelten Buchfuehrung (Summe Aktiva = Summe Passiva).

\[
	\Rightarrow i_L = \frac{ i_M D + i_{ZB} (B + MR) }{ D + B } =
	i_M \frac{ D }{ V } + i_{ZB} \frac{ rD + bV }{ V } =
	i_{ZB} \left(
	r (1-b) + b
	\right) + i_M \left(
	1 - b
	\right)
\]

Also sehen wir, in diesem Modell haengen die Zinsen auf Unternehmenskredite
garnicht von den Geldmengen oder der Mindestreserve ab, sondern nur von dessen
Anteilen jeweils am Privatsektor und natuerlich von den exogen gegebenen Zinsen. \\

Mit den partiellen Ableitungen wissen wir dann folgendes:

\begin{itemize}
	\item $\frac{ \partial i_L }{ \partial r } = i_{ZB} (1 - b)$
	      ,also ein hoeherer Mindestreservesatz fuehrt direkt
	      zu hoeheren Zinsen auf Unternehmenskredite. Und

	\item $\frac{ \partial i_L }{ \partial b } = -i_M + (1 - r) i_{ZB}$
	      dies koennen wir wie folgt erklaeren. Wenn die Bargeldhaltung ($b$)
	      steigt, dann steigen die Zinsen weil die Banken Geld zum $i_{ZB}$
	      Zins leihen muessen um die Kredite zu vergeben. Aber gleichzeitig
	      faellt der Zins undzwar genau um die Zinsen die sonst an die Kunden
	      wegen ihren Einlagen gezahlt werden wuerden, der Zins faellt aber auch
	      noch, da die Mindestreserve auf die Kundeneinlagen faellt also muss die
	      Bank sich hier auch weniger Geld leihen.
\end{itemize}

Genauer sehen wir:
\[
	\frac{ \partial i_L }{ \partial b } > 0 \Leftrightarrow
	(1 - r) i_{ZB} > i_M
\]

Mit Gewinnabschlag auf den Kundenzins (Uebung 2, 5.b),
aendert sich das Modell dann mit
\[
	i_M = (1 - \mu_M) i_{ZB}, \text{ mit } 1 > \mu_M > 0
\]

Und wir erhalten:

\[
	\frac{ \partial i_L }{ \partial b } =
	- i_{ZB} (1 - \mu_M) + (1 - r) i_{ZB} =
	(\mu_M - r) i_{ZB}
\]

Dann gilt aber:

\[
	\frac{ \partial i_L }{ \partial b } > 0 \Leftrightarrow
	(1 - r) i_{ZB} > (1 - \mu_M) i_{ZB} \overset{i_{ZB} \neq 0} \Leftrightarrow
	\mu_M > r
\]

In diesem Fall erhoeht die Bargeldhaltung den Unternehmenszins, genau dann wenn
der Gewinnabschlag auf den Kundenzins groesser ist als der Mindestreservesatz,
was in der Regel nicht der Fall ist.

c)

\[
	i_L \left(
	D + B
	\right) =
	i_M D + i_{ZB} B \Rightarrow
	i_L = \frac{ i_M D + i_{ZB} B }{ D + B } =
	i_M \frac{ D }{ V } + i_{ZB} \frac{ bV }{ V } =
	i_{ZB} b + i_M (1 - b)
\]

\[
	\Rightarrow \frac{ \partial i_L }{ \partial b } =
	i_{ZB} - i_M \text{ und }
	\frac{ \partial i_L }{ \partial r } = 0
\]

Aufgabe 6 (a):

\begin{enumerate}
	\item Berechne den aktuellen Marktwert (nach Haircut) der Staatsanleihen wie folgt:
	      \[
		      500.000 EUR \cdot 0,99 \cdot 0,98 = 485.100 EUR
	      \]
	      Soviel Geld bekommt die Bank fuer die Staatsanleihen als Kredit.
	\item Berechne wie viel auf die 1 Mio fehlt:
	      \[
		      1.000.000 EUR - 485.100 EUR = 514.900 EUR
	      \]
	\item Die Bank braucht Unternehmensanleihen in Hoehe der fehlenden Summe:
	      \[
		      514.900 EUR \div 0,97 \div 0,95 = 558.763 EUR
	      \]
	      Also werden Anleihen im Gesamt\textbf{nenn}wert von
	      \[
		      500.000 EUR + 558.763 EUR = 1.058.763 EUR
	      \]
	      hinterlegt.
\end{enumerate}

b) Aktuelle Marktwert der hinterlegten Anleihen:

\[
	558.763 EUR \cdot 0,97 + 500.000 EUR \cdot 0,99 = 1.037.000,11 EUR
\]

c) Zinszahlungen fuer die 1 Mio bei einem Uebrnachtkredit von 2 \% und 360 Tagen
im Jahr:

\[
	1.000.000 \cdot \frac{ 0,02 }{ 360 } \approx 55,56 EUR
\]

$\Rightarrow$ Die Zinsen werden hier einfach \textbf{linear} auf Tagesbasis skaliert.

\subsection{Geldschoepfungsmultiplikator}

Geldschoepfungsmultiplikator: Wie viel Geld wird durch die
Kreditvergabe der Banken geschaffen?

Wenn der Mindestreservesatz $r$ ist, dann ist der Geldschoepfungsmultiplikator $m = \frac{1}{r} > 0$.

\section{Laufzeitrendite (Term yield)}

Angenommen, es werden zwei (risikofreie) Bonds gehandelt auf dem Markt. Beide sind
Nullkuponanleihen mit gleichem Nennwert. Allerdings hat Bond A eine Laufzeit von
einem Jahr und Bond B eine Laufzeit von zwei Jahren. \\

Kauft man nun Bond B und wartet ein Jahr, dann muss gelten, dass Bond B genau
den gleichen Marktwert haben muss wie Bond A mit einer Restlaufzeit von einem Jahr.

\end{document}
