\documentclass[a4paper]{article}

\usepackage[margin=1in]{geometry} 
\usepackage{amsmath,amsthm,amssymb, graphicx, multicol, array}

\usepackage{tikz}
\usetikzlibrary{automata, positioning}

\pdfminorversion=7
\pdfsuppresswarningpagegroup=1

\newcommand{\R}{\mathbb{R}}
\newcommand{\N}{\mathbb{N}}
\newcommand{\Z}{\mathbb{Z}}
\newcommand{\beh}{\textit{Behauptung. }}

\setlength{\parindent}{0pt}

\begin{document}

    \tableofcontents
    \pagebreak

    \section{
        Bonds
    }

    Always calculate the present value (Barwert) of a bond (Anleihe) by 
    discounting its future cash flows.

    As an example, lets say Bond A pays 50 EUR every year and after 2 years it matures,
    paying back 100 EUR.

    \begin{align*}
        PV(A) = 
    \end{align*}

\end{document}