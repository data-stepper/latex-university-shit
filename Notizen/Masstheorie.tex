\documentclass[a4paper]{article}

\usepackage[margin=1in]{geometry} 
\usepackage{amsmath,amsthm,amssymb, graphicx, multicol, array}

\pdfminorversion=7
\pdfsuppresswarningpagegroup=1

\newcommand{\R}{\mathbb{R}}
\newcommand{\N}{\mathbb{N}}
\newcommand{\Z}{\mathbb{Z}}
\newcommand{\beh}{\textit{Behauptung. }}

\theoremstyle{definition}

% Use this if \subsection's are used instead of \section
% for enumerating contents
% \newtheorem{theorem}{Satz}[subsection]
% \newtheorem{theorem}{Satz}[section]
% \newtheorem{definition}[theorem]{Definition}
% \newtheorem{lemma}[theorem]{Lemma}
% \newtheorem{example}[theorem]{Beispiel}

\setlength{\parindent}{0pt}
\newenvironment{Aufgabe}[2][Aufgabe]{\begin{trivlist}
\item[\hskip \labelsep {\bfseries #1}\hskip \labelsep {\bfseries #2.}]}{\end{trivlist}}

\begin{document}
\title{ \textbf{Maßtheorie - Notizen} }
\author{Bent Müller}
\date{15. April 2021}
\maketitle
\tableofcontents
\pagebreak

\section{Maßräume}

\subsection{Maße und Mengensysteme}

\begin{theorem} % Sigma-Algebra
	\subsubsection{$\sigma$-Algebra}

	Sei $\Omega$ ein Grundraum.
	Wir nennen ein Mengensystem $A \subset 2^{\Omega}$ eine $\sigma$-Algebra 
	falls es folgende Eigenschaften erfüllt:
	\begin{itemize}
		\item[(i)] $\Omega \in A$
		\item[(ii)] $B \in A$ \Rightarrow $B^{c} \in A$
		\item[(iii)] $B_n \in A \forall n \in \N$ \; \Longrightarrow \;
			$\bigcup_{n\in \N} B_n \in A$
	\end{itemize}

	Dann nennen wir das Tupel $(\Omega, A)$ einen Messraum (aufpassen
	Messraum $\neq$ Maßraum).
\end{theorem}

\begin{theorem} % Maß
	\subsubsection{Maß}
	Sei $(\Omega, A)$ ein Messraum, so definieren wir eine Abbildung
	$\mu : A \longrightarrow [0, \infty]$ und nennen diese ein \textbf{Maß},
	falls sie folgende Eigenschaften besitzt:
	\begin{itemize}
		\item[(i)] $\mu (\emptyset) = 0$
		\item[(ii)] Für alle disjunkten Mengen $B_n \in A, n \in \N$ gilt
			$\mu \left( 
				\bigcup_{n\in \N} B_n = \sum_{n\in \N} \mu (B_n)
			\right) $
	\end{itemize}

	Jetzt nennen wir das Tupel $(\Omega, A, \mu)$ einen Maßraum.
\end{theorem}

\begin{theorem} % Erzeugenden-System
	\subsubsection{Erzeugenden-System}
	$C \subset 2 ^{\Omega}$ wir nennen das Mengensystem
	\[
		\sigma(C) := \underset{A \subset 2^{\Omega}, A \; \sigma-Algebra, C \subset A}
		\bigcap A
	\] 
	die kleinste $C$ umfassende $\sigma$-Algebra oder auch die von $C$ erzeugte
	$\sigma$-Algebra. Somit nennen wir $C$ ein Erzeugenden-System von $\sigma(C)$.
\end{theorem}

\end{document}
