\documentclass[a4paper]{article}

\usepackage[margin=1in]{geometry} 
\usepackage{amsmath,amsthm,amssymb, graphicx, multicol, array}


\pdfminorversion=7
\pdfsuppresswarningpagegroup=1

\newcommand{\R}{\mathbb{R}}
\newcommand{\N}{\mathbb{N}}
\newcommand{\Z}{\mathbb{Z}}
\newcommand{\beh}{\textit{Behauptung. }}

\setlength{\parindent}{0pt}
\newenvironment{Aufgabe}[2][Aufgabe]{\begin{trivlist}
\item[\hskip \labelsep {\bfseries #1}\hskip \labelsep {\bfseries #2.}]}{\end{trivlist}}

\begin{document}

\section{Dichten}
\subsection{Die Exponentielle Familie}

\begin{theorem} % Exponentielle Familie
	\subsubsection{Exponentielle Familie}

	Wir nennen eine Menge an Dichtefunktionen eine Exponentielle Familie genau dann wenn
	es Abbildungen $Q_1, ..., Q_k, D \; : \Theta \to \mathbb{R}$ und Borel-messbare
	Abbildungen $T_1, \ldots, T_k, S \; : \mathbb{R} ^{n} \to \mathbb{R}$ gibt, sodass wir
	jede Dichte schreiben können als:
	\[
		f_{\vartheta} (x) = \exp \left(
			\sum_{i=1}^{k} Q_i (\vartheta) T_i (x) + D(\vartheta) + S(x)
		\right) 
	\]
	Man kann außerdem zeigen, dass nun die $T_i$ ebenfalls eine exponentielle Familie bilden.
	Außerdem gelten die Aussagen aus 2.3 und 2.4 auch für k-parametrige exp. Familien.

	Wir können die Darstellung auch umschreiben wie folgt:
	\[
		f_\vartheta (x) = c(\vartheta) h(x) \exp(\sum_{i=1}^{k} Q_i (\vartheta) T_i (x))
	\] 
	Im stetigen Fall können wir $c (\vartheta)$ als den Normierungsfaktor
\end{theorem}

\section{Stichproben}

\begin{theorem} % Stichprobenmomente
	\subsubsection{Stichprobenmomente}

	Im Folgenden betrachten wir immer eine Zufallsvariable $X$ mit einer unbekannten
	Dichte (oder Verteilung).
\end{theorem}

\end{document}

