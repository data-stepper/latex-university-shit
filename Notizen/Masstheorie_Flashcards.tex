\documentclass[a4paper]{article}

\usepackage[paperwidth=.5\paperwidth,paperheight=.25\paperheight]{geometry}
\usepackage{pgfpages}
\pagestyle{empty}
\thispagestyle{empty}
\pgfpagesuselayout{8 on 1}[a4paper]
\makeatletter
\@tempcnta=1\relax
\loop\ifnum\@tempcnta<9\relax
\pgf@pset{\the\@tempcnta}{bordercode}{\pgfusepath{stroke}}
\advance\@tempcnta by 1\relax
\repeat
\makeatother

\usepackage[utf8]{inputenc}
\usepackage{amsfonts}
\usepackage{amsmath}

\begin{document}
\Big{
	% BEGIN FLASHCARD PAGE

	{
		\noindent \textsc{
			Definition 1.1
		}
		\vfill
		\centerline{{
			$\sigma$-Algebra
		}}
		\vfill
	}
	\newpage

	{
		\noindent \textsc{
			Begriff
		}
		\vfill
		\centerline{{
			Messraum und messbare Menge
		}}
		\vfill
	}
	\newpage

	{
		\noindent \textsc{
			Definition 1.2
		}
		\vfill
		\centerline{{
			Maß
		}}
		\vfill
	}
	\newpage

	{
		\noindent \textsc{
			Begriff
		}
		\vfill
		\centerline{{
			Maßraum
		}}
		\vfill
	}
	\newpage

	{
		\noindent \textsc{
			Definition
		}
		\vfill
		\centerline{{
			Wahrscheinlichkeitsmaß und
			-raum
		}}
		\vfill
	}
	\newpage

	{
		\noindent \textsc{
			Satz 1.3,
			Eigenschaften eines Maßraumes
		}
		\vfill
		\begin{gather*}
			(\Omega, A, \mu) \text{ ist Maßraum } \\
			\\
			\forall B_1 , ..., B_n \in A
			\text{ disjunkt }:
		\end{gather*}
		\vfill
	}
	\newpage

	{
		\noindent \textsc{
			Satz 1.3,
			Eigenschaften eines Maßraumes
		}
		\vfill
		\begin{gather*}
			(\Omega, A, \mu) \text{ ist Maßraum } \\
			\\
			\forall B, C \in A \\
			B \subset C \\
			\mu (B) < \infty :
		\end{gather*}
		\vfill
	}
	\newpage

	{
		\noindent \textsc{
			Satz 1.3,
			Eigenschaften eines Maßraumes
		}
		\vfill
		\begin{gather*}
			(\Omega, A, \mu) \text{ ist Maßraum } \\
			\\
			\forall B, C \in A \\
			\mu \left(
				B \cap C
			\right) < \infty :
		\end{gather*}
		\vfill
	}
	\newpage

	{
		\vspace*{\stretch{1}}
		\begin{gather*}
			(\Omega, A) \text{ heisst Messraum } \\
			\Leftrightarrow \\
			\Omega \text{ Grundraum } \\
			A \text{ ist $\sigma$-Algebra über $\Omega$ } \\
			\text{ und }\\
			\text{ Die } B \in A \text{ heissen messbare Mengen }
		\end{gather*}
		\vspace*{\stretch{1}}
	}
	\newpage

	{
		\vspace*{\stretch{1}}
		\begin{gather*}
			\Omega \neq \emptyset,
			A
			\subset 2 ^{\Omega}: \\
			A \text{ Mengensystem heisst $\sigma$-Algebra (über $\Omega$) } \\
			\Leftrightarrow \\
			\begin{align*}
				&(i) \; \Omega \in A \\
				&(ii) \; B \in A \Rightarrow B ^{c} \in A \\
				&(iii) \; \forall n \in \mathbb{N}: B_n \in A
				\Rightarrow \bigcup_{n \in \mathbb{N}}  B_n \in A
			\end{align*}
		\end{gather*}
		\vspace*{\stretch{1}}
	}
	\newpage

	{
		\vspace*{\stretch{1}}
		\begin{gather*}
			(\Omega, A, \mu) \text{ heisst Maßraum } \\
			\Leftrightarrow \\
			A \text{ ist $\sigma$-Algebra über $\Omega$ } \\
			\mu \text{ ist Maß auf $A$ }
		\end{gather*}
		\vspace*{\stretch{1}}
	}
	\newpage

	{
		\vspace*{\stretch{1}}
		\begin{gather*}
			(\Omega, A) \text{ Messraum, }
			\mu: A \to [0, \infty] \\
			\mu \text{ heisst Maß } \\
			\Leftrightarrow \\
			\begin{align*}
				&(i) \; \mu (\emptyset) = 0 \text{ (Nulltreue) } \\
				&(ii) \; \forall B_n \in A \text{ disjunkt }:
				\mu \left(
					\bigcup_{n \in \mathbb{N}}  B_n
				\right) = \sum_{n \in \mathbb{N}} \mu (B_n) \\
				&\qquad \qquad \text{ ($\sigma$-Additivität) }
			\end{align*}
		\end{gather*}
		\vspace*{\stretch{1}}
	}
	\newpage

	{
		\vspace*{\stretch{1}}
		\begin{gather*}
			\bigcup_{i=1}^{n} B_i \in A,
			\\
			\mu \left(
				\bigcup_{i=1}^{n} B_i
			\right) = 
			\sum_{i=1}^{n} \mu (B_i)
			\\
			\\
			\text{ $\rightsquigarrow$ für endlich viele! }
		\end{gather*}
		\vspace*{\stretch{1}}
	}
	\newpage

	{
		\vspace*{\stretch{1}}
		\begin{gather*}
			(\Omega, A, \mu) \text{ Maßraum }: \\
			\mu \text{ heisst Wahrscheinlichkeitsmaß } \\
			\Leftrightarrow \\
			\mu (\Omega) = 1 \\
			\text{ und } \\
			(\Omega, A, \mu) \text{ heisst Wahrscheinlichkeitsraum }
		\end{gather*}
		\vspace*{\stretch{1}}
	}
	\newpage

	{
		\vspace*{\stretch{1}}
		\begin{gather*}
			\mu \left(
				B \cup C
			\right) = \mu (B) + \mu (C) - \mu \left(
				B \cap C
			\right) 
		\end{gather*}
		\vspace*{\stretch{1}}
	}
	\newpage

	{
		\vspace*{\stretch{1}}
		\begin{gather*}
			C \setminus B \in A \\
			\\
			\mu \left(
				C \setminus B
			\right) =
			\mu (C) - \mu (B)
		\end{gather*}
		\vspace*{\stretch{1}}
	}
	\newpage

	% END FLASHCARD PAGE
}
\end{document}
