\documentclass[a4paper]{article}

\usepackage[margin=1in]{geometry} 
\usepackage{amsmath,amsthm,amssymb, graphicx, multicol, array}


\pdfminorversion=7
\pdfsuppresswarningpagegroup=1

\newcommand{\R}{\mathbb{R}}
\newcommand{\N}{\mathbb{N}}
\newcommand{\Z}{\mathbb{Z}}
\newcommand{\beh}{\textit{Behauptung. }}

\setlength{\parindent}{0pt}

\begin{document}
	\section{Die Markov-Eigenschaft:}
	Die Markov-Eigenschaft beschreibt eine wichtige Klasse stochastischer Prozesse
	die wir in oft in der Natur finden werden, diese sind die sogenannten Markovketten.

	\vspace{\baselineskip}
	\textbf{Definition 1.1:}
	Sei $\{
		X_n, \; n \in \mathbb{N}_0
	\} $ ein stochastischer Prozess mit einem abzählbaren Zustandsraum $I$.
	Als erstes fordern wir, dass
	\[
		P(X_0 = i_0, ..., X_n = i_n) > 0
	\]
	für Zustände $i_k \in I$. Dies ist wichtig, sodass später die bedingten Warscheinlichkeiten
	wohldefiniert sind. Dann nennen wir den Prozess $\{
		X_n, \; n \in \mathbb{N}_0 
	\} $ eine Markovkette genau dann wenn dieser die folgende Eigenschaft erfüllt:
	\[
		P(X_{n+1} = i_{n+1} \; | \; X_0 = i_0, ..., X_n = i_n)
		= P(X_{n+1} = i_{n+1} \; | \; X_n = i_n)
	\] 
	Bei einer Markovkette hängt also $P(X_{n+1} = i_{n+1})$ \textbf{nur} von $P(X_{n} = i_{n})$
	und ist unabhängig von $P(X_k = i_k)$ für alle $k < n$.
	\section{Markovketten in der Natur}
	\section{Eigenschaften von Markovketten}
	\section{Homogene Markovketten}
	\section{Absorption}
\end{document}

