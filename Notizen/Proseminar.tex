\documentclass[a4paper]{article}

\usepackage[margin=1in]{geometry} 
\usepackage{amsmath,amsthm,amssymb, graphicx, multicol, array}


\pdfminorversion=7
\pdfsuppresswarningpagegroup=1

\newcommand{\R}{\mathbb{R}}
\newcommand{\N}{\mathbb{N}}
\newcommand{\Z}{\mathbb{Z}}
\newcommand{\beh}{\textit{Behauptung. }}

\setlength{\parindent}{0pt}

\begin{document}
	\section{Die Markov-Eigenschaft:}
	Die Markov-Eigenschaft beschreibt eine wichtige Klasse stochastischer Prozesse
	die wir in oft in der Natur finden werden, diese sind die sogenannten Markovketten.

	\vspace{\baselineskip}
	\textbf{Definition 1.1:}
	Sei $\{
		X_n, \; n \in \mathbb{N}_0
	\} $ ein stochastischer Prozess mit einem abzählbaren Zustandsraum $I$.
	Als erstes fordern wir, dass
	\[
		P(X_0 = i_0, ..., X_n = i_n) > 0
	\]
	für Zustände $i_k \in I$. Dies ist wichtig, sodass später die bedingten Warscheinlichkeiten
	wohldefiniert sind. Dann nennen wir den Prozess $\{
		X_n, \; n \in \mathbb{N}_0 
	\} $ eine Markovkette genau dann wenn dieser die folgende Eigenschaft erfüllt:
	\[
		P(X_{n+1} = i_{n+1} \; | \; X_0 = i_0, ..., X_n = i_n)
		= P(X_{n+1} = i_{n+1} \; | \; X_n = i_n)
	\] 
	Bei einer Markovkette hängt also $P(X_{n+1} = i_{n+1})$ \textbf{nur} von $P(X_{n} = i_{n})$ ab
	und ist unabhängig von $P(X_k = i_k)$ für alle $k < n$.

	\vspace{\baselineskip}
	Wir können die selbe Markov-Eigenschaft aber auch anders und ein wenig abstrakter 
	formulieren mit folgendem Lemma:

	\textbf{Lemma 1.2:} 
	Seien $C_k$ disjunkte Ereignisse mit Vereinigung $C$ und gilt, dass
	$p = P(A \, | \, B \, \cap \, C_k)$ unabhängig von $k$ ist, dann gilt auch:
	\[
		p = P(A \; | \; B \cap C)
	\] 
	\section{Markovketten in der Natur}
	Unter anderem finden wir Markovketten immer dann wenn wir unseren stochastischen Prozess
	umformulieren können, sodass wir \textbf{Summen von unabhängigen Zufallsvariablen} haben.

	\vspace{\baselineskip}
	Nehmen wir also an $X_k$ sind unabhängige Zufallsvariablen. Dann beobachten wir folgendes:
	\[
		\underbrace{\sum_{i=1}^{n} X_i}_{Y_n}
		= \underbrace{\sum_{i=1}^{n-1} X_i}_{Y_{n-1}} + X_n
	\]
	Also sehen wir direkt, dass die neuen Zufallsvariablen $Y_k$ eine Markovkette bilden.
	\section{Eigenschaften von Markovketten}
	\section{Homogene Markovketten}
	\section{Absorption}
	\end{document}

